%%%%%%%%%%%%%%%%%%%%%%%%%%%%
% DOCUMENT CLASS
\documentclass[a4paper,12pt]{report}
%%%%%%%%%%%%%%%%%%%%%%%%%%%%

%%%%%%%%%%%%%%%%%%%%%%%%%%%%
% LINE SPACING
\newcommand{\linespacing}{2}
\renewcommand{\baselinestretch}{\linespacing}
%%%%%%%%%%%%%%%%%%%%%%%%%%%%

%%%%%%%%%%%%%%%%%%%%%%%%%%%%
% BIBLIOGRAPHY STYLE
\usepackage[authoryear]{natbib}
\bibliographystyle{apalike}

% REMOVE COMMA
\setcitestyle{aysep={}} 

% HYPERLINK YEAR
\usepackage{etoolbox}

\makeatletter

% Patch case where name and year are separated by aysep
\patchcmd{\NAT@citex}
  {\@citea\NAT@hyper@{%
     \NAT@nmfmt{\NAT@nm}%
     \hyper@natlinkbreak{\NAT@aysep\NAT@spacechar}{\@citeb\@extra@b@citeb}%
     \NAT@date}}
  {\@citea\NAT@nmfmt{\NAT@nm}%
   \NAT@aysep\NAT@spacechar\NAT@hyper@{\NAT@date}}{}{}

% Patch case where name and year are separated by opening bracket
\patchcmd{\NAT@citex}
  {\@citea\NAT@hyper@{%
     \NAT@nmfmt{\NAT@nm}%
     \hyper@natlinkbreak{\NAT@spacechar\NAT@@open\if*#1*\else#1\NAT@spacechar\fi}%
       {\@citeb\@extra@b@citeb}%
     \NAT@date}}
  {\@citea\NAT@nmfmt{\NAT@nm}%
   \NAT@spacechar\NAT@@open\if*#1*\else#1\NAT@spacechar\fi\NAT@hyper@{\NAT@date}}
  {}{}

\makeatother
%%%%%%%%%%%%%%%%%%%%%%%%%%%%

%%%%%%%%%%%%%%%%%%%%%%%%%%%%
% OTHER FORMATTING/LAYOUT DECLARATIONS
\usepackage[utf8]{inputenc}
\usepackage{microtype}
\usepackage{amssymb,amsmath}
\usepackage{lmodern}
\usepackage{libertine}
\usepackage[libertine]{newtxmath}
\usepackage{inconsolata}
\usepackage[stable]{footmisc}
\usepackage[dvipsnames]{xcolor}
\definecolor{darkblue}{rgb}{0.0,0.0,0.55}
\usepackage{setspace}
\usepackage[top=2cm,bottom=2cm,left=2cm,right=2cm]{geometry}
\usepackage[backref,pagebackref]{hyperref}
\usepackage{graphicx}
\usepackage{float}
\usepackage{rotating}
\usepackage{adjustbox}
\usepackage{pgf}
\usepackage{tikz}
\usetikzlibrary{arrows}
\usetikzlibrary{positioning}
\usepackage{mathtools}
\usepackage{caption}
\usepackage{dcolumn}
\usepackage[UKenglish]{babel}
\usepackage[UKenglish]{isodate}
\cleanlookdateon
\exhyphenpenalty=1000
\hyphenpenalty=1000
\widowpenalty=10000
\clubpenalty=10000
\usepackage[fencedCode,inlineFootnotes,citations,definitionLists,hashEnumerators,smartEllipses,hybrid]{markdown}
\markdownSetup{rendererPrototypes={
   link = {\href{#2}{#1}},
   image = {\begin{figure}[hbt!]
     \centering
     \includegraphics{#3}%
     \ifx\empty#4\empty\else
     \caption{#4}%
     \fi
     \label{fig:#1}%
     \end{figure}}
}}
%%%%%%%%%%%%%%%%%%%%%%%%%%%%

%%%%%%%%%%%%%%%%%%%%%%%%%%%%
% HYPERREF
\renewcommand*{\backref}[1]{}
\renewcommand*{\backrefalt}[4]{%
	\ifcase #1 (Not cited.)%
	\or        Cited on page~#2.%
	\else      Cited on pages~#2.%
	\fi}
\renewcommand{\backreftwosep}{ and~}
\renewcommand{\backreflastsep}{ and~}
\urlstyle{same}  % don't use monospace font for urls

\hypersetup{pdftitle={Essays in Political and Criminal Violence},
	pdfauthor={Danilo Alves M. Freire},
	pdfsubject={Political Economy},
	pdfkeywords={King's College London, Latin America, PhD Thesis, Political Economy, Violence},
	pdfborder={0 0 0},
	breaklinks=true,
	linkcolor=Mahogany,
	citecolor=Mahogany,
	urlcolor=darkblue,
	colorlinks=true}
%%%%%%%%%%%%%%%%%%%%%%%%%%%%

%%%%%%%%%%%%%%%%%%%%%%%%%%%%
% BEGIN DOCUMENT
\begin{document}
%%%%%%%%%%%%%%%%%%%%%%%%%%%%

%%%%%%%%%%%%%%%%%%%%%%%%%%%%
% PREAMBLE: roman page numbering i, ii, iii, ...
\pagenumbering{roman}
%%%%%%%%%%%%%%%%%%%%%%%%%%%%

%%%%%%%%%%%%%%%%%%%%%%%%%%%%
%% TITLE PAGE: 
\thispagestyle{empty}
\begin{center}
\includegraphics[width=6cm]{images/kcl.eps}
\end{center}	
\vskip40mm
\begin{center}

% TITLE
\huge\textbf{Essays on Political and Criminal Violence}
\vskip2mm
% SUBTITLE (optional)
%\LARGE\textit{How it all works}
\vskip5mm

% AUTHOR
\Large Danilo Alves M. Freire
\normalsize
\vfill
\large

% QUALIFICATION
Submitted for the degree of Doctor of Philosophy \\
Department of Political Economy \\
King's College London	\\

% DATE OF SUBMISSION
May 2018
\end{center}	
%%%%%%%%%%%%%%%%%%%%%%%%%%%%

%%%%%%%%%%%%%%%%%%%%%%%%%%%%
% DECLARATIONS
\chapter*{Declaration}
\noindent 
I hereby declare that except where specific reference is made to the work of others, the contents of this thesis are original and have not been submitted in whole or in part for consideration for any other degree or qualification in this, or any other university. This thesis is my own work and contains nothing which is the outcome of work done in collaboration with others, except as specified in the text and Acknowledgements. 

% ADDITIONAL DECLARATIONS HERE (IF ANY)
\vskip20mm
\noindent
Signature:
\vskip10mm
\begin{flushleft}
\includegraphics[scale=.25]{images/sig.pdf}
\end{flushleft}
% AUTHOR
\noindent 
Danilo Alves M. Freire
%%%%%%%%%%%%%%%%%%%%%%%%%%%%

%%%%%%%%%%%%%%%%%%%%%%%%%%%%
% SUMMARY PAGE
\chapter*{Abstract}
\renewcommand{\baselinestretch}{\linespacing}
\small\normalsize

% SUMMARY HERE (300 word limit for most subjects):
This thesis addresses three topics in political and criminal violence. The first essay is an empirical evaluation of a broad set of homicide reduction policies implemented in the state of São Paulo, Brazil. I employ the synthetic control method, a generalisation of differences-in-differences, to compare these measures against an artificial São Paulo. The results indicate a large drop in homicide rates in actual São Paulo when contrasted with the synthetic counterfactual, with about 20,000 lives saved during the period. 

The second essay offers a rational choice account for the Brazil's \textit{jogo do bicho}, or the `animal game', possibly the largest illegal gambling game in the world. I investigate the institutions that have caused the \textit{jogo do bicho}'s notable growth and long-term survival outside the boundaries of the Brazilian law. I show how \textit{bicheiros} or bookmakers promote social order, solve information asymmetries, and reduce negative externalities via costly signalling and the provision of club goods. I also explain the emergence of the informal rules that govern the game as well as their enforcement mechanisms. 

In the third essay, I employ extreme bounds analysis and distributed random forests to identify the key determinants of state-sponsored violence. Although the literature on mass killings has grown significantly over the last decades, it remains unclear whether estimates are robust to different model specifications or which variables accurately predict the onset of genocides and politicides. To address these issues, I test the sensitivity of 40 variables on a sample of 177 countries from 1945 to 2013. The results show that GDP per capita, the post-Cold War period, and stable political regimes are negatively associated with mass killings. In contrast, ethnic diversity, civil wars, and previous political turmoils increase the risk of state-led violence. Mass killings that occur during civil wars are correlated with a different set of covariates, and I find that territorial conflicts and the use of militias are negatively associated with the outcome variable. The random forests estimations indicate that years since the last episode of mass violence, GDP per capita, urban population, ethnic polarisation, the number of military personnel, and democracy make the greatest contribution to the models' out-of-sample predictive power.
%%%%%%%%%%%%%%%%%%%%%%%%%%%%


%%%%%%%%%%%%%%%%%%%%%%%%%%%%
% ACKNOWLEDGEMENTS
\chapter*{Acknowledgements}
\renewcommand{\baselinestretch}{\linespacing}
\small\normalsize

% ACKNOWLEDGEMENTS HERE:
I would like to extend thanks to the many people, in many countries, who so generously contributed to the work presented in this thesis. First and foremost, I want to thank my supervisor, Professor David Skarbek, for the guidance and encouragement he provided me throughout this research. David has been very supportive since the first time we met, and his dynamism, vision, empathy, and great sense of humour have deeply inspired me. I could not have wished for a better supervisor for my PhD studies. Thank you very much.

Similarly, profound gratitude goes to Professor Gabriel Leon, who gave me an incredible amount of support and guidance in all stages of this project. His help in the last semester of my PhD studies was invaluable and I am very thankful for his generosity.

I am heartily grateful to my examiners, Professor Toke Aidt and Professor Graham Denyer Willis, who accepted to dedicate their precious time to evaluate this work.

The Department of Political Economy at King's College London has provided me with a very stimulating environment in what concerns the extraordinary quality of its students and  academic staff, and that experience will leave marks beyond this thesis. I express my thanks to Professor Florian Foos, Professor Rubén Ruiz-Rufino, and all my PhD colleagues for the inspirational discussions and suggestions. David Muchlinski has taught me many things about machine learning and I am very thankful for that. Professors Kostas Matakos, John Meadowcroft, Mark Pennington, and Emily Skarbek have supported me in so many ways that I cannot even count. Thank you very much for the opportunities you provided, for your time and skills, and for having faith in me.

I have been very fortunate to work with bright and motivated co-authors during this and other related projects. I am grateful to Rodolpho Bernabel, Gabriel Cepaluni, Diogo Costa, Guilherme Jardim Duarte, Malte Hendrickx, Robert McDonnell, Umberto Mignozzetti, and Gary Uzonyi for having taught me so much about both scientific research and life in general.

One of the best experiences I have had in the previous years was to become affiliated with the Mercatus Center and the Institute for Humane Studies. There, I have met a community of smart, hard-working and innovative scholars that have in many ways contributed to this research. I would like to express my gratitude to Paul Dragos Aligica, Nigel Ashford, Peter Boettke, Don Boudreaux, Brandon Brice, John Buchmann, Jeff Carroll, Chris Coyne, Courtney Dunn, Luke Foster, Bobbi Herzberg, Marcus Hunt, Nick Jacobs, Isaac Jilbert, Peter Lipsey, Ian Madison, Kelly Nelson, Gabriel Oliva, Raj Patel, Pablo Prieto, Blaž Remic, Virgil Storr, Alienor van den Bosch, Richard Wagner, and Larry White for all the engaging conversations. I have learned so much from you.

Completing this work would have been all the more difficult were it not for the support and constructive criticism provided by my friends. Specially, I would like to thank Veronica Adaeva, André Amaro, Paulo Roberto Araujo, Aline Bagatin, Fábio Barros, Michel Hulmann, Guilherme Kerr, Maurício Pantaleão, Letícia Sene, and the participants of academic seminars at King's College London and the Prometheus Institute in Berlin. I am deeply indebted to them for their insightful comments and hard questions.

I am also sincerely thankful to several present and past members of the Friedrich Naumann Foundation for Freedom. In 2009, I received a scholarship to attend a short course in Germany, and that first experience abroad has transformed my life ever since. The present work is, to a large extent, a result of the push I received from the Naumann Foundation almost 10 years ago. I owe a great debt of gratitude to Cidália Achten, Gulmina Bilal, Rainer Erkers, Beate Forbriger, Verena Gierszewski, Birgit Lamm, Monica Lehmann, Stefan Melnik, and Wulf Pabst for their support. As time goes on, it is easy to see the huge impact they have had on my academic career.

A very special word of thanks goes for my mother Rosali and my grandmother Maria. They have been absolutely wonderful over the years and have gone above and beyond to provide me with a good education. This thesis is dedicated to them. \textit{Nós conseguimos}. 

%%%%%%%%%%%%%%%%%%%%%%%%%%%%

%%%%%%%%%%%%%%%%%%%%%%%%%%%%
% TABLE OF CONTENTS, LISTS OF TABLES & FIGURES
\newpage
{\hypersetup{linkcolor=black}
\pdfbookmark[0]{Contents}{contents_bookmark}
\tableofcontents
\listoftables
\phantomsection
\addcontentsline{toc}{chapter}{List of Tables}
\listoffigures
\phantomsection
\addcontentsline{toc}{chapter}{List of Figures}
}
%%%%%%%%%%%%%%%%%%%%%%%%%%%%

%%%%%%%%%%%%%%%%%%%%%%%%%%%%
% MAIN THESIS TEXT: Arabic page numbering 1, 2, 3, ...
\newpage
\pagenumbering{arabic}
%%%%%%%%%%%%%%%%%%%%%%%%%%%%

%-----------------------------------------------------
% Chapter 1: Introduction
%-----------------------------------------------------

\chapter{Introduction}
\label{chap:intro}

The literature on political and criminal violence has increased exponentially over the last decades. Although interstate wars have traditionally occupied a privileged position in political science, scholars have broadened their scope to include a myriad of hitherto understudied phenomena into their research agendas. Civil wars \citep{collier2004greed,fearon2003ethnicity,kalyvas2006logic}, genocides \citep{mamdani2014victims, power2013problem}, ethnic conflicts \citep{kaufmann1996possible, montalvo2005ethnic, sambanis2001ethnic}, wartime sexual abuse \citep{cohen2013explaining,wood2006variation,wood2009armed}, electoral violence \citep{hoglund2009electoral,wilkinson2006votes}, state-sponsored killings \citep{harff1988toward, krain1997state,krain2005international,uzonyi2014unpacking}, terrorism \citep{de2005quality,bueno2007propaganda,pape2003strategic}, drug-related violence \citep{holmes2006drugs,lessing2015logics,richani2013systems, shirk2010drug}, street gangs \citep{franzese2016youth,jones2009youth,rodgers2006living,sobel1987direct}, and prison gangs \citep{dias2011pulverizaccao,freire2014,skarbek2011governance,skarbek2012prison,skarbek2014social} have recently moved from the margins to the centre stage of the discipline. The present dissertation contributes to this expanding field.

In order to clarify crucial aspects of my research topic, I employ an eclectic combination of research designs. The methods range from qualitative case studies to randomised experiments and machine learning algorithms. This diversity not only reflects the multiple aspects of organised violence, but it is also a pragmatic response to problems which are common in this area, such as incomplete data, reporting bias, and model uncertainty. By using an array of methodological tools, I hope to overcome some of these challenges.

Regarding the geographical scope of this dissertation, two of the following chapters deal with issues of violence in Latin America, specially in Brazil. According to the World Bank, Latin America is home to about 8\% of the global population, yet it accounts for more than 30\% of the world's homicides.\footnote{See: \url{https://goo.gl/d2WC3V}. Access: April 2017.} Moreover, the yearly ranking by the Citizen's Council for Public Security and Criminal Justice (Consejo Ciudadano para la Seguridad Pública y la Justicia Penal), a Mexican non-governmental organisation, shows that 43 of the 50 most violent cities in the world are located in Latin America, including all of the top 10.\footnote{For the complete ranking, see \url{http://www.seguridadjusticiaypaz.org.mx/biblioteca/prensa/summary/6-prensa/239-las-50-ciudades-mas-violentas-del-mundo-2016-metodologia}. Access: February 2018.} Given the severity of violence in the continent, Latin America was expected to be an important part of this study.

Brazil exemplifies many of the challenges of fighting violence in Latin America. Brazil has the highest absolute number of homicides in the world, about 56,000 per year, and the country hosts 19 of the 50 world's deadliest cities according to the above-mentioned ranking \citep{mapa2014, mexico2014,unodc2013}. Homicide rates have increased markedly after democratisation (1985), and whereas the country has tried several policies to reduce violence, the results are yet to be evaluated in a consistent fashion. 

The first chapter attempts to address this issue. Although Brazil remains notably affected by civil violence, the state of São Paulo has made significant inroads into fighting criminality. In the last decade, São Paulo has witnessed a 70\% decline in homicide rates, a result that policy-makers attribute to a series of crime-reducing measures implemented by the state government \citep{goertzel2009,kahn2005papel}. While recent academic studies seem to confirm this downward trend, no estimation of the total impact of state policies on homicide rates currently exists. I fill this gap by employing the synthetic control method \citep{abadie2003,abadie2010,abadie2014}, a generalisation of differences-in-differences \citep{angrist2008mostly,bertrand2004much,imbens2009recent}, to compare these measures against an artificial São Paulo. The results indicate a large drop in homicide rates in actual São Paulo when contrasted with the synthetic counterfactual, with about 20,000 lives saved during the period. The theoretical usefulness of the synthetic control method for public policy analysis, the role of the \textit{Primeiro Comando da Capital}, a local prison gang, as a moderating variable, and the practical implications of the security measures taken by the São Paulo state government are also discussed.

The second chapter offers a rational choice account for Brazil's \textit{jogo do bicho}, or the `animal game', possibly the largest illegal gambling game in the world. The lottery has been running for over 120 years and according to estimations of Fundação Getúlio Vargas, a Brazilian think tank, it profits up to 800 million dollars per year.\footnote{See \url{http://goo.gl/9kNeX8} and \url{http://goo.gl/8FSAZl} (in Portuguese). Access: April 2017} The \emph{jogo do bicho} has exerted a significant impact on the Brazilian society. The lottery has been a major sponsor of the Carnival Parade in Rio de Janeiro, which is among the world's most famous popular festivals, and it has remained an important driver of state corruption in the country \citep{bezerra2009mecenato,chazkel2011laws,da1999aguias,labronici2012paratodos,magalhaes2005ganhou,soares1993jogo}. I investigate the institutions that have caused the \emph{jogo do bicho}'s notable growth and long-term survival outside the boundaries of the Brazilian law. I show how \textit{bicheiros} or bookmakers promote social order, solve information asymmetries, and reduce negative externalities via costly signalling and the provision of club goods. I also explain the emergence of the informal rules that govern the game as well as their enforcement mechanisms.



%-----------------------------------------------------
% Chapter 2: Synthetic Control
%-----------------------------------------------------

\input{chapters/02-synth}

%-----------------------------------------------------
% Chapter 3: Jogo do Bicho
%-----------------------------------------------------

\chapter{Beasts of Prey or Rational Animals? Private Governance in Brazil's \emph{Jogo do Bicho}}
\label{chap:bicho}

\section{Introduction}
\label{sec:intro}

In 1892, Baron João Batista de Viana Drummond came up with a new idea to fund his cash-strapped zoo. Situated in a quiet neighbourhood in the north of Rio de Janeiro, the \emph{Jardim Zoológico,} or Zoological Garden, hosted a variety of exotic species and offered breath-taking views of the city. But it lacked visitors. As an experienced businessman, Drummond soon realised the zoo would have to provide other kinds of entertainment to keep itself afloat. Among his suggestions, one seemed particularly promising: a lottery raffle.

The rules were straightforward. In the morning, the Baron would choose one animal from a list of 25 beasts and put its picture inside a wooden box at the zoo's entrance. Visitors who wanted to join the raffle received a ticket bearing the stamp of one of those 25 animals.\footnote{At first, the zoo staff distributed the tickets at random, making the game similar to a common raffle. But it did not take long until visitors could name their animals of choice. This change made the game considerably more appealing and lasts until this day \citep[71--74]{da1999aguias}.} At five in the afternoon, Drummond opened the box, showed the picture to the public, and paid to every winner a cash prize worth 20 times the zoo's admission fee.\footnote{The amount was higher than a carpenter's monthly wage \citep[542]{chazkel2007beyond}.} The lottery was labelled as the \emph{jogo do bicho}, or the animal game, and it was immediately adopted by the public. Eager to capitalise on that initial success, Drummond stated that visitors could buy tickets not only at the zoo, but in stores across Rio de Janeiro. He rightly predicted that this small change would increase profits, but there was one thing the Baron did not foresee. He unleashed a new gambling market.

The \emph{jogo do bicho} craze swept the whole city after independent sellers entered the marketplace \citep{magalhaes2005ganhou, soares1993jogo}. A network of street bookmakers, called \emph{bicheiros}, expanded spontaneously the original game in innovative ways. Evading state regulations, \emph{bicheiros} made the lottery available in every part of Rio by scalping tickets or promoting their own versions of the clandestine numbers game  \citep[37]{chazkel2011laws}. The \emph{jogo do bicho} became so widespread that Olavo Bilac, a major literary figure in nineteenth-century Brazil, summarised the situation as follows: `Today {[}1895{]}, in Rio de Janeiro, the game is everything. {[}\ldots{}{]} Nobody works! Everybody plays' \citep[43]{pacheco1957antologia}.\footnote{Unless otherwise noted, all translations from the Portuguese are my own.} But this tolerant state of affairs did not last. Civil servants and police officers criminalised the \emph{jogo do bicho} on the grounds of `public safety', and in the late 1890s they launched a country-wide campaign against the lottery \citep{benatte2002jogos, krelling2014jogos, villar2008contravencao}.\footnote{The National Lottery Company (\emph{Companhia das Loterias Nacionaes do Brazil}), a public-private partnership founded four years after the creation of the \emph{jogo do bicho}, also lobbied actively for a hard-line stance against the \emph{bicheiros} \citep[82]{da1999aguias}.}

Yet the game has survived. The animal game has outlasted more than 30 Brazilian presidents and thrived under military regimes and democratic governments alike \citep{gaspari2002ditadura, jupiara2015poroes}. But more than a act of defiance, the \emph{jogo do bicho} is a successful capitalist enterprise \citep{labronici2014sorteio, magalhaes2005ganhou}. A recent study by Fundação Getúlio Vargas, a Brazilian think tank, affirmed that the \emph{jogo do bicho} earns from BRL 1.3 to BRL 2.8 billion per year (USD 400 to USD 850 million), making it the largest clandestine gambling game in the world.\footnote{See \url{http://goo.gl/9kNeX8} and \url{http://goo.gl/8FSAZl} (in Portuguese). Access: December 2016.} \citet[171]{schneider1996brazil} estimated that in the 1990s, the game furnished about 50,000 jobs in the Rio de Janeiro city alone, almost the same number of employees that the oil giant Petrobras had in 2011 \citep{exame2013petrobras}.\footnote{In 1966, Time Magazine wrote that the \emph{jogo do bicho} was `the largest single industry in Latin America' and employed about 1\% of the Brazilian workforce. See \url{http://content.time.com/time/magazine/article/0,9171,842527-1,00.html}. Access: December 2016.}

\begin{figure}[!htbp]
	\centering
	\begin{minipage}[b]{0.45\textwidth}
		\includegraphics[width=\textwidth, height=6cm]{images/bicho01.jpg}
	\end{minipage}
	\hfill
	\begin{minipage}[b]{0.45\textwidth}
		\includegraphics[width=\textwidth, height=6cm]{images/bicho02.jpg}
	\end{minipage}
	\caption{Left: cartoon of the Baron of Drummond and the animals of the \emph{jogo do bicho} (1896). Right: entry ticket to Rio's Zoological Garden that allowed the bearer to join the raffle. Sources: Instituto Histórico e Geográfico Brasileiro, Rio de Janeiro, Revista Illustrada, ano 21, no. 718 (1896) and Museu da Imagem e do Som, Rio de Janeiro. Reproduced in \citet[35--36]{chazkel2011laws}.}
	\label{fig:barao}
\end{figure}

Moreover, the animal game plays a crucial role in the expansion of Rio de Janeiro's Carnival Parade, a popular festivity synonymous with Brazil at home and abroad \citep{araujo2003carnaval,costa2001100,da1973carnaval, da1979carnavais,vianna1995misterio}. \emph{Bicheiros} donate hefty sums to `samba schools' to gather support of poor communities and, no less importantly, to co-opt local politicians attracted by the financial and electoral gains offered by the festival \citep{cavalcanti2006carnaval, queiroz1992carnaval}. This patron-client relationship has been proven effective: in 2016, the Carnival generated about USD 900 million in revenue and the Rio de Janeiro state received more than one million tourists.\footnote{Data provided by the Brazilian government. See \url{https://goo.gl/XMcbTM} (in Portuguese). Access: December 2016.}

In this article I offer a rational choice interpretation of the \emph{jogo do bicho} and discuss how \emph{bicheiros} promote social order, solve information asymmetries, and reduce negative externalities. My analysis discusses three strands of academic literature. First, this work contributes to the scholarship on extra-legal institutions, mainly to the literature on collective action within criminal organisations. For instance, \citet{gambetta1996sicilian} examines the strategies used by the Sicilian Mafia to settle disputes among their members and enforce rules in the areas they exercise control. \citet{leeson2007arrgh,leeson2009invisible,leeson2010pirational} affirms that pirate groups employed hard-to-fake signals to increase the profitability of their operations. \citet{skarbek2011governance,skarbek2012prison,skarbek2014social}, in turn, highlights the role of written and implicit norms in mitigating rent-seeking and coordinating productive activities in California prison gangs. I argue that \emph{bicheiros} have employed reputation strategies and provided club goods to enforce private contracts and foster trust among criminals. Moreover, I also describe how \emph{bicheiros} have developed sophisticated financial mechanisms, such as informal hedging operations and risk-sharing contracts, to prevent predatory behaviour in their community.

Second, this work relates to the literature on repugnant transactions and the relationship between morality and the market \citep{boettke1995morality, roth2007repugnance, sandel2012money, satz2010some, simmel1900geldes, zelizer1979morals}. In the following sections, I claim that the Brazilian elites have attached pejorative meaning to the \emph{jogo do bicho} to constrain the gambling market. I provide evidence that \emph{bicheiros} were aware of this problem, and as a response, they devised a series of rules aimed at reducing the costs associated with repugnance \citep{labronici2014sorteio, magalhaes2005ganhou}. \emph{Bicheiros} have made considerable efforts to increase the levels of trust in the system and distance themselves from other types of illegal activities. Their main tool to increase credibility was costly signalling, that is, the \emph{bicheiros} hoped the public would see them as credible brokers by sacrificing their immediate interests  \citep{gambetta2009codes,kimbrough2015commitment, schelling1960strategy}.

Lastly, this work connects to the literature on state capture, which is among the most important topics in public choice theory \citep{hellman2003seize, rose1978corruption, rose1999corruption, shleifer2002grabbing, tollison1982rent}. More specifically, I use the Brazilian case to illustrate how politicians and civil servants can be co-opted by criminal groups and produce sub-optimal social outcomes. \citet{queiroz1992carnaval} explored why \emph{bicheiros} turned into patrons of the Carnival's samba schools and affirmed that this influence gave them leverage over political authorities. \citet{misse2007illegal} investigated the links between bicheiros and police officers, and suggested that the illegal lottery had been the main cause of police corruption in Rio de Janeiro until the 1970s. In a similar vein, \citet{jupiara2015poroes} analyse the relationship between the \emph{jogo do bicho} and the military regime in Brazil (1964--1985). I supplement this literature by highlighting how asymmetrical information, agency dilemmas, and rent-seeking behaviour offer convincing explanations to the issues presented above. Although those concepts have a long tradition in public choice, scholars have not applied those ideas thus far to understand the dynamics of the \emph{jogo do bicho}. By doing so, I integrate seemingly contradictory historical facts into a single narrative that connects micro-level decisions to macro-level outcomes.

The remainder of this article proceeds as follows. Section \ref{sec:organisation} presents a brief historical overview of the \emph{jogo do bicho}. It describes the necessary conditions for the emergence of the game and presents its basic organisational structure. Section \ref{sec:governance} details the \emph{jogo do bicho}'s governance mechanisms, particularly the strategies employed by vendors to increase trust in markets that operate at the margins of the law. Section \ref{sec:capture} discusses the links between illegal gambling markets, samba schools and the Brazilian state. Section \ref{sec:conclusion} offers some concluding remarks.

\section{\emph{Jogo do Bicho} as an Emergent Institution}
\label{sec:organisation}

\subsection{Historical Background}
\label{sub:background}

The early history of the \emph{jogo do bicho} is a textbook example of spontaneous order. Spontaneous orders are emergent macro-level phenomena that result from voluntary actions of purposive, self-interested individuals utilising their contextual knowledge \citep{boettke1990theory, boettke2005methodological, hayek1945use, hayek1960constitution, hayek1973law, leeson2008coordination, menger1871grundsatze, polanyi1948planning, polanyi1951logic}. Drummond, the Zoological Garden's original owner, designed the basic framework for the \emph{jogo do bicho}; but independent bookmakers were the ones who popularised the game \citep[77]{magalhaes2005ganhou}. Ticket sellers could quickly respond to market signals and then allocate their products where they were more valuable because of the lack of central coordination. Moreover, competition among sellers fostered innovation, and the \emph{bicheiros} invented new game rules to make the lottery more appealing to their customers \citep[61]{mello1989historia}. In this sense, the animal game is the materialisation of an evolutionary process of entrepreneurial discovery in which the interactions that provided the highest value to consumers were preserved over time \citep{boettke2008gordon, boettke2014entrepreneurship, buchanan1964should, hayek1978competition, kirzner1997entrepreneurial}.

However, the \emph{jogo do bicho} only emerged because of historically contingent circumstances. The late nineteenth-century Brazil had four characteristics that explain how the animal game came to being: 1) a growing urban population excluded from the formal labour market; 2) an inflow of immigrants whose extended family networks helped them engage in trade; 3) an expansion of the monetary supply in the first years of the republic (1880s--1890s); and 4) a judicial system that, albeit repressive, had only imperfect law enforcement. Figure \ref{fig:dag} presents a simple directed acyclic graph (DAG) \citep{pearl2009causality} that shows the relationships between these explanatory variables and the development of the \emph{jogo do bicho}.\footnote{The main purpose of direct acyclic graphs is to graphically display the possible links among the exposure variables, confounders, and outcomes \citep{morgan2014counterfactuals, pearl2009causality}. DAGs are transparent by definition, as all theoretical choices made by the researcher are stated explicitly in the model. Each single-headed arrow in a DAG indicates that the variable at the origin causes the variable at the end of the directed edge. Dashed edges suggest that two variables are jointly dependent on unobserved common causes. There are no assumptions regarding the functional form of the relationships, and unless mentioned otherwise, the arrows represent fully non-parametric associations. Variables between two nodes are mediators, and variables pointed at by two or more factors have multiple causes. They are called \emph{colliders}. For the sake of clarity, errors are assumed independent and often excluded from the graphs. For an accessible introduction to DAGs see \citet[chap. 3--4]{morgan2014counterfactuals} and \citet{pearl2016causal}.}

\begin{figure}[!htbp]
	\begin{tikzpicture}[->,>=stealth',shorten >=4pt,auto,node distance=3cm, semithick][h]
		% nodes %
		\node[circle,fill,inner sep=3pt,label=above right: \shortstack{Urban \\ Poverty}] (p) {};
		\node[circle,fill,inner sep=3pt,label=right: \shortstack{Jogo \\ do Bicho}, right = 8 of p] (y) {};
		\node[circle,fill,inner sep=3pt,label=below left: {Immigration}, below = 3 of p] (i) {};
		\node[circle,fill,inner sep=3pt,label=above:{Money Supply}, above = 3 of p] (m) {};
		\node[circle,fill,inner sep=3pt,label=below right:{Social Ties}, right = 3 of i] (s) {};
		\node[circle,fill,inner sep=3pt,label=above:{Weak Law Enforcement}, above = 3 of y] (l) {};
		\node[circle,fill=white, draw, outer sep=0pt, inner sep=3pt,label= left: \shortstack{Correlated \\ Background Factors}, above left = 2 of p] (u) {};
		% edges %
		\draw[->, line width = 1.2] (p) -- (y);
		\draw[->, line width = 1] (s) -- (y);
		\draw[->, line width = 1] (i) -- (s);
		\draw[->, line width = 1.2] (l) -- (y);
		\draw[->, line width = 1] (m) -- (y);
		\draw[->, line width = 1.2] (i) -- (p);
		\draw[->, dashed, line width = 1] (u) -- (p);
		\draw[->, dashed, line width = 1] (u) -- (m);
	\end{tikzpicture}
	\caption{Directed Acyclic Graph -- Explanatory Variables for the \emph{Jogo do Bicho}}
	\label{fig:dag}
\end{figure}

I start with the impact of urban poverty on the animal game. Brazil abolished slavery in the late 1880s, a period in which the country was rapidly urbanising and freed slaves migrated to its growing cities \citep{andrews1991blacks, fausto2014concise, naro1992revision, skidmore1993black}. The former slaves were joined by increasing numbers of Asian and European immigrants \citep{hall1969origins, lesser2013immigration, smith1979ethnic}. Nevertheless, the job market tightened considerably after the \emph{Encilhamento} financial crisis of 1891 \citep{topik2014political, triner2005baring}. During the economic downturn, the informal economy was an obvious destination for the urban poor. Given its widespread popularity, the \emph{jogo do bicho} attracted hopeful entrepreneurs, either Brazilian or foreign-born, who could not enter the formal labour force.\footnote{The underground economy was also more democratic than the formal sector. As \citet[115]{chazkel2011laws} observes, one of the few professions open to poor women and foreigners in the early 1900s was that of street vendor. These vendors used to sell different types of merchandise and many of them would later offer \emph{jogo do bicho} tickets.}

The immigration also influenced the \emph{jogo do bicho} via social ties. Most foreigners who moved to Brazil came from countries, such as Portugal, Spain or Italy, where extended families were the basic form of social organisation \citep{klein1994imigraccao, lobo2001imigraccao, trento1989outro}. Family and neighbourhood networks created incentives for immigrants to establish trade relations and enforce cooperation through community responsibility systems \citep{roth2014prison}. Because of these particular social characteristics, in the 1890s foreigners were over-represented in the Brazilian trade in general \citep{mattos1991vadios, oliveira2001brasil, truzzi2008patricios} and in the \emph{jogo do bicho} in particular \citep{godoi2012imigraccao, magalhaes2005ganhou, torcato2011repressao, villar2008contravencao}. Although kinship bonds became less relevant over time, these links offered an important element of social cohesion in the \emph{jogo do bicho}'s formative years.

Next is the impact of expanded monetary supply. The abolition of slavery and the growing industrialisation of Brazil increased the amount of capital available in the country \citep{franco1987reformas, schulz2008financial}. The 1888 Banking Act gave extra liquidity to local financial markets, and the \emph{jogo do bicho} entrepreneurs utilised that increase in the monetary base to extend the scope of their business. Some years later, the animal game would be available not only across the city of Rio de Janeiro but throughout Brazil \citep[76]{da1999aguias}.

The country's lax financial policy might correlate with poverty through unspecified factors. For instance, political decisions may have caused inadvertently both poverty and the expansion of the monetary base \citep{mattos2013shantytown, schmidt1982modernization}; alternatively, external events such as institutional instability \citep{costantini2014index, fausto2014concise, luna2014economic} or commodity shocks \citep{musacchio2014colonial} could be the cause of those two variables. There is not enough evidence to discard such scenarios. To illustrate this uncertainty, the two nodes, namely, money supply and urban poverty, are connected with a dashed edge in Figure \ref{fig:dag}.

The last necessary condition for the emergence of the \emph{jogo do bicho} is weak law enforcement. \citet[69--100]{chazkel2011laws} notes that until the 1940s police district chiefs operated within a large margin of discretion and repression against bookmakers was idiosyncratic. Prosecution against the \emph{bicheiros} had hardened in 1917, but only in 1946, when the federal government banned all gambling activities in the country \citep[155--156]{magalhaes2005ganhou}, the law was consistently enforced.

\subsection{Organisational Structure}
\label{sub:organisation}

The animal game has three levels of hierarchy. At the bottom are the \emph{bicheiros}, who are those in charge of selling \emph{jogo do bicho} tickets \citep{chazkel2007beyond, chazkel2011laws, da1999aguias, labronici2014sorteio, magalhaes2005ganhou, misse2007illegal}. \emph{Bicheiros} are the most visible part of the \emph{jogo do bicho} structure. The bookmakers often build their vending stands inside the premises of a local shop, such as a small grocery store or a pub, and are recognisable by their chairs facing the street, stamps and blocks of paper \citep[259]{chazkel2011laws}. \emph{Bicheiros} usually work alone, but they may employ up to 10 people depending on how busy their betting site is \citep[69]{labronici2014sorteio}.

The \emph{gerentes} (managers) oversee all \emph{jogo do bicho} stands in a given area. Their task is akin to that of a firm accountant. Gerentes control the cash flow between the \emph{bicheiros} and the bankers, manage the payroll of the employees, and provide financial information to the top members of the organisation. They also supervise individuals who carry menial tasks in the business, transfer money to other gambling branches and double-check the balance sheets of the betting sites \citetext{\citealp[71]{labronici2012paratodos}; \citealp[142]{misse2007illegal}}.

\begin{figure}[!htbp]
	\centering
	\begin{minipage}[b]{0.45\textwidth}
		\includegraphics[width=\textwidth, height=6cm]{images/bicho03.jpg}
	\end{minipage}
	\hfill
	\begin{minipage}[b]{0.45\textwidth}
		\includegraphics[width=\textwidth, height=6cm]{images/bicho04.jpg}
	\end{minipage}
	\caption{\emph{Jogo do bicho} betting sites in 1917 and in 2011. The picture on the right shows a \emph{bicheiro}, a street-corner vendor. Sources: \citet{alecrim2012bicho} and \citet{ferrarini2011bicho}.}
	\label{fig:banca1917}
\end{figure}

The \emph{banqueiros}, or the Portuguese for bankers, occupy the top position in the \emph{jogo do bicho} hierarchy, comprising the small financial elite of the game. A 2012 report by the Brazilian Federal Police affirmed that 10 \emph{banqueiros} controlled the market throughout the country; five of them based in the state of Rio de Janeiro \citep{globo2012contraventores}. Apart from funding the game, the bankers provide support for the employees to undertake their activities. The \emph{banqueiros}' main attributions include paying bribes to police personnel, bailing out sellers arrested by security forces, and offering judicial assistance to employees in case of legal persecution \citep[75]{labronici2012paratodos}.

\emph{Banqueiros} run their businesses from fortified houses in unknown locations, the \emph{fortalezas} (`forts'). The first \emph{fortalezas} likely appeared in the 1950s, when the animal game was already well-established across the Brazilian territory. The period coincides with a time when the \emph{jogo do bicho} finances had become increasingly concentrated in fewer hands \citep[259]{chazkel2011laws}. Due to the growing size of the \emph{jogo do bicho} economy, \emph{banqueiros} decided to move their operations away from the public to avoid police persecution and reduce coordination costs.

Although the forts provided safety to the bankers, the existence of those hideouts posed a challenge to the organisation. Bankers removed from the public view are not accountable to players and booking agents. Similarly, bankers and managers working in the \emph{fortalezas} cannot oversee their employees as effectively as before. Considering that the animal game itself is illegal and the amount of money involved in the bets is often substantial, both \emph{banqueiros} and booking agents have strong incentives to defect. Players, in turn, have no evident reason to trust \emph{banqueiros} or \emph{bicheiros}. How do \emph{jogo do bicho} agents overcome trust issues and cooperate under uncertainty?

I argue that the \emph{jogo do bicho} solves problems of internal cooperation by providing club goods \citep{buchanan1965economic, berman2008religion, berman2009radical, leeson2011government, roth2014prison} while simultaneously shunning cheaters through punishments and appeals to `the shadow of the future' \citep{axelrod1984evolution, axelrod1985achieving, bo2005cooperation, roth1978equilibrium}. Clients and \emph{bicheiros} cooperate based on trust-enhancing mechanisms, most of them devised specifically for the \emph{jogo do bicho} \citep{da1999aguias, magalhaes2005ganhou}. Such mechanisms are relevant because they have allowed the \emph{jogo do bicho} to distance itself from other shadow markets and become a profitable enterprise in the long run.

\section{Governance of the \emph{Jogo do Bicho}}
\label{sec:governance}

\subsection{Gambling Markets and Repugnant Transactions}
\label{sub:repugnance}

The \emph{jogo do bicho} is a repugnant market. Individuals that like to gamble cannot do so because of strong moral objections from outsiders \citep{brisset2016marche, roth2007repugnance, satz2010some, zelizer1979morals}. As early as in 1890, Brazilian public authorities positioned themselves against the \emph{jogo do bicho} arguing that `{[}\ldots{}{]} this type of amusement is prejudicial to the interests of the unwise, who are naively seduced by the deceptive hope of uncertain lucre' \citep[544]{chazkel2007beyond}. In 1941, the government banned the animal game;\footnote{See: \url{http://www.planalto.gov.br/ccivil_03/decreto-lei/Del3688.htm} (in Portuguese). Access: December 2016.} five years later, it prohibited all games of chance.\footnote{The 1946 decree stated that gambling was `harmful to morality and the good customs', hence `[\dots] the repression against games of chance [was] an imperative of the universal consciousness'. The text can be read at: \url{http://www.planalto.gov.br/ccivil_03/decreto-lei/Del9215.htm} (in Portuguese). Access: December 2016.} The \emph{jogo do bicho}, casinos and bingos remain illegal in the country. Recent estimations show that the prohibition of the \emph{jogo do bicho} have prevented the state from earning BRL 15 to BRL 20 billion (USD 4.5 to USD 6 billion) per year in expected taxation revenues, aside from the subjective utility losses for players. \citep{congressoemfoco2015bicho, fsp2016legalizarbicho}.

In contrast with the official statements, the noxious element of the \emph{jogo do bicho} does not come from its inherent randomness. The game is `repugnant' precisely because it is \emph{a market}, a setting in which individuals can monetise the entertainment for private profit \citep{chazkel2007beyond, chazkel2011laws}. The Brazilian state has never seen any contradiction between banning games of chance and running a national lottery company of its own; even the Catholic Church, which has long condemned the \emph{jogo do bicho}, frequently organises raffles to fund its activities \citep[49]{abreu1996imperio, magalhaes2005ganhou}. Only after the introduction of private money that Brazilians objected the idea of benefiting from someone else's bad luck.

In this sense, the main obstacle that confronted \emph{bicheiros} was to convince others that the animal game would not cause the Brazilian society `to slide down a slippery slope to genuinely repugnant transactions' \citep[45]{roth2007repugnance} such as prostitution or debt bondage. As the century-old history of the game can attest, \emph{bicheiros} have succeeded in this task. But how? The literature on repugnant costs tell us little about how markets transition from noxious to tolerated. Here I posit two mechanisms that reduced the stigma associated with the game: 1) \emph{a strong reputation of honesty} expressed by costly signals from sellers, and 2) the provision of \emph{selective incentives} for both clients and booking agents. Below, I offer evidence that these two factors allowed the animal game to reach its current semi-legal status in Brazil.

\subsection{External Cooperation}
\label{sub:external}

Evolutionary game theory \citep{axelrod1984evolution, axelrod1985achieving, smith1982evolution} and experimental studies \citep{dawes1977behavior, isaac1984divergent, kim1984free, marwell1981economists} have both demonstrated that long-term cooperation is possible whenever players expect future pay-offs to be higher than present ones. Fear of retaliation induces individuals not to cheat. Nevertheless, illegal organisations tend to discount the future even more heavily than the other groups, what makes cooperative behaviour among criminals uncommon \citep{gambetta2009codes, skarbek2011governance,skarbek2012prison,skarbek2014social}. The \emph{jogo do bicho} is an exception to this rule. The market properties of the game and inconsistent repression by Brazilian authorities have permitted \emph{bicheiros} to overcome the stigma of repugnance and improve the game's long-term profitability.

The \emph{jogo do bicho} entrepreneurs have made considerable efforts to present themselves as honest brokers. The first trust-enhancing mechanism they have employed to foster external cooperation was the use of a \emph{fixed-multiplier formula} for pay-outs. It works as follows. If a player wins the lowest prize of the animal game, he or she receives 18 times his/her investment regardless of the size of the bet. Bigger prizes naturally offer higher returns; a lucky winner of the top prize wins up to 4,000 times the value of his/her bet \citetext{\citealp[89]{labronici2012paratodos}; \citealp[20]{magalhaes2005ganhou}}.

This stands in sharp contrast to the common practice of sharing a prize among winners. Lottery pay-outs demand high levels of interpersonal trust: players rely on unverifiable information about the total funds collected by the lottery, and they can never be sure whether the payments are evenly distributed. The fixed-multiplier formula alleviates such problems of adverse selection \citep{akerlof1970market, cohen2010testing, levin2001information}. As players and vendors known the prize value beforehand, the method provides consumers with complete information about their individual prizes while also binding the \emph{bicheiros} to a contract that can be easily enforced. This technique offers buyers a simple yet effective screening strategy that induces \emph{bicheiros} to provide honest information about the game \citep{spence1973job, stiglitz1981credit}.

\emph{Bicheiros} have addressed information asymmetries in another ways. Since the 1950s, when the \emph{jogo do bicho} bankers had moved their operations to the \emph{fortalezas}, the public could not oversee the lottery draws \citep[259]{chazkel2011laws}. This could lead to a decline in trust among buyers and vendors of lottery tickets and, as a result, to reduced profits. \emph{Bicheiros} have mitigated this problem with a two-pronged strategy. First, they started to utilise the winning numbers from the licit government-run lottery, the \emph{Loteria Federal}, instead of their own draws \citetext{\citealp[546]{chazkel2007beyond}; \citealp[89]{labronici2012paratodos}; \citealp[39-40]{mello1989historia}}. The federal lottery numbers are public information. The media broadcasts the draws on radio and TV, so any interested player can verify the selected numbers. The Loteria Federal is also audited by two independent state institutions, a private accounting firm, and voluntary members of the public; hence, \emph{bicheiros} can free ride on the lottery's long-standing reputation of credibility.\footnote{As of April 2016, the lottery was audited by the \emph{Controladoria Geral da União} (Comptroller General of Brazil), the \emph{Tribunal de Contas da União} (General Accounting Office), and by Ernst \& Young. The balls are measured and weighted every three months by the National Institute of Metrology, Quality and Technology (Inmetro), the Brazilian equivalent of United Kingdom's National Physical Laboratory or the American National Standards Institute. See \url{http://noticias.uol.com.br/cotidiano/ultimas-noticias/2016/04/08/auditoria-dos-sorteios-da-caixa-e-confiavel-veja-como-e-o-processo.htm} (in Portuguese). Access: December 2016.}

\begin{figure}[!htbp]
	\centering
	\includegraphics[width=\textwidth, height=6cm]{images/bicho06.jpg}
	\caption{Results of a \emph{jogo do bicho} draw from 09 January 2016. `Federal' means that the winning numbers were drawn by the federal government lottery. The first prize was group 14, the cat. Source: Unknown. Available at: \url{https://goo.gl/6PHV8u}. Access: December 2016.}
	\label{fig:federal}
\end{figure}

Second, they included representatives of all major \emph{jogo do bicho} bankers in every draw and independently publicise the game results. Certain \emph{bicheiros} went as far as publishing the numbers in Rio's newspapers. In the early twentieth century, some tabloids were entirely dedicated to the game \citep[60]{magalhaes2005ganhou}. Booking agents see this strategy as a credible signal from the game financiers, as providing contrasting information would indicate game manipulation. Moreover, collusion can also be spotted if the draws show repeated numbers or unusual patterns.

\begin{figure}[!htbp]
	\centering
	\includegraphics[width=\textwidth, height=6cm]{images/bicho05.jpg}
	\caption{\emph{Jogo do bicho} results are fixed on light poles in Rio de Janeiro. Source: \citet{gomes1998bicho}.}
	\label{fig:poste}
\end{figure}

These efforts have proved popular with the game enthusiasts. One often-repeated saying about the \emph{jogo do bicho} is that `in the \emph{jogo do bicho}, what is written down counts' \citep[159]{chazkel2011laws}, that is, buyers and sellers do fulfill their informal obligations without third-party enforcement. Such mutual confidence reduces the potential for conflict in the game. As the public does not see the \emph{jogo do bicho} as violent or harmful, the stigma of repugnance associated with gambling becomes less pervasive. By reducing the possibilities of cheating and putting long-term interests first, the \emph{jogo do bicho} bankers have avoided the fate of other repugnant markets and run their business relatively undisturbed for decades \citep[20]{da1999aguias}.

\subsection{Internal Governance}
\label{sub:internal}

Individuals working at different levels of hierarchy often have non-aligned interests. As a result, it may occur that one party (the agent) behaves rationally in a manner that maximises his/her benefits, but that is contrary to the interests of his/her superior (the principal). This dilemma is pervasive in formal organisations \citep{holmstrom1979moral,jensen1976theory,moe1984new,shapiro2005agency,spence1971insurance}; in illegal markets perhaps it is even more so \citep{campana2013cooperation,gambetta2009codes,skarbek2011governance,skarbek2014social}. As monitoring costs in criminal businesses are higher than in formal ones, principals face considerable difficulties to induce cooperation from agents. Moreover, criminals often engage in opportunistic behaviour and `hidden actions', that is, they do not put the required levels of effort if they know they are not being monitored \citep[38--42]{arrow1985agency}.

In the \emph{jogo do bicho} setting, one such problems concerns the trade-off between short- and long-term incentives for managers and bankers on the one side and bookmakers on the other. Managers have a permanent interest in the long-run profitability of the game, whereas street-corner booking agents tend to discount the future more heavily because their financial gains are small compared to that of their superiors. Additionally, bookmakers may denounce their employers to the police if they feel threatened.

One way by which the \emph{jogo do bicho} principals solve the agency dilemma is by supplying club goods and selective incentives for low-tier members. Club goods are goods that can be simultaneously enjoyed by more than one individual but where exclusion mechanisms prevent consumption by non-members \citep{buchanan1965economic, cornes1996theory, olson1965logic, sandler1980economic, sandler1997club}. Basically, club goods are `public goods \emph{sans} non-excludability' \citep[928]{mcnutt1999public}. The first club good offered to \emph{bicheiros} by their bosses is private security. The game bankers have built an extensive network of gunmen and bribed police officers to protect their employees (and their profits) from other criminals \citetext{\citealp[48]{chinelli1993vazio}; \citealp[51]{labronici2012paratodos}}. The \emph{jogo do bicho} network has a powerful deterrence effect and lethal force is rarely employed. However, threats are constant. `Zé' (Little Joe), a bicheiro interviewed by \citet[52]{labronici2012paratodos}, described eloquently the deterring effect of the \emph{jogo do bicho} informal security personnel:

\begin{quote}
	[\dots] bums are scared and they don't mess around with us; they think there's a guard nearby or something like that. Look at all this money here! [shows the interviewer a handful of cash] It's not ours [referring to street-corner bookmakers]. And if it's not ours, it's someone else's. When I worked in Penha (\emph{a low middle-class neighbourhood in the city of Rio de Janeiro -- translator's note}), the owner of a pub close to where I used to work always asked me to stay at the front door of his pub. People know that bums are afraid of \emph{bicheiros}.
\end{quote}

Apart from guaranteeing the physical integrity of the \emph{bicheiros}, bankers and managers also provide financial incentives for the bookmakers. \emph{Bicheiros} are allowed to receive tips from players, often have small expenses paid by managers, and may even request interest-free loans to cover unexpected costs such as illness-related expenses \citep{labronici2012paratodos}.

However, the most important financial mechanism implemented by bankers to help \emph{bicheiros} is the \emph{descarga}, which is loosely translated as `the unloading'. The descarga is the \emph{jogo do bicho}'s main hedging technique and its purpose is to insure bookmakers against credit risk \citetext{\citealp[59]{labronici2012paratodos}; \citealp[178]{magalhaes2005ganhou}; \citealp[16]{misse2007illegal}; \citealp[75]{soares1993jogo}}. Booking agents are sometimes unable to honour expensive bets. As mentioned above, the top prize in the animal game pays up to 4,000 times the amount invested, thus \emph{bicheiros} may have to raise thousands of Brazilian Reals in a single day. To prevent the \emph{quebra da banca} (`bust of the bank'), \emph{bicheiros} and small bankers buy an insurance from wealthier financiers, who offer this service for a fee that ranges from 20\% to 25\% of the total selling amount \citep{fsp2006descarga}. The \emph{descarga} guarantees that small bookmakers will not have liquidity problems, thus permitting bookmakers to continue investing in the \emph{jogo do bicho}.

The descarga has played an important role in reducing individual risk; nevertheless, it has also changed the distribution of resources in the \emph{jogo do bicho}. Simple probability dictates that a booking agent rarely pays the highest prize in the \emph{jogo do bicho}; in contrast, the bankers receive a commission for \emph{every game} they hedge. Over time, there is a transfer of income from the bottom to the top of the animal game structure led by this constant inflow of fees. This accumulation of capital is probably one of the reasons why bankers were able to diversify their businesses and offer other types of entertainment such as slot machines and sports lotteries \citep{estado2006cacaniquel,globo2015cacaniquel,terra2011cacaniquel}. The descarga has made the game more resilient at the aggregated level, although it increased profits for the richest financiers at the expense of small bookmakers.

\section{Tropical State Capture: \emph{Jogo do Bicho}, Samba and Politics}
\label{sec:capture}

The impact of the \emph{jogo do bicho} is not restricted to the Brazilian economy. Since the 1960s, \emph{bicheiros} have been the key sponsors of the country's most important cultural and social festivity, the Rio de Janeiro Carnival parade \citep{bezerra2009mecenato,cavalcanti2006carnaval,chinelli1993vazio,queiroz1992carnaval}. The \emph{jogo do bicho} accounts for such large share of the funding of the parade that a famous \emph{banqueiro} once remarked that `without the \emph{jogo do bicho} the Carnival would have ended' \citep{odia2016aniz}. Owing to that support, \emph{bicheiros} have established an extensive patronage network with samba schools and local politicians \citetext{\citealp[4641]{arguello2012criminalizaccao}; \citealp{congressoemfoco2007bicho}; \citealp{jornaldobrasil2011bicho}; \citealp[16]{misse2011crime}}. Although that network brings large material benefits to their members, the patronage system has created perverse incentives for government officials.

The \emph{jogo do bicho}'s clientelism is more evident in the state of Rio de Janeiro than in other parts of the country. Historical factors explain why this is the case. Firstly, Rio de Janeiro city was the capital of Brazil for almost 200 years; despite losing the position to Brasília in 1960, it remains one of the country's main cultural and financial centres. Secondly, \emph{jogo do bicho} operators had historical ties with popular movements, which they eventually exploited to their advantage. Thirdly, the emergence of state-sponsored Carnival parades created a window of opportunity for \emph{bicheiros} to expand their influence over public authorities, either via bribing or by funding political campaigns. In this regard, Rio provided a suitable environment for self-interested politicians, community leaders and animal game financiers to collaborate. These illegal networks are crucial to understand why samba and Carnival became constituent features of Brazil's national identity, and how the festival has contributed to Rio's high levels of state corruption.

\subsection{The `Medici of Samba': \emph{Bicheiros} as Patrons of Carnival}
\label{sub:patrons}

In 1930, opposition leader Getúlio Vargas led a bloodless coup d'état that brought Brazil's First Republic to an end. During his first presidency (1930--1945), Vargas promoted a radical shift in Brazilian politics by dismantling effectively federalism in favour of a powerful executive branch and an expanded federal bureaucracy \citep[e.g.][]{bethell2008politicsvargas,desouza1983estado,fausto1972revoluccao,fausto2014concise,skidmore1967politics}. In terms of ideology, Vargas's authoritarian-corporatist \emph{Estado Novo} (``New State'') promoted a politicised nationalism designed to transcend the regional aspects of Brazilian culture \citep{lauerhass1972getulio,nava1998lessons,williams2001culture}. Popular music, in turn, occupied an important place in Vargas's project of `brazilianing Brazil'. Created in the late 1920s in the shanty towns of Rio de Janeiro, modern samba embodied the idea of the multicultural, racially-tolerant country the government aspired to forge \citep{avelar2011brazilian,mccann2004hello,stockler2011samba,vassberg1969villa,vassberg1975villa}.

By the late 1930s, samba reached a unique position in Brazil's cultural identity. In a period when civil and political rights were limited \citep{de2001cidadania,duarte1993vicissitudes}, Vargas used samba as a means to incorporate ethnic minorities and the new urban classes into the Brazilian mainstream \citep[213]{chinelli1993vazio}. Patriotic sambas exalted the country's natural beauties and the figure of the `friendly, happy, cordial and industrious' mulatto\footnote{A mulatto is a person of mixed white and black ancestry. The etymology of the word is originally derogatory as it alludes to `mule' (Latin: \emph{mulus}), the infertile offspring of the male donkey and a female horse. However, in the 1930s the word loses its pejorative connotation in Brazil. Mainly due to the work of sociologist \citet{freyre1933casa}, the idea of a racial democracy becomes pervasive in the government discourse, and as a result the word gains a positive tone \citep[4]{reiter2009brazil}.} \citetext{\citealp[47]{dangelo2016samba}; \citealp[51]{vianna1995misterio}}. The institutionalisation of the Carnival parade in 1935, and the subsequent increases in public funding to the festival, cemented the relationship between politicians and samba groups \citep{almeida2017carnaval,cabral2016escolas,soihet1998subversao}.

However, the samba groups were not passive members in this process. Since the 1960s, the Rio Carnival expanded in scope and, stimulated by growing numbers of spectators, the parades became more elaborate \citetext{\citealp{cabral2016escolas}; \citealp[214]{chinelli1993vazio}; \citealp[240]{hertzman2013making}}. Unable to cope with the rising costs of the show, the `samba schools', which are large samba groups that compete in the Carnival, resorted to the \emph{jogo do bicho} financiers to fund their activities \citep{misse2007illegal}. This informal agreement between samba school organisers and wealthy \emph{bicheiros} remains effective to this day, and many of Rio's most famous samba schools are officially presided by high-profile members of the \emph{jogo do bicho} elite \citep{bezerra2009mecenato,cavalcanti2006carnaval,farias2013carnival,misse2011crime,queiroz1992carnaval}.

As I have mentioned in the previous section, the animal game at times faced opposition by the local population. The public often perceived the game as immoral and repugnant. Moreover, even after the bicho was well-established in Rio de Janeiro, the transition from a competitive betting market to an oligopoly involved the threat and often the use of physical violence against bookmakers who resisted the change \citetext{\citealp[143]{bezerra2009mecenato}, \citealp[52]{labronici2012paratodos}}. \emph{Bicheiros} were aware of the reputation costs their strategy entailed. They decided to finance samba schools hoping to win `the hearts and minds' of the population and attach a more positive image of the game among urban classes. Members of the \emph{jogo do bicho} had been involved in the Carnival since the early 1920s, but only as individuals who had a private interest in samba \citep[209]{chinelli1993vazio}. In 1984, a group of rich \emph{jogo do bicho} financiers founded collectively the LIESA (\emph{Liga Independente das Escolas de Samba}, Independent League of the Samba Schools), a civil association intended to direct and sponsor the Carnival parade in Rio de Janeiro. The LIESA marked a shift in the Carnival. For the first time, \emph{bicheiros} decided to act as a group rather than individuals. The organisation consolidated the power of \emph{bicheiros} over the parade and provided a formal mechanism to solve disputes among the samba school patrons \citetext{\citealp[43]{cavalcanti2006carnaval}; \citealp[171]{farias2013carnival}; \citealp[55]{labronici2012paratodos}}.

\begin{figure}[!htbp]
	\centering
	\includegraphics[width=.8\textwidth, height=8cm]{images/bicho07.jpg}
	\caption{Castor de Andrade is shown celebrating after the samba school he sponsored, Mocidade Independente, won the Carnival parade in 1996. He was surrounded by colleagues and police officers. Andrade was the founding president of LIESA (1984--1985) and the wealthiest \emph{bicheiro} of Rio de Janeiro at the time. Source: Folha Imagem. Reproduced in \citet[139]{misse2007illegal}.}
	\label{fig:castor}
\end{figure}

The funding of the samba schools had an indirect effect to the animal game. The patronage also reduced agent-principal problems within the \emph{jogo do bicho}. \emph{Bicheiros} donate to samba school to gather support of the communities, and by doing so they gain access to local information on their business. Clients who have a positive image of the \emph{bicheiro} may denounce fraudsters to their superiors, thus monitoring is cost-effective for animal game managers. Thus, street bookmakers have fewer incentives to cheat. In addition, street sellers are often recruited from the poor communities, so they tend to be immediate beneficiaries of \emph{bicheiros}'s donations \citep{bbc2012aniz}. Hence, funds donated to samba schools and other charities organisations help align the interests of different members of the \emph{jogo do bicho} organisation. The patronage can be interpreted as an illegal version of `profit-sharing', a mechanism which has induced effectively cooperative behaviour in both small and large corporations \citep{cahuc1997profit,fitzroy1987cooperation, kruse1992profit}.

The samba schools have profited from this association too. First, they have gained autonomy from the government. The samba schools do not need to rely exclusively on public funds to organise the parade, and money from the \emph{jogo do bicho} permitted the schools to act independently \citep[209]{chinelli1993vazio}. Second, the support of the \emph{jogo do bicho} has increased the political and social clout of the samba schools. In a country where the state is not present throughout the territory and human right abuses are frequent \citep{ahnen2003,odonnell1993state,pinheiro2000,pinheiro2001}, \emph{jogo do bicho} bankers, and more recently drug traffickers, have provided private governance to poor areas of Rio de Janeiro by enforcing property rights, mediating disputes, and preventing police abuse in the favelas \citep{arias2006dynamics,goldstein2013laughter,leeds1996cocaine}. In return for funds and protection from the \emph{bicheiros}, samba schools have served as intermediaries between the underworld and the political system. Although the \emph{banqueiros} are interested in weak law enforcement against the animal game, politicians have resorted to samba schools to contact \emph{bicheiros} and use their financial and electoral influence in the shanty towns \citep[17]{misse2011crime}. The samba schools, therefore, have increased their bargaining power in the political sphere and extended their reach within Rio's poor communities \citep[215]{chinelli1993vazio}.

\subsection{Political Support}

If politicians were opposed to the \emph{jogo do bicho} in the early twentieth century, their relationship with the animal game bankers have become more ambivalent in the last decades. The collaboration between public authorities and \emph{bicheiros} gained prominence during the military dictatorship (1964--1985) \citetext{\citealp{gaspari2002ditadura}; \citealp{jupiara2015poroes}; \citealp[39]{zaluar2007democratizaccao}}. Given the absence of democratic checks and balances, paramilitaries and police forces colluded to repress potential dissidents of the regime and, frequently, to extort civilians \citep{gorender1999combate,magalhaes1997logica,misse2009acumulaccao,skidmore1990politics}. \emph{Bicheiros} saw the corruption of some members of the military as an opportunity. Wealthy \emph{jogo do bicho} bankers hired rogue police officers not only to work as security guards but to threaten eventual competitors in their regions of influence. The agreement between \emph{bicheiros} and corrupt members of the military was the ultimate responsible for the transformation of the \emph{jogo do bicho} into a `coercive oligopoly' \citep{jupiara2015poroes}. The support of the armed forces meant that new groups would be prohibited from entering the market and that the illegal lottery could operate undisturbed by the government.

The links between \emph{bicheiros} and the public authorities changed after Brazil became a democracy in 1985. In the military regime, government officials were mainly interested in bribes from the animal game. But in the democratic period, votes became a sought-after political resource. \emph{Bicheiros} are important in this sense as they have direct influence over a number of poor communities. Their patronage networks ensure that candidates supported by \emph{bicheiros} receive a substantial amount of votes from areas where campaigning is too difficult or too costly \citep[17]{misse2011crime}.

The Brazilian political system is particularly conductive to clientelistic practices. Brazil has one of the most fragmented party systems in the world, which induces political entrepreneurs to run highly individualised campaigns \citep{figueiredo2000presidential,geddes1992institutional}. In addition, Brazil uses a open-list proportional representation electoral system, that is, each of the 27 states of the federation are considered as at-large electoral districts \citetext{\citealp{ames1995electoral}; \citealp{mainwaring1992brazilian}; \citealp[483]{samuels2000ambition}}. These two elements indicate that Brazilian politicians are often free from the strong requirements of political parties and can run their campaigns with a high degree of independence. Nevertheless, that independence means candidates rely mostly on themselves to raise funds and establish communication with potential voters. Hence, political campaigns in Brazil tend to be expensive and personality-centred.

The support from the \emph{jogo do bicho} mitigates both problems. With respect to the financial costs of campaigns, illegal donations from \emph{bicheiros} help to cover advertising expenses while having the additional benefit of not appearing in the official records of the candidates \citep{congressoemfoco2007bicho,gazetadopovo2007bicho,globo2012bicheiro}. This suggests that \emph{jogo do bicho}-funded politicians can circumvent spending limits and have an electoral advantage over their competitors. As candidates do not know whether their competitors receive funding from the \emph{jogo do bicho} nor the amount each one was paid, their dominant position is to contact the \emph{bicheiros} and join their networks. The situation is a prisoner's dilemma in which candidates would be better off by running cheaper campaigns and not being dependent of \emph{jogo do bicho} bankers, but asymmetric information prevents them from reaching an optimal solution.

\begin{figure}[!htbp]
	\centering
	\includegraphics[width=.6\textwidth, height=8cm]{images/bicho08.jpg}
	\caption{Political advertising for Abraãozinho David (right), nephew of the \emph{jogo do bicho} banker Aniz Abraão David (third from left to right). The banner reads: `The Candidates of the Abraão Family: Fighting for the People is a Family Heritage'. Source: \citet{extra2012aniz}.}
	\label{fig:aniz}
\end{figure}

The votes from poor communities are instrumental for aspiring politicians. Brazil has an enforced compulsory voting system; therefore, turnout rates tend to be higher than in other democracies. Consequently, votes have high marginal utility for politicians. As elections may be decided by a small difference, the \emph{bicheiros}' clientelistic ties guarantee a minimum number of votes that politicians can rely upon on election day. Nonetheless, the patronage subverts the preferences of the public and, as such, the democratic process per se. Individuals may be punished if the candidate does not receive the expected number of votes, and are often compelled to vote for politicians that have only loose connections with their communities. Therefore, although voters have the right to choose their representatives, in practice the suffrage is limited for a share of Brazil’s lower classes.

Finally, the \emph{jogo do bicho} patronage highlights a crucial social dilemma within the Brazilian public law. Even though federal judges have prosecuted \emph{bicheiros}, politicians and police forces have no incentives to enforce the punishment. Although Brazilian judges enjoy job stability, the latter groups constantly require local-level support from the \emph{bicheiros}. Politicians and police officers may have accurate information on \emph{jogo do bicho} operations and \emph{bicheiros}' whereabouts, but the federal government cannot rely upon their cooperation. That can be one of the reasons why even after many attempts to arrest \emph{bicheiros}, there has been little progress in that regard in Brazil's latest democratic period (1985--present).

\section{Conclusion}
\label{sec:conclusion}

Past research has shown that criminal organisations face considerable challenges to elicit cooperation from their members and establish close ties with the population \citep[e.g.][]{gambetta1996sicilian,skarbek2011governance,skarbek2012prison,varese2001russian,varese2011mafias}. Yet, the \emph{jogo do bicho} offers a convincing example that it is possible for an illegal syndicate to operate with low levels of violence for more than a hundred years. \emph{Bicheiros} employ a number of strategies to obtain reliable information from their subordinates while offering club goods and other selected benefits to workers. Furthermore, by investing in the Carnival parade \emph{bicheiros} have been able to gather popular and government support. Poor communities have associated with the \emph{bicheiros} to receive welfare provision, whereas politicians have collaborated with them to reap the financial and electoral benefits the \emph{jogo do bicho}'s networks can provide.

Nevertheless, the \emph{jogo do bicho} has also created negative externalities. Violence is used to punish defectors and to constrain competitors. The clientelistic relationship that \emph{bicheiros} have with local politicians have lead to sub-optimal outcomes, such as predatory political campaigning, distortions in electoral representation, and impunity for human rights violations. These negative externalities have long-term effects and still impact the Brazilian public sphere.

Although the \emph{jogo do bicho} has received an increasing attention from scholars, much of its inner workings remain poorly understood. First, the relationship between \emph{bicheiros} and drug dealers is a topic that deserves attention. Brazil has become one of the world's largest consumers of illicit drugs and South America's principal drug trafficking transit route \citep{miraglia2015drugs,misse2011crime}. The question whether \emph{bicheiros} collaborated or opposed the emergent drug dealing business is still unclear. Second, the extent to which \emph{bicheiros} use other businesses, such as hotels or factories, to laundry money has been mentioned by members of the Brazilian judiciary \citep{globo2012bicheiro,globo2015cacaniquel}; however, there is no reliable estimate on its size. Lastly, more research is required to clarify how \emph{bicheiros} from different parts of Brazil coordinate their activities and prevent large-scale conflicts. Cases studies are usually focused on Rio de Janeiro's \emph{bicheiros}, but scholars would benefit from comparative analyses with a larger number of states. This is an important step to elucidate how \emph{bicheiros} continue to influence politics and the public across Brazil.



%-----------------------------------------------------
% Chapter 4: Mass Killings
%-----------------------------------------------------

\chapter{What Drives State-Sponsored Violence?: Evidence from Extreme Bounds Analysis and Ensemble Learning Models}
\label{chap:killings}

\section{Introduction}
\label{sec:intro4}

Since the end of World War II, mass killings, genocides, and politicides have claimed over 34.5 million lives \citep{marshall2017pitf}.\footnote{Genocide and politicide are the attempted intentional destruction of communal or political groups, respectively \citep[see][]{harff1988toward}. Mass killing includes these atrocities, as well as attacks against civilians that result in at least 1,000 deaths but are not intended to destroy a particular group \citep[see][]{ulfelder2008assessing}. While some conflate the logic of these types of atrocities \citep[e.g.,][]{stanton2015regulating, valentino2004draining}, others claim genocide and politicide follow a different logic from other forms of government violence \citep{kalyvas2006logic,stanton2015regulating}. Note that these arguments focus on the logic of the violence, not legal definitions that separate and prioritise genocide as the crime of crimes \citep[see][]{schabas2000genocide}.} The international community has responded with an effort to prevent further state-sponsored mass murder by strengthening laws against crimes against humanity, genocide, and war crimes. It formed \textit{ad hoc}, hybrid, and a permanent criminal court to punish and deter future atrocities. Furthermore, the United Nations established a Special Adviser on the Prevention of Genocide and recognised its members’ responsibility to protect civilian populations within and outside their own borders. Yet, such atrocities still occur. Recently, President al-Assad of Syria has massacred tens of thousands of civilians during the Syrian Civil War \citep{goldman2017nyt}. Similarly, South Sudan's President Kiir is actively starving and killing civilians from dissident and rival tribes \citep{nichols2017reuters}. While there is some evidence that such atrocities may be declining since the Cold War \citep{valentino2014we}, the international community has been far from successful in realising slogans like ``Never Again'' and ``Not on My Watch'' \citep{cheadle2007not}.

Ultimately, successful prevention requires us to understand why these atrocities occur. In this vein, the academic community has laboured tremendously to establish evidence-based theories as to why governments engage in brutality against their civilian populations. Indeed, since 1995, there have been over 45 quantitative political science articles focused on explaining government-sponsored killing of civilians. Note, this number does not consider the myriad other books, qualitative articles, and non-political science research conducted into this matter. Overall, the mass violence literature generally agrees that as threat increases, so does the likelihood of atrocity, if the costs to such violence are not prohibitive. However, there is little consensus on what factors influence the level of threat or costs a regime faces. Part of the reason for this uncertainty is that scholars use very different model specifications when testing their arguments. Examining the quantitative literature on government violence against civilians, we found that in 45 studies, scholars used nearly 180 measurements to capture roughly 30 key concepts related to threat and costs. While models should be constructed specifically to test particular arguments, one concern is that small changes in model specification could in influence the robustness of empirical results and the inferences that we can draw from these results.

To overcome these limitations and provide a better understanding of government atrocity, we employ extreme bounds analysis and random forests to identify the most robust determinants of state-sponsored atrocities. Our approach is similar to \cite{hegre2006sensitivity} seminal analysis on the causes of civil war onset, but we provide additional tests to check which variables are able to predict out-of-sample cases of mass violence during both peacetime and wartime. In conducting this analysis, we address three debates in the mass violence literature:

\begin{enumerate}
    \item Why do some governments engage in mass killings, genocides, or politicides? This is the primary question asked by advocates, policymakers, and scholars in this field of research. It is also the question that the 45 studies mentioned above each attempt to answer.
    \item Does the logic underpinning government decision-making follow different patterns during peacetime and wartime. Recent research suggests that government atrocity occurs predominantly during periods of civil war \citep{harff2003no} which has led some scholars to restrict their analyses to only periods of civil war \citep[e.g.,][]{colaresi2008kill, valentino2004draining} or concentrate on predicting both the onset of civil war and atrocity \citep{goldsmith2013forecasting}. Yet, others estimate models of all country-year \citep[e.g.,][]{krain1997state, montalvo2008discrete}, raising questions of how well these studies speak to each other.
    \item Is there a difference in logic between those atrocities labelled as genocide or politicide, compared to other mass killings? While the Political Instability Task Force \citep{marshall2017pitf} provides the most widely used data on government atrocity, these data are limited only to cases of genocide and politicide. Others provide data with much more lenient inclusion criteria \citep[e.g.,][]{eck2007one, rummel1995democracy, stanton2015regulating, ulfelder2012forecasting}. These differences in definition of atrocity have led to divergent results, raising questions about important determinants of government behaviour \citep[for discussion, see][]{uzonyi2016domestic, wayman2010explaining}.
\end{enumerate}

Our analysis tests the sensitivity of 40 variables on a sample of 177 countries from 1945 to 2013. We find that GDP per capita, stable political regimes, and the post-Cold War period are negatively associated with mass killings. Conversely, previous political turmoil, ethnic diversity, and civil wars increase the risk of such violence. The distributed random forest algorithm indicates that years since the last episode of mass violence, GDP per capita, urban population, ethnic polarisation, the number of military personnel, and democracy make the greatest contribution to our models' out-of-sample predictive power. While we discuss the importance of these findings in more detail later in this article, a few notes are imperative to make here. First, in some ways, these findings confirm previous research–unstable countries are more likely to witness the regime employ atrocity \citep[e.g.,]{goldsmith2013forecasting,harff2003no,krain1997state}.  However, many of the factors scholars often cite as observable indicators of such instability--regime transitions, coups d'état, the presence of militias, etc.--are not good proxies for instability. Thus, policymakers may be looking for the incorrect signs of impending atrocity when seeking to prevent its onset. Second, these findings raise concerns about policy options for preventing violence against civilians. If our conclusion is that unstable countries are violent, then preventing atrocity likely requires significant investments of time and resources in state-building, which is often politically and practically unfeasible \citep{doyle2006making}. Lastly, our findings are robust to a myriad of specifications, functional forms, and additional tests as we describe below. This analysis contributes significantly to the political violence literature by highlighting the parsimonious nature of the logic behind government atrocity and clearing away much of the empirical clutter surrounding this conclusion. 

\section{Empirical Methods}
\label{sec:methods4}

We employ two methods to test the robustness of the potential determinants of state-led violence. First, we use a variant of extreme bounds analysis (EBA) to check the statistical significance of 35 explanatory variables cited in the academic literature.\footnote{The complete list of variables included in this article is available in appendix \ref{sec:mk-appendix}. Our selection procedure was as follows. We included all variables that appeared in at least two two quantitative papers on mass killings which employed logit or probit models in a global sample. We considered only published articles in our list.} Researchers have employed the EBA to assess the sensitivity of the determinants of civil war \citep{hegre2006sensitivity}, coups d'état \citep{gassebner2016expect}, democratisation \citep{gassebner2013extreme}, economic growth \citep{levine1992sensitivity, sala1997just, sturm2005determinants}, nuclear deterrence \citep{bell2015examining}, political repression \citep{hafner2005right}, and state corruption \citep{serra2006empirical}. The method is particularly useful when there is no consensus about which covariates belong in the ``true'' regression model \citep[178]{sala1997just} and scholars worry that omitted or unnecessary predictors could bias the parameter estimates \citep[60]{angrist2008mostly, clarke2005phantom, elwert2014endogenous, spector2011methodological}. By combining a large set of regression models into a single posterior distribution, the EBA reduces the uncertainty associated with the \textit{ad hoc} selection of predictors and provides an intuitive way of assessing the relative strength of the statistical findings \citep{leamer1985sensitivity, sala1997just}.

Second, we employ distributed random forest (DRF) \citep{breiman2001random, h2o2017} to measure the predictive power of our set of independent variables. Random forest has been widely used in the machine learning community and consistently ranks among the best algorithms for predicting structured data.\footnote{Random forests, gradient boosting machines, and neural networks have won the highest number of competitions on Kaggle, a crowd-sourced platform for predictive modelling \citep{carpenter2011may}. While data analysts prefer neural networks for problems with a high number of hidden features such as image recognition, random forest and gradient boosting are the top choices for dealing with tabular data. See: \href{https://www.kaggle.com/antgoldbloom/what-algorithms-are-most-successful-on-kaggle}{https://www.kaggle.com/antgoldbloom/what-algorithms-are-most-successful-on-kaggle} (access: December 2017).} DRF does not require distribution assumptions, can be used with any type of response variables, and are able to automatically detect nonlinear relationships between correlates \citep{fernandez2014we, hill2014empirical, jones2015exploratory, muchlinski2015comparing}. Moreover, unlike neural networks and other deep learning algorithms \citep{castelvecchi2016can,rojas2013neural,shwartz2017opening}, the output of DRF models can be interpreted meaningfully. DRF offers precise estimates of the predictive ability of each individual variable, and it also allows researchers to graphically visualise the marginal effect of the covariates on the response variable \citep{friedman2001greedy,friedman2001elements,goldstein2015peeking}.


\subsection{Extreme Bounds Analysis}
\label{subsec:eba}

The main purpose of the extreme bounds analysis is to estimate the distribution of coefficients of each predictor $x$ in an exhaustive combination of regression models with $y$ as a dependent variable. \cite[308]{leamer1985sensitivity} argued that the EBA could certify scholars that ``minor changes in the list of variables do not alter fundamentally the conclusions, nor does a slight re-weighting of observations, nor correction for dependence among observations, etcetera, etcetera.'' He proposed that ``sturdy'' variables are those whose minimum and maximum of their coefficient distribution have the same sign and are situated at a distance from zero. If we are to use the conventional value of $p < 0.05$, the mean of the variable coefficients' distribution should be located at least $1.96$ standard deviations away from zero. 

Leamer's criterion is intuitive, but other authors contend it is too strict for most social science applications. \cite{sala1997just} argued that Leamer's EBA would increase the number of false negatives; in other words, it would classify as fragile covariates that are truly associated with the response. If a given variable of interest appears as both positive and negative in the literature, it is likely that after a large a number of regressions the predictor will change signs and therefore be deemed fragile \citep[179]{sala1997just}. Sala-i-Martin then offered a less stringent version of the EBA in which researchers analyse the full distribution of point estimates instead of relying only on the coefficients' extreme bounds. 

In this paper, we follow his advice and consider the whole range of values of $CDF(0)$. We choose to use the whole distribution because the aggregate $CDF(0)$ allows us to move away from a binary indicator of robustness, the one we obtain with Leamer's extreme bounds, and present the estimations with their appropriate degrees of confidence \citep[179]{sala1997just}\footnote{As \citet[179]{sala1997just} describes, ``if 95 percent of the density function for the estimates of $\beta_{1}$ lies to the right of zero and only 52 percent of the density function for $\beta_{2}$  lies to the right of zero, one will probably think of variable 1 as being more likely to be correlated with [a dependent variable $Y$] than variable 2.''}. Our main focus is the percentage of the variable's cumulative distribution function that is smaller or greater than zero. We do not assume that the CDF has a normal distribution: Many of our coefficients do not follow a standard distribution, so the generic model provides a better fit to our data.\footnote{Histograms for all coefficients are available in appendix \ref{sec:mk-appendix}.} We specify our models as follows:

\begin{equation}
\text{\textit{Mass Killing Onset}}_{it} = \beta_{M}M_{it} + \beta_{F}F_{it} + \beta_{Z}Z_{it} + v_{it}
\end{equation}

Our main dependent variable is \textit{Mass Killing Onset}, which denotes the onset of government-sponsored killings. It was coded by \cite{ulfelder2008assessing}. The authors define a mass killing as ``any event in which the actions of state agents result in the intentional death of at least 1,000 noncombatants from a discrete group in a period of sustained violence'' \citep[2]{ulfelder2008assessing}. In our additional tests, we also estimate the models with an indicator of Genocide/Politicide Onset by \cite{harff2003no}. We prefer the first measure not only because it provides a clear numerical threshold that separates mass killings from low-level violence, but also because it focuses on the killing of members of a given community, not on the absolute number of deaths. 

The indices $i$ and $t$ indicate country and year, respectively. $M$ is a set of three covariates that are included in every model due to their prominence in the literature \citep{levine1992vale}. In our analysis, we have added the natural logarithm of the GDP per capita to control for income, the Polity IV index to control for level of democracy, and a linear time trend since the last episode of government-led atrocity to account for temporal dependence. $F$ denotes a vector of variables of interest, and $Z$ is a vector of other control variables in addition to those included in $M$. Our first model includes the variable of interest, 3 additional control variables, and 3 covariates that appear in every regression. $v$ is the error term. In practice, however, since we are interested in the effect of all variables in the data set and we do not have true control variables except from $M$, in our case $F$ and $Z$ are interchangeable. Following \citet[514]{hegre2006sensitivity}, the independent variables have been lagged one year to reduce the risk of endogeneity. Although our dependent variable is dichotomous, we use linear probability models in our main analysis. As argued by \citet[298]{gassebner2016expect}, linear probability models are less prone to convergence problems, their estimation is faster, and their results can be readily interpreted. While we do employ logistic and probit models as robustness tests, we report the coefficients of the linear probability models as our primary estimates. Moreover, since the data are grouped into countries, we use cluster-robust standard errors to control for within-cluster error correlation.
 
One of the shortcomings of our data is that some variables are strongly correlated. For instance, we include the Correlates of War's Composite Index of National Capability (\textit{CINC}), which is composed by six different variables \citep{cow2017cinc,singer1988reconstructing}.\footnote{For more information about the CINC index, please refer to the Correlates of War website: \href{http://www.correlatesofwar.org/data-sets/national-material-capabilities/national-material-capabilities-v4-0}{http://www.correlatesofwar.org/data-sets/national-material-capabilities/national-material-capabilities-v4-0} (access: December 2017).} Multicollinearity would induce bias to our estimates, so we add CINC and three of its components\footnote{The variables correspond to the ratio between the national and the global values.} -- military expenditure, military personnel, and total population -- as mutually exclusive variables. These predictors are not included in the same regression model to avoid collinearity issues.
 
We add another set of mutually exclusive variables to reduce collinearity. There are three covariates that denote ongoing civil conflicts: one measured by the Uppsala Conflict Data Program \citep{allansson2017organized,gleditsch2002armed}, another coded by the Correlates of War \citep{sarkees2010resort}, and a third indicating the onset of ethnic conflict as coded by \citet{cederman2010ethnic}. We include only one of these measures at a time. 
 
As a last precaution against collinearity, we place a limit on the Variance Inflation Factor (\textit{VIF}) of all regression coefficients. The VIF estimates how much of the variance of each predictor is dependent on the other covariates in a model. A VIF of 1 indicates that the predictor is uncorrelated with the remaining covariates. The VIF limits are often arbitrary \citep{bell2015examining,o2007caution}, thus we use a moderately conservative VIF of 7 in our estimates. As robustness tests, we run the same models with different VIF cut-offs and without restriction.

Two variables were omitted from the EBA models but included in the machine learning models below. The first is \textit{democracy}, a dummy variable that indicates whether the country is a has a Polity IV score equal or higher than 5. The second is \textit{interstate war}, a binary covariate measuring if the country is at war in a given year \citep{sarkees2010resort}. We have decided to omit democracy because of its obvious correlation with the Polity measure and interstate war due to its correlation with our dependent variables. The EBA models do not converge if those variables are added. Since this problem does not affect machine learning algorithms, the two variables were included in the second set of estimations.

Lastly, we depart slightly from Sala-i-Martin's suggested method and do not assign weights to the EBA. Although he recommends using goodness-of-fit measures to construct regression weights, we agree with \citet{sturm2002robust} and \citet[299]{gassebner2016expect} and use the unweighted version of the CDF instead. Goodness-of-fit indicators are not equivalent to the probability of a given model being true \citep{achen1977measuring,anscombe1973graphs,king1986not}, and the weights constructed this way are not invariant to transformations in the dependent variable. Moreover, our data set has a number of missing observations, so that model comparison would be misleading \citep{lall2016multiple}. Thus, the results below are for unweighted estimations. 

\subsection{Distributed Random Forest}
\label{sub:drf}

Random forest is a machine learning algorithm that consists of a combination of individual decision trees. In a classification problem, each decision tree uses a vector of covariates to split the dependent variable into two increasingly homogeneous parts \citep{breiman2001statistical}. However, decision trees are prone to overfitting, i.e., they match the original data set so closely that they tend to perform poorly with new data \citep{dietterich1995comparison,ho1998random}. Random forest, in contrast, avoids this issue by growing a decision tree only to a bootstrap sample of the original data, selecting random features at each split, then aggregating the different trees into a single prediction. If the independent variable is continuous, the algorithm will simply choose the average value of the predictions as the best candidate; if the covariate is discrete, the majority class will be employed. The simple procedure of leaving out some data points and growing separate trees with a random subset of covariates is sufficient to eliminate the risk of overfitting \citep[9-10]{jones2015exploratory}.

Random forest has many desirable properties, such as ``highly accurate predictions, robustness to noise and outliers, internally unbiased estimate of the generalisation error, efficient computation, and the ability to handle large dimensions and many predictors'' \citep[7]{muchlinski2015comparing}. Thus, random forest allows the researcher to estimate very flexible models with minimal assumptions. Unlike parametric methods such as ordinary least squares or logistic regressions, the analyst does not have to impose any distributional form to the data-generating process. As a result, random forest is able to effectively uncover complex, nonlinear interaction effects in the data without prespecification \citep{jones2015exploratory,strobl2007bias}.

The algorithm can also deal with heavily imbalanced data. Researchers can balance the classes of their response variable by assigning different sampling probability to each category. This enables the algorithm to make accurate prediction for rare events without adding prior distributions or resorting to subjective modelling choices \citep{chen2004using,del2014use,muchlinski2015comparing}.

In this paper we use distributed random forest (DRF) to model our data \citep{h2o2017}. The DRF is essentially identical to original random forest algorithm, but it has two additional features that are useful for our purposes. Firstly, DRF is optimised for big data, as it grows decision trees on separate cores to speed up computation time. Secondly, in DRF, non-observed cases are not assumed to be missing at random, but rather as values that contain information in themselves. When building decision trees, DRF treats the missing observations as a separate category that can go either left or right. This is a more conservative approach than assuming that missing cases fit into an underlying parametric distribution.\footnote{For more information about how the distributed random forest algorithm deals with missing observations, please refer to: \href{http://docs.h2o.ai/h2o/latest-stable/h2o-docs/data-science/drf.html}{http://docs.h2o.ai/h2o/latest-stable/h2o-docs/data-science/drf.html} (access: December 2017).}

The DRF has a series of hyperparameters that can be tuned to improve the algorithm's predictive performance. For instance, users can control the number of decision trees in each iteration, how deep trees should grow, which performance metric to adopt, whether to balance classes or to use cross-validation, and many other options. The interaction between parameters is generally complex and may involve thousands of potential combinations. As an example, a researcher interested in four parameters with 10 possible values each would have to estimate 10,000 models before deciding which is the most efficient one. Also, machine learning parameters are sensitive to the data at hand; so what seems an optimal solution for one problem cannot be readily implemented in another data set \citep{genuer2008random,goldstein2010application,jones2015exploratory}.

To address this issue, we have adopted an automated procedure to select the model parameters. We perform a grid search where the algorithm starts with a random combination of parameters, jumps to another randomly-chosen set, and then stops after it reaches a certain threshold \citep[123]{cook2017h2o}. We follow the literature on predictive political science and use the area under the ROC curve (AUC) as our model evaluation metric \citep[e.g.,][]{clayton2014will,hill2014empirical,ward2010perils,ward2013learning,weidmann2010predicting}. We set the metric as follows: If five random models had not increased the AUC by at least in 0.1\% comparing to the previous ones, the algorithm considers the result to be optimal. We set the maximum number of models to 1,000.

We have added several parameters to the grid search. The first is the number of independent trees to grow in each random forest model. The starting values are 256, 512, and 1024 trees. The machine learning literature does not provide a heuristic on how large a random forest should be, but \citet[166]{oshiro2012many} affirm that ``from 128 trees there is no more significant difference between the forests using 256, 512, 1024, 2048 and 4096 trees.'' We employ a more conservative approach and start from a higher value that the authors suggest as adding more trees do not reduce prediction accuracy \citep[7]{breiman2001statistical}. However, we find little evidence that the number of trees makes a noticeable difference to the estimations.

The depth of each decision tree also influences the algorithm performance. Deeper trees indicate more complex models, and in general they provide a better fit to the data. Nevertheless, this complexity comes at the risk of overfitting, so deeper trees are not necessarily the most adequate solution for every model \citep[596]{friedman2001greedy,segal2004machine}. In this article, we let the algorithm decide among using 10, 20, or 40 levels for each tree. In our view, these number offer a good balance between parsimony and complexity.

We test whether having balanced classes of our dependent variable (mass killing onset) affects the predictive ability of the model. Since the response measure is heavily imbalanced, oversampling the positive responses could potentially improve our results \citep{chawla2004special,del2014use,japkowicz2002class}. Somewhat unexpectedly, DRF had a better fit using unbalanced classes, which is the result we report in the next section.

We also vary how many variables should be considered for each split in the data. The DRF's default is to use $\sqrt{p}$, where p is the number of columns in the data set. As we have 34 covariates of interest, we have selected 5, 6 and 7 variables per split. The DRF uses a majority voting procedure to select which variable is most important. Additionally, the algorithm chooses the percentage of the training set to be modelled by each tree. The default option is 63.2\%, but we include the options of using 50\% and 100\% of the data. Similarly, we give a range of options for choosing how many columns will be included in each tree. The algorithm can randomly choose among 50\%, 90\% or 100\% of the independent variables when estimating a decision tree.

Finally, we use four types of histogram to find optimal split points for each independent variable. Decision trees consider every value of a given independent variable as a potential candidate for a split in the training data. This process is notably time-consuming, and computation time can be significantly reduced at little loss of precision by taking discrete values of the predictor distribution. The DRF algorithm offers four choices of histogram selection and we include all of them in our estimations.

\section{Results}
\label{sec:results4}

Table \ref{tab:eba1} summarises our main EBA results with Ulfelder and Valentino's \citeyear{ulfelder2008assessing} \textit{Mass Killing Onset} as the dependent variable. The table shows the average coefficient estimate of all regressions for each robust variable along with their mean standard deviations.\footnote{A list of all independent variables and coding rules are available in the online appendix.} The table also displays the percentage of regressions that are statistically significant at the 90\% level. $CDF(0)$ represents the cumulative distribution function, which is the area of the distribution that falls above or below zero.\footnote{We show whichever area is the largest. The sign of the average $\beta$ coefficient indicates if most of the cumulative distribution is located above or below zero.} This is our main statistic of interest, and we consider a covariate to be robust if it has a $CDF(0)$ of 0.9 or higher \citep[181]{sala1997just}. Lastly, we report the number of estimated regressions models which included each variable.

\vspace{1cm}

\begin{table}[H]
\centering
\begin{tabular}{lrrrrr}
\hline
\textbf{Variable} & \textbf{Avg. $\beta$} & \textbf{Avg. SE} & \textbf{$\%$ Sig.} & \textbf{CDF(0)} & \textbf{Models} \\ \hline
\textit{Base variables} &  &  &  &  &  \\
Log GDP per capita & -0.0091 & 0.0052 & 76.055 & 0.9335 & 226707 \\
 &  &  &  &  &  \\
\textit{Additional variables} &  &  &  &  &  \\
Post-Cold War years & -0.0133 & 0.0085 & 72.845 & 0.9472 & 35614 \\
UCDP civil war onset & 0.0529 & 0.0321 & 52.378 & 0.9441 & 20854 \\
Previous riots & 0.0140 & 0.0100 & 56.242 & 0.9216 & 35614 \\
UCDP ongoing civil war & 0.0172 & 0.0115 & 65.652 & 0.9092 & 20854 \\
Ethnic diversity (ELF) & 0.0184 & 0.0137 & 56.674 & 0.9050 & 35614 \\
Polity IV squared & -0.0002 & 0.0001 & 61.206 & 0.9031 & 35614 \\ \hline
\end{tabular}
\caption{Extreme Bounds Analysis -- Mass Killings (Robust Variables Only)}
\label{tab:eba1}
\end{table}

Seven variables pass our EBA criteria. Three variables decrease the likelihood of mass killings. First, as widely suggested in the literature, the natural logarithm of GDP per capita is robustly associated with the onset of mass killings 
\citep[e.g.,][]{besanccon2005relative, easterly2006development,esteban2015strategic}. In our models, GDP per capita has a negative effect on onset risk is statistically significant in 76\% of the models. Second, the post-Cold War years are correlated with lower levels of government violence. Despite eventual outbreaks of large-scale attacks against civilians \citep[e.g.,][]{prunier2005darfur,tatum2010genocide,waldorf2009revisiting}, after 1991 these events have become less prevalent. Indeed, this finding is in line with several studies that point to a general decline in violence over the last decades, including riots, civil wars, and urban crime \citep{eisner2003long,pinker2011better}. The third variable negatively correlated with mass killings is the squared term of the Polity IV political regime index. Although the linear Polity IV indicator does not reach our threshold for significance, about 90\% of the distribution area of Polity IV squared is located below zero. This finding points to a nonlinear relationship between political regime and mass killings, thus providing further evidence that democracy reduces state-sponsored violence \citep{rost2013will,rummel1995democracy} and that regimes that mix democratic with autocratic features have the highest risk of conflict \citep{hegre2001toward,mitchell2013domestic,muchlinski2014grievances,regan2010changing}.

Four variables are positively associated with Ulfelder and Valentino's (\citeyear{ulfelder2008assessing}) indicator of state-sponsored violence. The onset and continuation of civil wars are correlated with mass killings, but only when we employ the UCDP measures of violent conflict. We find no effect for the variables compiled by the Correlates of War project or \cite{cederman2010ethnic}. Coding choices likely account for this difference, and the result is robust to a myriad of specifications. Former instances of political turmoil also have a positive coefficient in our models. Countries with a previous history of riots are more prone to state violence, which suggests that government repression is path dependent 
\citep[e.g.,][]{gurr2000peoples,harff2003no,krain1997state,nyseth2017re}. Our results also show that a higher levels of ethnic diversity increase the likelihood of atrocities against civilians. Nevertheless, the variable does not pass all additional tests we implement below and the sturdiness of this finding remains open to question. The uncertainty about the impact of ELF on mass killings is also reflected in the literature. As \citet[237]{hoeffler2016development} points out, \citet{rummel1995democracy} reports that ELF is not statistically significant in his models, \citet{wood2014opportunities} finds a negative relationship and \citet{querido2009state} finds a positive one. Our analysis does not offer a definitive answer for this question either.

A different set of variables reach significance if we only consider mass killings that occur during civil wars. We ran the models on three indicators of violent armed conflict. The first is the civil war measure by the Uppsala Conflict Database Program  (\citeyear{allansson2017organized,gleditsch2002armed}). Here, two variables meet our criteria for robustness, Post-Cold War years and conflicts with territorial aims, both negatively correlated with mass killing onset. When we limit our analysis to cases marked as civil wars by the Correlates of War project \citep{sarkees2010resort}, years since the last mass killing episode, previous riots, former human rights abuses, and ethnic diversity appear have a positive effect on the outcome. Rather surprisingly, the presence of militias has a negative effect on genocides and politicides. Finally, our model using ethnic civil war \citep{cederman2010ethnic} shows similar results: Militias have a negative impact on the likelihood of mass killings during ethnic conflict, as do conflicts over territorial aims. This finding is in contradiction with previous research \citep{koren2017means} and is deserving of further exploration. Overall, these results suggest two takeaways. First, the determinants of mass killing onset during peacetime and wartime are different from each other. While there are more restraints on atrocities during peacetime, these restraints fail once fighting begins. This raises particularly interesting questions about the pacifying nature of democracy. Second, how civil wars are classified significantly alters the results produced. This raises concerns for scholars about making claims based on only one data set.

\vspace{1cm}

\begin{table}[H]
\centering
\begin{tabular}{lrrrrr}
\hline
\textbf{Variable} & \textbf{Avg. $\beta$} & \textbf{Avg. SE} & \textbf{$\%$ Sig.} & \textbf{CDF(0)} & \textbf{Models} \\ \hline
\textit{UCDP data} &  &  &  &  &  \\
Territory aims & -0.044 & 0.019 & 74.997 & 0.9804 & 17902 \\
Post-Cold War years & -0.038 & 0.019 & 66.574 & 0.9222 & 17902 \\
 &  &  &  &  &  \\
\textit{COW data} &  &  &  &  &  \\
Physical integrity & 0.024 & 0.013 & 66.674 & 0.9564 & 17902 \\
Militias & -0.099 & 0.048 & 73.104 & 0.9490 & 17902 \\
Years since last mass killing & 0.006 & 0.002 & 88.208 & 0.9472 & 101583 \\
Previous riots & 0.078 & 0.041 & 65.412 & 0.9348 & 17902 \\
Ethnic diversity (ELF) & 0.095 & 0.062 & 48.615 & 0.9000 & 17902 \\
 &  &  &  &  &  \\
\textit{Cederman et al. data} &  &  &  &  &  \\
Territory aims & -0.051 & 0.026 & 74.288 & 0.9167 & 17902 \\
Militias & -0.050 & 0.035 & 52.240 & 0.9101 & 17902 \\ \hline
\end{tabular}
\caption{EBA -- Mass Killings during Civil Wars (Robust Variables Only)}
\label{tab:ucdp}
\end{table}

The random forest models confirm some of the main findings of the EBA, yet they also show some interesting prediction patterns. Figure \ref{fig:drfuv} presents the six most important predictors of state-sponsored violence in out-of-sample tests. Overall, the machine learning estimations have a good fit, with an AUC of about 0.8 in the test samples.

However, the findings should be interpreted with caution since variable importance metrics are sensitive to the choice of random seed numbers. But despite that expected variation, the results in the random forest models are quite stable even when estimated with different random seeds. 

\begin{figure}[h]
\includegraphics[width=.9\textwidth, height=8cm]{images/drf.pdf}
\caption{Distributed Random Forest -- Variable Importance (Scaled)}
\label{fig:drfuv}
\end{figure}

We see that the most important predictor in our models is the natural log of GDP per capita. This confirms the findings of the EBA model. Years since the last mass killing episode has a nonlinear relationship with large-scale violence: The likelihood of renewed genocides is small in the first years, although it increases slightly over the next decades. The percentage of urban population exhibits a similar pattern. Rural areas are less prone to mass killings, yet the probability of such events rise according to a country’s level of urbanisation. We understand both GDP per capita and percentage of urban population as proxies for state capacity. Conducting a mass killing campaign demands some minimal level of military hierarchy and resources, and countries that are overwhelmingly rural are probably less able to amass the human and physical capital required to conduct such task. Furthermore, urban population tend to be less dispersed, what facilitates a large attack by government forces. 

\vspace{.5cm}

\begin{figure}[h!]
\begin{center}
\includegraphics[width=\textwidth, height=9cm]{images/drfdpp.pdf}
\caption{Distributed Random Forest -- Partial Dependence Plots}
\label{fig:drfdpp}
\end{center}
\end{figure}

The ethnic polarisation variable indicates that countries in which a large share of the population is excluded from politically relevant positions, the lower the probability of a mass killing. Groups that manage to control the state apparatus rarely face dangerous opposition movements, and as such they rarely resort to mass violence to impose their rule over the population. Next, the size of the military personnel substantially increases the likelihood of state-sponsored violence. While small and weak states tend to be more prone to civil war and other types of conflict \citep{collier2004greed,fearon2003ethnicity}, they are less likely to experience one-sided state violence. Lastly, the Polity IV index shows that mixed regimes have a higher likelihood of atrocity, and democracies tend to have lower levels of state-led violence.

With respect to mass atrocities that occur during armed conflicts, we see that log GDP per capita and years since mass killings are again relevant predictors. Additionally, both covariates are among the 10 most important predictors in the three models. The results obtained with the UCDP civil war data are broadly similar to those of the previous model, but it highlights that total battle deaths is the predictor that has a noticeable impact on mass killings onset. The partial dependence graph shows a large positive effect along the curve, with a pronounced jump at around 200,000 deaths. 

\vspace{1cm}

\begin{figure}[h!]
\begin{center}
\includegraphics[width=\textwidth, height=9cm]{images/drfdpp2.pdf}
\caption{Partial Dependence Plots -- Mass Killings during Civil Wars (UCDP data)}
\label{fig:drfdpp2}
\end{center}
\end{figure}


\begin{figure}[h!]
\begin{center}
\includegraphics[width=\textwidth, height=9cm]{images/drfdpp3.pdf}
\caption{Partial Dependence Plots -- Mass Killings during Civil Wars (COW data)}
\label{fig:drfdpp3}
\end{center}
\end{figure}


\begin{figure}[h!]
\begin{center}
\includegraphics[width=\textwidth, height=9cm]{images/drfdpp4.pdf}
\caption{Partial Dependence Plots -- Mass Killings during Civil Wars (Cederman et al. data)}
\label{fig:drfdpp4}
\end{center}
\end{figure}

\newpage

When using the COW data, we find that GDP per capita and total battle deaths have little influence in the likelihood of mass killings, and years since the last state-led atrocity increases the chances of a future event, a result we have also seen in the EBA models. As expected, higher levels of abuse against the physical integrity of citizens are associated with mass killings. Ethnic polarisation and the threshold of excluded population have a somewhat constant effect in our estimations, yet both show a sharp decline in probability when they reach their maximum levels. This provides more evidence to the idea that when certain groups are able to concentrate power, they tend not to resort to mass-scale violence as a political means.

Our last model concerns the onset of ethnic conflicts as defined by \citet{cederman2010ethnic}. The graphs show ethnic wars operate with a different dynamics than other armed conflicts. We see that an increase in all the most important predictors have a large negative impact on the likelihood of mass killings, thus we conclude that poor, unstable countries are more prone mass killings during ethnic conflicts.

\section{Additional Tests}
\label{sec:additional-tests4}

We estimate a set of additional regressions to assess the robustness of our main findings. In regard to the EBA, we include 10 variants of our original model. All of them largely confirm our prior results. First, we varied the number of covariates included in each regression to 3 and 5 while keeping the $M$ set of 3 variables included in every model, that is, the natural logarithm of GDP per capita, the Polity IV index, and a linear time trend since the last episode of state-sponsored violence. The results are identical to the main model when we add 3 control variables at a time. The same covariates discussed above remain significant and with the same sign: log of GDP per capita (negative), post-Cold War period (negative), previous riots (positive), civil war onset and ongoing indicators as measured by the UCDP data set (both positive), Polity IV squared (negative), and ethnic diversity (negative). When 5 control variables are added to every model, three variables have a CDF(0) of 0.9 or higher. They are the natural logarithm of GDP per capita, civil war onset (UCDP), previous riots, and the dummy indicating the post-Cold War period. Ethnic fractionalisation and Polity IV squared become marginally significant with a CDF(0) of about 0.88.

Second, we place different restrictions on the variance inflation factor (VIF) to test whether multicollinearity is driving our results. The test is similar to that employed by Bell (2015). While in the main analysis we use a VIF of 7, in the additional analyses we include a conservative VIF of 2.5, a less strict value of 10, and a model without any restriction. The results are unchanged. The two models with different values of VIF show identical results to the results displayed in table \ref{tab:eba1}. In the model with no VIF restriction, however, ethnic fractionalisation fails to meet our threshold by a very small margin. The CDF(0) of that covariate is 0.897, very close to the required value of 0.9.

We also estimate our models using logit and probit regressions. In order to deal with the issue of complete separation \citep{bell2015questioning,zorn2005solution} we follow \citet{gelman2008weakly} and add a weakly informative prior distribution to the coefficients. We first scaled the non-binary variables to have a mean of 0 and a standard deviation of 0.5, then we added a Cauchy distribution with center 0 and scale 2.5. The probit regressions use a scale of $2.5 \times 1.6$, which is also recommended by the authors \citep{arm2017rpackage}.

In both cases, the logarithm of GDP per capita, post-Cold War period, previous riots, and Polity IV squared remain significant. In contrast, ethnic fractionalisation and former conflict as measured by the UCDP data set are again close to the 0.9 significance level. 

In regard to random forests, grid searches are themselves a data-driven selection of many possible machine learning models, thus it is not strictly necessary to run a batch of additional tests. Nevertheless, we performed a series of grid searches using three different seeds obtained from \texttt{Random.Org} to estimate how the choice of different starting numbers would influence the model outcomes. The output of those models are largely comparable.

As a last set of robustness models, we estimate the same regressions using Harff's (\citeyear{harff2003no}) indicator of genocide and politicide. No variable appear as significant in our EBA models for genocide or politicide onset in peacetime. When we limit our sample to civil war years, the Post-Cold War period is again negatively correlated with the outcome when using the Correlates of War data set. Additionally, excluded population has a negative sign in more than 90\% of the models using both Correlates of War’s and Cederman et al's (\citeyear{cederman2010ethnic}) indicators of conflict. Displaced population also has a negative effect in the Correlates of War data set. During ethnic conflicts, our dummy variable for political assassinations has a negative impact on the onset of genocides. Overall, we conclude then that the significant covariate of genocide and politicide onset differ significantly from those of more general forms of government mass violence. Finally, the machine learning models are comparable to the ones we present above, with a similar set of variables appearing in the random forest estimations. The results are available in appendix \ref{sec:mk-appendix}.

\section{Conclusion}
\label{sec:conclusion4}

In this paper, we apply extreme bounds analysis and distributed random forests to estimate the robustness and predictive ability of 40 variables that have been pointed out as potential determinants of mass killings. We find that from all variables we employ, GDP per capita is the covariate that appears statistically significant more often in our models. In the EBA estimations, we see that the level of democracy, and the post-Cold War period are negatively associated with our dependent variable, whereas ethnic diversity, civil wars, and previous political turmoils increase the likelihood of those events. When we run a machine learning algorithm to estimate out-of-sample predictive performance, GDP per capita and years since last mass killings again appear as important predictors, and variables related to military capabilities and the composition of the population are included in most models. In a nutshell, our tests provide strong evidence in favour of two well-established findings in the specialised literature: Mass killings are unlikely to happen in rich, stable countries and there is a positive association between several types of political conflict, such as previous riots, civil wars and government atrocities.

Nevertheless, there is considerable heterogeneity in some of our findings. Mass killings are rare outcomes, so it is possible that our analysis does not capture all variables that explain the onset of large-scale violence, or that the effect of predictors vary across time and space \citep[8]{bell2015examining}. Additionally, the findings point out that mass killings may have different causes according to the context in which they erupt, so a general theory of state atrocities may obscure important details in our understanding of the root causes of state killings. 

Yet we see this diversity of outcomes under a positive light. The findings presented in this text suggest new avenues for research, and we believe they also highlight the importance of scholars moving from purely correlational, cross-country regressions to other methods that can yield more robust predictions and causal explanations. For instance, why are mass killings in ethnic conflicts correlated with a different set of variables than in armed conflicts in general? Would the results remain robust had scholars decided to code ethnic conflicts in another way? More theoretical advancement would also be welcome. Given that GDP per capita is negatively correlated to state atrocities in virtually every model, it would be interesting to unpack the causal mechanisms by which it operates by testing more specific mechanisms. Disaggregated data can also be of great help to further current research on genocides, politicide, and mass killings.

In terms of practical implications, the results indicate that regime change and pro-growth economic policies are the most efficient ways to prevent mass killings. The international community can therefore play a role in deterring leaders from using force against their own population, either by offering support for domestic opposition groups, intervening, or by fostering economic development. Although costly in the short run, these measures would substantially decrease the likelihood of state violence by breaking the ``conflict trap'' in which past conflicts create the condition for new ones \citep{collier2003breaking}. 

\newpage

\section{Appendix} 
\label{sec:mk-appendix}

This appendix contains all required information to replicate the numerical analyses presented in sections \ref{sec:results4} and \ref{sec:additional-tests4}. \textt{R} code can be found in subsection \ref{sec:mk-code} and the data are available on the following GitHub repository: \href{https://github.com/danilofreire/mass-killings}{https://github.com/danilofreire/mass-killings}. We used \texttt{R} version 3.4.4 (15-03-2018) and Ubuntu 16.04.4 LTS to perform all statistical calculations.

\subsection{Variable Selection}
\label{sec:mk-vs}

We employ some criteria to select our explanatory variables. First, we included only published articles in our sample. Although working papers and policy may also provide important insights about the onset of mass killings, we believe that peer-reviewed research is probably better suited for our purposes. Also, we included only papers that use regression methods on a global sample and were published from 1995 to 2015. Our final sample comprises 45 articles: \citet{anderton2015new}, \citet{balcells2010rivalry, balcells2011continuation}, \citet{besanccon2005relative}, \citet{bulutgil2015social}, \citet{bundervoet2009livestock}, \citet{clayton2016civilianizing}, \citet{colaresi2008kill}, \citet{downes2006desperate, downes2007restraint},  \citet{easterly2006development}, \citet{eck2007one}, \citet{esteban2015strategic}, \citet{fazal2015particular}, \citet{fjelde2014weakening}, \citet{goldsmith2013forecasting}, \citet{harff2003no}, \citet{joshi2017kills}, \citet{kim2010makes}, \citet{kim2016revolutionary}, \citet{kisangani2007political}, \citet{koren2017means}, \citet{krain1997state}, \citet{manekin2013violence}, \citet{mcdoom2013killed,mcdoom2014predicting}, \citet{melander2009new}, \citet{montalvo2008discrete}, \citet{pilster2016differentiation}, \citet{querido2009state}, \citet{raleigh2012violence}, \citet{rost2013will}, \citet{rummel1995democracy}, \citet{schneider2013accounting}, \citet{siroky2015empire}, \citet{stanton2015regulating}, \citet{sullivan2012blood}, \citet{tir2008domestic}, \citet{ulfelder2008assessing}, \citet{ulfelder2012forecasting}, \citet{uzonyi2015civil, uzonyi2016domestic} \citet{valentino2004draining}, \citet{valentino2006covenants}, \citet{verpoorten2012leave}, \citet{wayman2010explaining}, \citet{wig2016local}, and \citet{yanagizawa2014propaganda}.

We find that in those 45 studies scholars made use of nearly 180 measurements to capture roughly 30 key concepts related to threat and costs of mass killings. To be added to our models, a variable should appear in at least two articles. The covariates are summarised in table \ref{tab:mk-vs}. A complete list of variables is available at \href{https://github.com/danilofreire/mass-killings}{https://github.com/danilofreire/mass-killings}.


\begin{table}[!htbp] \centering 
  \caption{Independent Variables} 
  \label{tab:mk-vs} 
\footnotesize
\begin{tabular}{@{\extracolsep{5pt}}lcc} 
\\[-1.8ex]\hline 
\hline \\[-1.8ex] {Variable} & \multicolumn{1}{c}{Coded} & \multicolumn{1}{c}{Source}\\ 
\hline \\[-1.8ex] 
Assassination & Dichotomous & \citet{banks1999cross} \\ 
CINC & Continuous & \citet{cow2017cinc}\\ 
Coup d'état & Dichotomous & \citet{marshall2017pitf}  \\ 
COW civil war onset & Dichotomous & \citet{cow2017cinc,singer1988reconstructing} \\ 
COW civil war ongoing & Dichotomous & \citet{cow2017cinc,singer1988reconstructing} \\ 
Democracy (Polity IV $\geq 6$) & Dichotomous  & Authors' own calculations \\ 
Discriminated dummy & Dichotomous & \citet{cederman2010ethnic}\\ 
Discriminated population & Continuous & \citet{cederman2010ethnic} \\ 
Ethnic diversity (ELF) & Continuous & \citet{fearon2003ethnicity} \\ 
Ethnic war start & Dichotomous & \citet{cederman2010ethnic} \\ 
Ethnic war ongoing & Dichotomous & \citet{cederman2010ethnic} \\ 
Excluded population & Continuous & \citet{cederman2010ethnic} \\ 
Interstate war & Dichotomous & \citet{singer1988reconstructing,cow2017cinc} \\ 
Guerrilla & Dichotomous & \citet{balcells2014does}\\ 
Military expenditure & Continuous & \citet{cow2017cinc} \\ 
Military personnel & Continuous & \citet{cow2017cinc} \\ 
Militias & Dichotomous & \citet{carey2013states} \\ 
Mountainous Terrain & Continuous & \citet{fearon2003ethnicity} \\ 
Physical integrity & Continuous & \citet{cingranelli2010cingranelli}\\ 
Polarisation (all groups/main group) & Continuous &  Authors' own calculations \\ 
Polarisation (all groups/population) & Continuous &  Authors' own calculations  \\ 
Polarisation (included groups/population) & Continuous &  Authors' own calculations  \\ 
Polarisation (included groups/main group) & Continuous &  Authors' own calculations  \\ 
Polity IV & Continuous & \citet{marshall2017pitf}\\ 
Polity IV squared & Continuous & Authors' own calculations \\ 
Population & Continuous & \citet{gleditsch2002expanded} \\
Post-Cold War & Dichotomous & Authors' own calculations \\ 
Real GDP & Continuous & \citet{gleditsch2002expanded} \\ 
Real GDP per capita & Continuous & \citet{gleditsch2002expanded} \\ 
Real GDP per capita (log) & Continuous & Authors' own calculations  \\ 
Regime transition & Continuous & Authors' own calculations \\ 
Riot & Dichotomous & \citet{banks1999cross}\\ 
Total battle deaths & Continuous & \citet{lacina2005monitoring} \\  
Total trade & Continuous & \citet{cow2017cinc} \\ 
Trade dependence (total trade/real GDP) & Continuous & Authors' own calculations \\ 
UCDP civil war onset & Dichotomous & \citet{allansson2017organized,gleditsch2002armed} \\ 
UCDP civil war ongoing & Dichotomous & \citet{allansson2017organized,gleditsch2002armed} \\ 
Urban population (percentage) & Continuous & \citet{cow2017cinc} \\ 
Years since last mass killing & Continuous & Authors' own calculations \\ 
War with territory aims & Dichotomous & \citet{allansson2017organized,gleditsch2002armed} \\ 
\hline \\[-1.8ex] 
\end{tabular} 
\end{table} 

\newpage

\subsection{Descriptive Statistics}
\label{sec:mk-ds}

\begin{table}[!htbp] \centering 
  \caption{Descriptive Statistics} 
  \label{tab:mk-ds} 
\footnotesize 
\begin{tabular}{@{\extracolsep{5pt}}lccccc} 
\\[-1.8ex]\hline 
\hline \\[-1.8ex] 
Statistic & \multicolumn{1}{c}{N} & \multicolumn{1}{c}{Mean} & \multicolumn{1}{c}{St. Dev.} & \multicolumn{1}{c}{Min} & \multicolumn{1}{c}{Max} \\ 
\hline \\[-1.8ex] 
Country code & 9,162 & 452.84 & 247.74 & 2 & 950 \\ 
Year & 9,162 & 1,983.56 & 18.77 & 1,945 & 2,013 \\ 
Genocide/politicide onset & 8,933 & 0.005 & 0.07 & 0 & 1\\ 
Mass killing onset & 9,162 & 0.01 & 0.11 & 0 & 1 \\ 
&&&&&\\
\textit{Independent Variables} & & & & \\
&&&&&\\
Assassination dummy & 8,991 & 0.08 & 0.27 & 0 & 1 \\ 
CINC & 8,767 & 0.01 & 0.02 & 0.00 & 0.38 \\ 
Coup dummy & 8,587 & 0.05 & 0.21 & 0 & 1 \\ 
COW civil war onset & 8,160 & 0.01 & 0.12 & 0 & 1 \\ 
COW civil war ongoing & 8,160 & 0.07 & 0.25 & 0 & 1 \\ 
Democracy dummy & 8,991 & 0.37 & 0.48 & 0 & 1 \\ 
Discriminated dummy & 6,981 & 0.35 & 0.48 & 0 & 1 \\ 
Discriminated population & 6,981 & 0.06 & 0.15 & 0.00 & 0.98 \\ 
Ethnic diversity (ELF) & 6,981 & 0.41 & 0.31 & 0 & 1 \\ 
Ethnic war start & 7,760 & 0.01 & 0.12 & 0 & 1 \\ 
Ethnic war ongoing & 7,760 & 0.11 & 0.31 & 0 & 1 \\ 
Excluded population & 6,981 & 0.16 & 0.22 & 0.00 & 0.98 \\ 
Interstate war & 8,159 & 0.04 & 0.19 & 0 & 1 \\ 
Guerrilla dummy & 714 & 0.81 & 0.40 & 0 & 1 \\ 
Military expenditure & 8,290 & 4,607,120 & 27,785,906 & 0 & 693,600,000 \\ 
Military personnel & 8,620 & 176.70 & 520.90 & 0 & 12,500 \\ 
Militias & 4,097 & 0.22 & 0.42 & 0 & 1 \\ 
Mountainous Terrain & 7,358 & 2.14 & 1.43 & 0.00 & 4.56 \\ 
Physical integrity & 4,499 & 4.73 & 2.31 & 0 & 8 \\ 
Polarisation (all groups/main group) & 6,981 & 0.70 & 0.26 & 0.05 & 1 \\ 
Polarisation (all groups/population) & 6,981 & 0.63 & 0.32 & 0 & 1 \\ 
Polarisation (included groups/population) & 5,610 & 0.64 & 0.32 & 0 & 1 \\ 
Polarisation (included groups/main group) & 6,981 & 0.23 & 0.35 & 0 & 1 \\ 
Polity IV & 8,558 & 0.42 & 7.50 & $-$10 & 10 \\ 
Polity IV squared & 8,558 & 56.35 & 32.59 & 0 & 100 \\ 
Population & 8,293 & 32,993.61 & 112,886.40 & 118.21 & 1,324,353.00 \\
Post-Cold War & 8,991 & 0.40 & 0.49 & 0 & 1 \\ 
Real GDP & 8,293 & 215,317.70 & 804,827.20 & 129.68 & 13,193,478.00 \\ 
Real GDP per capita & 8,293 & 8,104.20 & 18,376.73 & 132.82 & 632,239.50 \\ 
Real GDP per capita (log) & 8,293 & 8.25 & 1.20 & 4.89 & 13.36 \\ 
Regime transition & 1,221 & $-$4.24 & 41.50 & $-$77 & 99 \\ 
Riot dummy & 8,991 & 0.16 & 0.36 & 0 & 1 \\ 
Total battle deaths & 714 & 6,050.86 & 24,404.78 & 100 & 350,000 \\  
Total trade & 8,174 & 53,804.01 & 222,209.90 & 0.80 & 4,825,363.00 \\ 
Trade dependence & 7,670 & 0.26 & 0.69 & 0.0001 & 22.11 \\ 
UCDP civil war onset & 8,733 & 0.02 & 0.14 & 0 & 1 \\ 
UCDP civil war ongoing & 8,733 & 0.15 & 0.36 & 0 & 1 \\ 
Urban population (percentage) & 8,767 & 0.22 & 0.17 & 0.00 & 1.51 \\ 
Years since last mass killing & 9,162 & 23.81 & 17.71 & 0 & 68 \\ 
War with territory aims & 8,924 & 0.07 & 0.26 & 0 & 1 \\ 
\hline \\[-1.8ex] 
\end{tabular} 
\raggedright{\newline \textit{Note}: All independent variables were lagged one year.}
\end{table} 
\normalsize

\newpage

\subsection{Extreme Bounds Analysis Extensions}
\label{sec:mk-ebae}

\subsubsection{Main Model}

We present a series of histograms with the coefficients' distribution of all variables in the main EBA model. There are 36 variables in total, seven of which are robust: Log GDP per capita, post-Cold War period, onset and ongoing civil wars (measured by the UCDP), previous riots, ethnic diversity and the squared term of the Polity IV index.

\vspace{1cm}

\begin{table}[H]
\centering
\begin{tabular}{lrrrrr}
\hline
\textbf{Variable} & \textbf{Avg. $\beta$} & \textbf{Avg. SE} & \textbf{$\%$ Sig.} & \textbf{CDF(0)} & \textbf{Models} \\ \hline
\textit{Base variables} &  &  &  &  &  \\
Log GDP per capita & -0.0091 & 0.0052 & 76.055 & 0.9335 & 226707 \\
 &  &  &  &  &  \\
\textit{Additional variables} &  &  &  &  &  \\
Post-Cold War years & -0.0133 & 0.0085 & 72.845 & 0.9472 & 35614 \\
UCDP civil war onset & 0.0529 & 0.0321 & 52.378 & 0.9441 & 20854 \\
Previous riots & 0.0140 & 0.0100 & 56.242 & 0.9216 & 35614 \\
UCDP ongoing civil war & 0.0172 & 0.0115 & 65.652 & 0.9092 & 20854 \\
Ethnic diversity (ELF) & 0.0184 & 0.0137 & 56.674 & 0.9050 & 35614 \\
Polity IV squared & -0.0002 & 0.0001 & 61.206 & 0.9031 & 35614 \\ \hline
\end{tabular}
\caption{Extreme Bounds Analysis -- Mass killings}
\label{tab:mk}
\end{table}

\clearpage
\begin{sidewaysfigure}
    \centering
    \includegraphics[width=\textwidth]{images/mk.pdf}
    \caption{Extreme Bounds Analysis -- Mass Killings}
    \label{fig:mk}
\end{sidewaysfigure}
\clearpage

\subsubsection{Genocides during Civil Wars}
\label{sec:civil-wars}

Next, we discuss genocides that occur during wartime. We use three covariates that denote ongoing civil conflicts: one by the Uppsala Conflict Data Program \citep{allansson2017organized,gleditsch2002armed}, another by the Correlates of War \citep{sarkees2010resort}, and a third indicating the onset of ethnic conflict as coded by \citet{cederman2010ethnic}. The variables that reach significance in this set of models below are notably different from those obtained in the main estimation. This result provides evidence that mass violence during wartime time follows a separate logic from state killings in peacetime.

\vspace{1cm}

\begin{table}[H]
\centering
\begin{tabular}{lrrrrr}
\hline
\textbf{Variable} & \textbf{Avg. $\beta$} & \textbf{Avg. SE} & \textbf{$\%$ Sig.} & \textbf{CDF(0)} & \textbf{Models} \\ \hline
\textit{UCDP data} &  &  &  &  &  \\
Territory aims & -0.044 & 0.019 & 74.997 & 0.9804 & 17902 \\
Post-Cold War years & -0.038 & 0.019 & 66.574 & 0.9222 & 17902 \\
 &  &  &  &  &  \\
\textit{COW data} &  &  &  &  &  \\
Physical integrity & 0.024 & 0.013 & 66.674 & 0.9564 & 17902 \\
Militias & -0.099 & 0.048 & 73.104 & 0.9490 & 17902 \\
Years since last mass killing & 0.006 & 0.002 & 88.208 & 0.9472 & 101583 \\
Previous riots & 0.078 & 0.041 & 65.412 & 0.9348 & 17902 \\
Ethnic diversity (ELF) & 0.095 & 0.062 & 48.615 & 0.9000 & 17902 \\
 &  &  &  &  &  \\
\textit{Cederman et al. data} &  &  &  &  &  \\
Territory aims & -0.051 & 0.026 & 74.288 & 0.9167 & 17902 \\
Militias & -0.050 & 0.035 & 52.240 & 0.9101 & 17902 \\ \hline
\end{tabular}
\caption{EBA -- Mass Killings during Civil Wars}
\label{tab:ucdp1}
\end{table}

\clearpage
\begin{sidewaysfigure}
    \centering
    \includegraphics[width=\textwidth]{images/mk-ucdp.pdf}
    \caption{EBA -- Mass Killings during Civil Wars (UCDP Data)}
    \label{fig:mk-ucdp}
\end{sidewaysfigure}
\clearpage

\clearpage
\begin{sidewaysfigure}
    \centering
    \includegraphics[width=\textwidth]{images/mk-cow.pdf}
    \caption{EBA -- Mass Killings during Civil Wars (COW Data)}
    \label{fig:mk-cow}
\end{sidewaysfigure}
\clearpage

\clearpage
\begin{sidewaysfigure}
    \centering
    \includegraphics[width=\textwidth]{images/mk-eth.pdf}
    \caption{EBA -- Mass Killings during ethnic civil wars (Cederman et al. Data)}
    \label{fig:mk-eth}
\end{sidewaysfigure}
\clearpage

\subsubsection{Alternative Number of Variables}

The models below are based on 50,000 random draws from the full set of all possible regression models. \citet[819]{salaimartin2004determinants} argue that random sampling produces unbiased estimates of the regression coefficients with low computational time. The models presented in section \ref{sec:results4}, however, include the full set of possible regressions.

The following table shows the results of an EBA with 3 variable combinations per model. The results are very similar to those reported above.

\vspace{1cm}

\begin{table}[H]
\centering
\begin{tabular}{lrrrrr}
\hline
\textbf{Variable} & \textbf{Avg. $\beta$} & \textbf{Avg. SE} & \textbf{$\%$ Sig.} & \textbf{CDF(0)} & \textbf{Models} \\ \hline
\textit{Base variables} &  &  &  &  &  \\
Log GDP per capita & 0.0082 & 0.0043 & 81.439 & 0.9504 & 40677 \\
 &  &  &  &  &  \\
\textit{Additional variables} &  &  &  &  &  \\
Post-Cold War years & -0.0121 & 0.0069 & 77.804 & 0.9609 & 5064 \\
UCDP civil war onset & 0.0523 & 0.0292 & 62.561 & 0.9574 & 3304 \\
Previous riots &0.0134 & 0.0084 & 65.936 & 0.9401 & 5064 \\
UCDP ongoing civil war & 0.0177 & 0.0094 & 72.367 & 0.9372 & 3304 \\
Polity IV squared & -0.0002 & 0.0001 & 66.035 & 0.9268 & 5064 \\ 
Ethnic diversity (ELF) & 0.0162 & 0.0110 & 70.794 & 0.9266 & 5064 \\\hline
\end{tabular}
\caption{EBA -- 3 Variables}
\label{tab:mk-3vars}
\end{table}

\clearpage
\begin{sidewaysfigure}
    \centering
    \includegraphics[width=\textwidth]{images/mk-3vars.pdf}
    \caption{EBA -- 3 Variables}
    \label{fig:mk-3vars}
\end{sidewaysfigure}
\clearpage

Table \ref{tab:mk-5vars} presents the results for models with up 5 variables in each regressions. In contrast with our main EBA model, the indicators of UCDP ongoing civil wars, ethnic diversity, and Polity IV square drop out of significance. Their individual CDFs(0) are about 0.88, just marginally below our specified threshold of 0.9.

\vspace{1cm}

\begin{table}[!htpb]
\centering
\begin{tabular}{lrrrrr}
\hline
\textbf{Variable} & \textbf{Avg. $\beta$} & \textbf{Avg. SE} & \textbf{$\%$ Sig.} & \textbf{CDF(0)} & \textbf{Models} \\ \hline
\textit{Base variables} &  &  &  &  &  \\
Log GDP per capita & -0.010 & 0.006 & 70.806 & 0.9161 & 50000 \\
 &  &  &  &  &  \\
\textit{Additional variables} &  &  &  &  &  \\
Post-Cold War years & -0.014 & 0.010 & 68.496 & 0.9336 & 9532 \\
UCDP civil war onset & 0.053 & 0.035 & 44.784 & 0.9308 & 5100 \\
Previous riots & 0.015 & 0.012 & 47.988 & 0.9047 & 9569 \\\hline
\end{tabular}
\caption{EBA -- 5 Variables}
\label{tab:mk-5vars}
\end{table}

\clearpage
\begin{sidewaysfigure}
    \centering
    \includegraphics[width=\textwidth]{images/mk-5vars.pdf}
    \caption{EBA -- 5 Variables}
    \label{fig:mk-5vars}
\end{sidewaysfigure}
\clearpage

\subsubsection{Alternative Variance Inflation Factors}

In this subsection, we estimate EBA models with different values of Variance Inflation Factor (VIF), which is a measure of multicollinearity. There is no standard definition about what constitutes an acceptable VIF value, although researchers often use 10 as rule of thumb to indicate strong multicollinearity \citep[674]{o2007caution}. Our original model used a slightly more conservative value of 7 as a cutoff. Here, we test the same model with VIF $=$ 10 (less strict), 2.5 (more conservative), and a model without VIF restrictions. The results are essentially identical to those of the main model. In the model with no VIF restriction, however, ethnic fractionalisation fails to meet the threshold by a very small margin. The CDF(0) of that covariate is 0.897, very close to the required value of 0.9. 

\vspace{1cm}

\begin{table}[H]
\centering
\begin{tabular}{lrrrrr}
\hline
\textbf{Variable} & \textbf{Avg. $\beta$} & \textbf{Avg. SE} & \textbf{$\%$ Sig.} & \textbf{CDF(0)} & \textbf{Models} \\ \hline
\textit{Base variables} &  &  &  &  &  \\
Log GDP per capita & -0.0091 & 0.0052 & 76.354 & 0.9343 & 50000 \\
 &  &  &  &  &  \\
\textit{Additional variables} &  &  &  &  &  \\
Post-Cold War years & -0.0134 & 0.0084 & 73.540 & 0.9495 & 7929 \\
UCDP civil war onset & 0.0529 & 0.0322 & 52.141 & 0.9438 & 4553 \\
Previous riots & 0.0140 & 0.0100 & 56.433 & 0.9216 & 7772 \\
UCDP ongoing civil war & 0.0172 & 0.0113 & 66.013 & 0.9113 & 4587 \\
Ethnic diversity (ELF) & 0.0182 & 0.0136 & 56.872 & 0.9056 & 8076 \\
Polity IV squared & -0.0002 & 0.0001 & 60.791 & 0.9021 & 7835 \\ \hline
\end{tabular}
\caption{EBA -- VIF 10}
\label{tab:mk-high-vif}
\end{table}


\clearpage
\begin{sidewaysfigure}
    \centering
    \includegraphics[width=\textwidth]{images/mk-high-vif.pdf}
    \caption{EBA -- VIF 10}
    \label{fig:mk-high-vif}
\end{sidewaysfigure}
\clearpage

\vspace{1cm}

\begin{table}[H]
\centering
\begin{tabular}{lrrrrr}
\hline
\textbf{Variable} & \textbf{Avg. $\beta$} & \textbf{Avg. SE} & \textbf{$\%$ Sig.} & \textbf{CDF(0)} & \textbf{Models} \\ \hline
\textit{Base variables} &  &  &  &  &  \\
Log GDP per capita & -0.0090 & 0.0051 & 76.055 & 0.9343 & 49620 \\
 &  &  &  &  &  \\
\textit{Additional variables} &  &  &  &  &  \\
Post-Cold War years & -0.0132 & 0.0084 & 72.845 & 0.9490 & 7929 \\
UCDP civil war onset & 0.0529 & 0.0322 & 52.378 & 0.9438 & 4553 \\
Previous riots & 0.0141 & 0.0101 & 56.242 & 0.9199 & 7772 \\
UCDP ongoing civil war & 0.0174 & 0.0114 & 65.652 & 0.9103 & 4587 \\
Ethnic diversity (ELF) & 0.0184 & 0.0137 & 56.674 & 0.9054 & 8076 \\
Polity IV squared & -0.0002 & 0.0001 & 61.206 & 0.90267 & 7835 \\ \hline
\end{tabular}
\caption{EBA -- VIF 2.5}
\label{tab:low-vif}
\end{table}

\clearpage
\begin{sidewaysfigure}
    \centering
    \includegraphics[width=\textwidth]{images/mk-low-vif.pdf}
    \caption{EBA -- VIF 2.5}
    \label{fig:mk-low-vif}
\end{sidewaysfigure}
\clearpage

\begin{table}[H]
\centering
\begin{tabular}{lrrrrr}
\hline
\textbf{Variable} & \textbf{Avg. $\beta$} & \textbf{Avg. SE} & \textbf{$\%$ Sig.} & \textbf{CDF(0)} & \textbf{Models} \\ \hline
\textit{Base variables} &  &  &  &  &  \\
Log GDP per capita & -0.0091 & 0.0052 & 75.940 & 0.9343 & 50000 \\
 &  &  &  &  &  \\
\textit{Additional variables} &  &  &  &  &  \\
Post-Cold War years & -0.0133 & 0.0085 & 72.756 & 0.9469 & 7800 \\
UCDP civil war onset & 0.0531 & 0.0321 & 53.068 & 0.9452 & 4596 \\
Previous riots & 0.0140 & 0.0101 & 56.139 & 0.9200 & 7811 \\
UCDP ongoing civil war & 0.0170 & 0.0116 & 64.487 & 0.9057 & 4497 \\
Ethnic diversity (ELF) & 0.0184 & 0.0137 & 56.814 & 0.9056 & 7808 \\
Polity IV squared & -0.0002 & 0.0001 & 60.825 & 0.9009 & 7903 \\ \hline
\end{tabular}
\caption{EBA -- No VIF Restriction}
\label{tab:mk-no-vif}
\end{table}

\clearpage
\begin{sidewaysfigure}
    \centering
    \includegraphics[width=\textwidth]{images/mk-no-vif.pdf}
    \caption{EBA -- No VIF restriction}
    \label{fig:mk-no-vif}
\end{sidewaysfigure}
\clearpage

\subsubsection{Generalised Linear Models}

We reestimate the main EBA model with logit and probit models. Nevertheless, logistic and probit regressions may have issues of complete separation, that is, some covariates may perfectly separate zeros and ones in the outcome variable. In that case, the estimations fail to converge. We address this problem by adding a weak prior to the regression coefficients as suggested by \citet{gelman2008weakly}.\footnote{We thank Mark Bell for sharing \texttt{R} code to estimate penalised-likelihood models.} First, we scaled the non-binary variables to have a mean of 0 and a standard deviation of 0.5, then added a Cauchy distribution with centre 0 and scale 2.5. The probit regressions use a scale of $2.5 \times 1.6$, which is also recommended by the authors \citep{arm2017rpackage}. Ethnic diversity and ongoing civil wars come close to meeting our threshold values (0.88 and 0.84, respectively), and civil war onset (UCDP) has a higher percentage of significant coefficients and a high CDF(0) area than in the linear probability models.

\vspace{1cm}

\begin{table}[H]
\centering
\begin{tabular}{lrrrrr}
\hline
\textbf{Variable} & \textbf{Avg. $\beta$} & \textbf{Avg. SE} & \textbf{$\%$ Sig.} & \textbf{CDF(0)} & \textbf{Models} \\ \hline
\textit{Base variables} &  &  &  &  &  \\
Log GDP per capita & 0.434 & 0.223 & 75.570 & 0.9267 & 50000 \\
 &  &  &  &  &  \\
\textit{Additional variables} &  &  &  &  &  \\
UCDP civil war onset & 1.308 & 0.530 & 87.261 & 0.9742 & 4506 \\
Post-Cold War years & -0.911 & 0.428 & 70.456 & 0.9448 & 7890 \\
Previous riots & 0.744 & 0.38 & 66.778 & 0.9383 & 7805 \\
Polity IV squared & -0.015 & 0.008 & 68.038 & 0.9285 & 7975 \\ \hline
\end{tabular}
\caption{EBA -- Logistic Regression}
\label{tab:mk-logit}
\end{table}

\clearpage
\begin{sidewaysfigure}
    \centering
    \includegraphics[width=\textwidth]{images/mk-logit.pdf}
    \caption{EBA -- Logistic Regression}
    \label{fig:mk-logit}
\end{sidewaysfigure}
\clearpage

\begin{table}[H]
\centering
\begin{tabular}{lrrrrr}
\hline
\textbf{Variable} & \textbf{Avg. $\beta$} & \textbf{Avg. SE} & \textbf{$\%$ Sig.} & \textbf{CDF(0)} & \textbf{Models} \\ \hline
\textit{Base variables} &  &  &  &  &  \\
Log GDP per capita & -0.1924 & 0.1031 & 76.118 & 0.9258 & 50000 \\
 &  &  &  &  &  \\
\textit{Additional variables} &  &  &  &  &  \\
UCDP civil war onset & 0.6422 & 0.2582 & 89.225 & 0.9772 & 4501 \\
Previous riots & 0.3367 & 0.1743 & 71.813 & 0.9436 & 7851 \\
Post-Cold War years & -0.3709 & 0.1830 & 71.465 & 0.9404 & 7836 \\
Polity IV squared & -0.0061 & 0.0032 & 70.155 & 0.9315 & 7931 \\ \hline
\end{tabular}
\caption{EBA -- Probit Regression}
\label{tab:eba1}
\end{table}

\clearpage
\begin{sidewaysfigure}
    \centering
    \includegraphics[width=\textwidth]{images/mk-probit.pdf}
    \caption{EBA -- Probit Regression}
    \label{fig:mk-probit}
\end{sidewaysfigure}
\clearpage

\subsection{Random Forest Extensions}
\label{sec:mk-rfe}

\subsubsection{Alternative Random Seeds}

As noted in section \ref{sec:methods4}, we perform a grid search to optimise the hyperparameters of the random forest models. The grid search evaluates a wide range of parameter values at once, therefore it is generally unnecessary to run additional tests to assess the robustness of the results. Nevertheless, as random forests themselves are an approximation to a number of possible parameter combinations, changes in seed numbers may influence the output. Thus, we start the models with different random seed numbers to evaluate how sturdy are our original results.\footnote{The numbers were generated at \href{https://www.random.org/}{https://www.random.org/}.} The findings holds quite well: Although variable importance changes from one model to another, the most significant variables appear repeatedly in the estimations. The marginal plots also show that their effects on the outcome variable remains similar despite eventual nonlinearities. We show the six most significant predictors of mass killings and their respective partial dependence plots.

\begin{figure}[H]
    \centering
    \includegraphics{images/drf-mk2.pdf}
    \caption{Variable Importance -- Seed 44849999}
    \label{fig:my_label}
\end{figure}

\begin{figure}[H]
    \centering
    \includegraphics[width=\textwidth, height=9cm]{images/drfdpp2.pdf}
    \caption{Partial Plots -- Seed 44849999}
    \label{fig:my_label}
\end{figure}

\newpage

Second model -- seed 1502436: 

\begin{figure}[H]
    \centering
    \includegraphics{images/drf-mk3.pdf}
    \caption{Variable Importance -- Seed 1502436}
    \label{fig:my_label}
\end{figure}

\begin{figure}[H]
    \centering
    \includegraphics[width=\textwidth, height=9cm]{images/drfdpp3a.pdf}
    \caption{Partial Plots -- Seed 1502436}
    \label{fig:my_label}
\end{figure}

\newpage

\subsection{Genocides/Politicides}
\label{sec:mk-other-variable}

In this section, we evaluate the models presented above with a measure of genocide and politicide by \citet{harff2003no}. The results show important contrasts with the previous analyses. First, no variable appear as significant in the main extreme bounds analysis. That is, none of the 36 predictors reached the threshold of CDF(0) $> 0.9$. The variable that came closest to significance was a dummy indicator of coups d'état, which has a CDF(0) of 0.897 and, as expected, is positively correlated with the onset of genocides. The distibution of the covariates' coefficients are available in figure \ref{fig:uamk}.

\clearpage
\begin{sidewaysfigure}
    \centering
    \includegraphics[width=\textwidth]{images/uamk.pdf}
    \caption{EBA -- Genocides/Politicides}
    \label{fig:uamk}
\end{sidewaysfigure}
\clearpage

\subsubsection{Genocides/Politicides during Civil Wars}

Next, we evaluate what covariates are robust when considering only genocides and politicide that occur during civil conflicts. Post-Cold War years again appear as a significant variable and with a negative sign; excluded population also has a negative impact on the outcome variable in two analyses.

\vspace{1cm}

\begin{table}[H]
\centering
\begin{tabular}{lrrrrr}
\hline
\textbf{Variable} & \textbf{Avg. $\beta$} & \textbf{Avg. SE} & \textbf{$\%$ Sig.} & \textbf{CDF(0)} & \textbf{Models} \\ \hline
\textit{UCDP data} &  &  &  &  &  \\
Excluded population & -0.037 & 0.022 & 64.524 & 0.9176 & 8758 \\
 &  &  &  &  &  \\
\textit{COW data} &  &  &  &  &  \\
Excluded population & -0.057 & 0.031 & 65.703 & 0.9570 & 8820 \\
Discriminated population & -0.050 & 0.029 & 53.850 & 0.93.67 & 8767 \\
Post-Cold War years & -0.019 & 0.013 & 42.531 & 0.9203 & 8904 \\
 &  &  &  &  &  \\
\textit{Cederman et al. data} &  &  &  &  &  \\
Assassination dummy & -0.009 & 0.006 & 47.723 & 0.9232 & 8828 \\ \hline
\end{tabular}
\caption{EBA -- Genocides/Politicides}
\label{tab:uamk1}
\end{table}

\newpage
\clearpage
\begin{sidewaysfigure}
    \centering
    \includegraphics[width=\textwidth]{images/uamk-ucdp.pdf}
    \caption{EBA -- Genocides and Politicides during Civil Wars (UCDP Data)}
    \label{fig:uamk-ucdp}
\end{sidewaysfigure}
\clearpage

\clearpage
\begin{sidewaysfigure}
    \centering
    \includegraphics[width=\textwidth]{images/uamk-cow.pdf}
    \caption{EBA -- Genocides and Politicides during Civil Wars (COW Data)}
    \label{fig:uamk-cow}
\end{sidewaysfigure}
\clearpage

\clearpage
\begin{sidewaysfigure}
    \centering
    \includegraphics[width=\textwidth]{images/uamk-eth.pdf}
    \caption{EBA -- Genocides and Politicides during Ethnic Civil Wars (Cederman et al. Data)}
    \label{fig:uamk-eth}
\end{sidewaysfigure}
\clearpage

\newpage
\subsection{Genocides/Politicides -- Random Forests}

Lastly, we present three models using the distributed random forest algorithm \citep{h2o2017}. The results are in line with those obtained with the mass killing variable by  \citet{ulfelder2008assessing}. Again, we see that CINC, the percentage of urban population, and variables concerning the military are some of the most important predictors of state-led violence. The results confirm the overall finding of the chapter: Poor countries are more likely to experience mass killings, and states with a stronger army see a significant upward shift in genocide risk.

\begin{figure}[H]
    \centering
    \includegraphics{images/drf-gp.pdf}
    \caption{Variable Importance -- Genocides/Politicides}
    \label{fig:my_label}
\end{figure}

\begin{figure}[H]
    \centering
    \includegraphics[width=\textwidth, height=9cm]{images/drfdpp4a.pdf}
    \caption{Partial Plots -- Genocides/Politicides}
    \label{fig:my_label}
\end{figure}

\newpage

The next graphs show the most important predictors of genocides that occur during civil wars and their respective partial dependence plots. 

\begin{figure}[H]
    \centering
    \includegraphics{images/drf-gp1.pdf}
    \caption{Variable Importance -- Genocides/Politicides (UCDP Data)}
    \label{fig:my_label}
\end{figure}

\begin{figure}[H]
    \centering
    \includegraphics[width=\textwidth, height=9cm]{images/drfdpp5a.pdf}
    \caption{Partial Plots -- Genocides/Politicides (UCDP Data)}
    \label{fig:my_label}
\end{figure}

\begin{figure}[H]
    \centering
    \includegraphics{images/drf-gp2.pdf}
    \caption{Variable Importance -- Genocides/Politicides (COW Data)}
    \label{fig:my_label}
\end{figure}

\begin{figure}[H]
    \centering
    \includegraphics[width=\textwidth, height=9cm]{images/drfdpp6a.pdf}
    \caption{Partial Plots -- Genocides/Politicides (COW Data)}
    \label{fig:my_label}
\end{figure}

\begin{figure}[H]
    \centering
    \includegraphics{images/drf-gp3.pdf}
    \caption{Variable Importance -- Genocides/Politicides (Cederman et al. Data)}
    \label{fig:my_label}
\end{figure}

\begin{figure}[H]
    \centering
    \includegraphics[width=\textwidth, height=9cm]{images/drfdpp7a.pdf}
    \caption{Partial Plots -- Genocides/Politicides (Cederman et al. Data)}
    \label{fig:my_label}
\end{figure}

\newpage

\subsection{\texttt{R} Code}
\label{sec:mk-code}

The \texttt{R} code below reproduces the analyses presented in chapter \ref{chap:killings}.

\singlespacing
\small
\begin{verbatim}
######################
### Data Wrangling ###
######################

## Install and load required packages 
if (!require("tidyverse")) {
        install.packages("tidyverse")
}
if (!require("data.table")) {
        install.packages("data.table")
}
if (!require("ExtremeBounds")) {
        install.packages("ExtremeBounds")
}
if (!require("h2o")) {
        install.packages("h2o")
}
if (!require("sandwich")) {
        install.packages("sandwich")
}
if (!require("arm")) {
        install.packages("arm")
}
if (!require("stargazer")) {
        install.packages("stargazer")
}

## Load dataset
df <- haven::read_dta("data/base variables.dta") %>% setDT()

## Select and lag variables
sd.cols <- c("UCDPcivilwarstart", "UCDPcivilwarongoing", "COWcivilwarstart",
             "COWcivilwarongoing", "ethnowarstart", "ethnowarongoing",
             "assdummy", "demdummy", "elf", "lmtnest", "pop", "realgdp",
             "rgdppc", "polity2", "exclpop", "discpop", "polrqnew",
             "poltrqnew", "egiptpolrqnew", "egippolrqnew", "discrim",
             "elf2", "interstatewar", "milex", "milper", "percentpopurban",
             "postcoldwar", "coupdummy", "riotdummy", "territoryaims",
             "totaltrade", "tradedependence", "militias", "physint", "cinc",
             "totalbeaths", "change", "guerrilladummy", "sf", "regtrans")

df1 <- cbind(df, df[, shift(.SD, 1, give.names = TRUE),
                    by = ccode, .SDcols = sd.cols]) 

# Remove the second `ccode` variable
df1 <- as.data.frame(df1[, -c(70)])

# Add new variables
df1$logrgdppc_lag_1 <- log(df1$rgdppc_lag_1)
df1$polity2sq_lag_1 <- df1$polity2_lag_1^2

# UCDP civil war == 1
df.ucdp <- df1 %>% filter(UCDPcivilwarongoing == 1)
df.ucdp <- as.data.frame(df.ucdp[, c(1:7, 76:111)])
names(df.ucdp) <- sub("_.*","", names(df.ucdp)) 

# COW civil war == 1
df.cow <- df1 %>% filter(COWcivilwarongoing == 1)
df.cow <- as.data.frame(df.cow[, c(1:7, 76:111)])
names(df.cow) <- sub("_.*","", names(df.cow)) 

# Ethnic civil war == 1
df.eth <- df1 %>% filter(ethnowarongoing == 1)
df.eth <- as.data.frame(df.eth[, c(1:7, 76:111)])
names(df.eth) <- sub("_.*","", names(df.eth)) 

# Regular model
df2 <- as.data.frame(df1[, c(1:7, 70:111)])
names(df2) <- sub("_.*","", names(df2)) 


###############################
### Extreme Bounds Analyses ###
###############################

## Classifying a few variables as mutually exclusive.
## "Change" was removed because it was correlated at 0.99 with "regtrans". 
free.variables <- c("logrgdppc", "polity2", "mksyr")
civilwar.variables <- c("UCDPcivilwarongoing", "UCDPcivilwarstart",
                        "COWcivilwarongoing", "COWcivilwarstart",
                        "ethnowarongoing", "ethnowarstart")
doubtful.variables <- c("UCDPcivilwarongoing", "UCDPcivilwarstart",
                        "COWcivilwarongoing", "COWcivilwarstart",
                        "ethnowarongoing", "ethnowarstart", "assdummy",
                        "totaltrade", "tradedependence", "milper", "milex",
                        "pop", "totalbeaths", "guerrilladummy", "regtrans",
                        "riotdummy", "territoryaims", "militias",
                        "physint", "percentpopurban", "coupdummy",
                        "postcoldwar",  "lmtnest", "realgdp", "discrim",
                        "exclpop", "discpop", "elf",  "polrqnew",
                        "egippolrqnew", "poltrqnew", "egiptpolrqnew",
                        "polity2sq")

# Cluster-robust standard errors
se.clustered.robust <- function(model.object){
        model.fit <- vcovHC(model.object, type = "HC", cluster = "country")
        out <- sqrt(diag(model.fit))
        return(out)
}

# Main model
m1 <- eba(y = "MKstart", free = free.variables,
          exclusive = list(civilwar.variables),
          doubtful = doubtful.variables, k = 0:4,
          data = df2, vif = 7, level = 0.9,
          se.fun = se.clustered.robust)

summary(m1)
hist(m1, variables = c("logrgdppc", "polity2", "polity2sq", "mksyr",
                       "UCDPcivilwarongoing",
                       "UCDPcivilwarstart", "COWcivilwarongoing",
                       "COWcivilwarstart", "ethnowarongoing", "ethnowarstart",
                       "assdummy", "totaltrade", "tradedependence", "milper",
                       "milex","pop", "totalbeaths", "guerrilladummy", "regtrans",
                       "riotdummy", "territoryaims", "militias", "physint",
                       "percentpopurban", "coupdummy", "postcoldwar",
                       "lmtnest", "realgdp", "discrim", "exclpop", "discpop",
                       "elf", "polrqnew", "egippolrqnew", "poltrqnew",
                       "egiptpolrqnew"),
     main = c("Log GDP capita", "Polity IV", "Polity IV^2", "Years last mass killing",
              "UCDP ongoing", "UCDP onset", "COW ongoing", "COW onset", 
              "Ethnic ongoing", "Ethnic onset", "Assassination", "Total trade", 
              "Trade dependence", "Military personnel", "Military expenditure", "Population", 
              "Total deaths", "Guerrilla", "Regime transition", "Riots",
              "Territory Aims", "Militias", "Physical integrity", "% Urban",
              "Coups", "Post-Cold War", "Mountainous terrain", "Real GDP",
              "Discrimination", "Excl pop", "Discrim pop", "ELF", "Groups/Eth relevant", 
              "Group/Tot pop", "Inc groups/Eth relevant", "Inc groups/Tot pop"),
     density.col = "black", mu.col = "red3")
          
# Mass killings during civil war
# UCDP civil conflicts == 1
doubtful.variables <- c("assdummy", "totaltrade", "tradedependence",
                        "milper", "milex", "pop", "totalbeaths",
                        "guerrilladummy", "regtrans", "riotdummy",
                        "territoryaims", "militias", "physint",
                        "percentpopurban", "coupdummy", "postcoldwar",
                        "lmtnest", "realgdp", "discrim", "exclpop",
                        "discpop", "elf",  "polrqnew", "egippolrqnew",
                        "poltrqnew", "egiptpolrqnew", "polity2sq")

m1 <- eba(y = "MKstart", free = free.variables,
          doubtful = doubtful.variables, k = 0:4,
          data = df.ucdp, vif = 7,
          level = 0.9, se.fun = se.clustered.robust)
          
summary(m1)
hist(m1, variables = c("logrgdppc", "polity2", "polity2sq", "mksyr",
                       "assdummy", "totaltrade", "tradedependence", "milper",
                       "milex","pop", "totalbeaths", "guerrilladummy", "regtrans",
                       "riotdummy", "territoryaims", "militias", "physint",
                       "percentpopurban", "coupdummy", "postcoldwar",
                       "lmtnest", "realgdp", "discrim", "exclpop", "discpop",
                       "elf", "polrqnew", "egippolrqnew", "poltrqnew",
                       "egiptpolrqnew"),
     main = c("Log GDP capita", "Polity IV", "Polity IV^2", "Years last mass killing",
              "Assassination", "Total trade", 
              "Trade dependence", "Military personnel", "Military expenditure", "Population", 
              "Total deaths", "Guerrilla", "Regime transition", "Riots",
              "Territory Aims", "Militias", "Physical integrity", "% Urban",
              "Coups", "Post-Cold War", "Mountainous terrain", "Real GDP",
              "Discrimination", "Excl pop", "Discrim pop", "ELF", "Groups/Eth relevant", 
              "Group/Tot pop", "Inc groups/Eth relevant", "Inc groups/Tot pop"),
     density.col = "black", mu.col = "red3")
     
# COW civil wars == 1
doubtful.variables <- c("assdummy", "totaltrade", "tradedependence",
                        "milper", "milex", "pop", "totalbeaths",
                        "guerrilladummy", "regtrans", "riotdummy",
                        "territoryaims", "militias", "physint",
                        "percentpopurban", "coupdummy", "postcoldwar",
                        "lmtnest", "realgdp", "discrim", "exclpop",
                        "discpop", "elf",  "polrqnew", "egippolrqnew",
                        "poltrqnew", "egiptpolrqnew", "polity2sq")

m1 <- eba(y = "MKstart", free = free.variables,
          doubtful = doubtful.variables, k = 0:4,
          data = df.cow, vif = 7,
          level = 0.9, se.fun = se.clustered.robust)
          
summary(m1)
hist(m1, variables = c("logrgdppc", "polity2", "polity2sq", "mksyr",
                       "assdummy", "totaltrade", "tradedependence", "milper",
                       "milex","pop", "totalbeaths", "guerrilladummy", "regtrans",
                       "riotdummy", "territoryaims", "militias", "physint",
                       "percentpopurban", "coupdummy", "postcoldwar",
                       "lmtnest", "realgdp", "discrim", "exclpop", "discpop",
                       "elf", "polrqnew", "egippolrqnew", "poltrqnew",
                       "egiptpolrqnew"),
     main = c("Log GDP capita", "Polity IV", "Polity IV^2", "Years last mass killing",
              "Assassination", "Total trade", 
              "Trade dependence", "Military personnel", "Military expenditure", "Population", 
              "Total deaths", "Guerrilla", "Regime transition", "Riots",
              "Territory Aims", "Militias", "Physical integrity", "% Urban",
              "Coups", "Post-Cold War", "Mountainous terrain", "Real GDP",
              "Discrimination", "Excl pop", "Discrim pop", "ELF", "Groups/Eth relevant", 
              "Group/Tot pop", "Inc groups/Eth relevant", "Inc groups/Tot pop"),
     density.col = "black", mu.col = "red3")
     
# Ethnic civil war == 1
doubtful.variables <- c("assdummy", "totaltrade", "tradedependence",
                        "milper", "milex", "pop", "totalbeaths",
                        "guerrilladummy", "regtrans", "riotdummy",
                        "territoryaims", "militias", "physint",
                        "percentpopurban", "coupdummy", "postcoldwar",
                        "lmtnest", "realgdp", "discrim", "exclpop", 
                        "discpop", "elf",  "polrqnew", "egippolrqnew",
                        "poltrqnew", "egiptpolrqnew", "polity2sq")

m1 <- eba(y = "MKstart", free = free.variables,
          doubtful = doubtful.variables, k = 0:4,
          data = df.eth, vif = 7,
          level = 0.9, se.fun = se.clustered.robust)
          
summary(m1)
hist(m1, variables = c("logrgdppc", "polity2", "polity2sq", "mksyr",
                       "assdummy", "totaltrade", "tradedependence", "milper",
                       "milex","pop", "totalbeaths", "guerrilladummy", "regtrans",
                       "riotdummy", "territoryaims", "militias", "physint",
                       "percentpopurban", "coupdummy", "postcoldwar",
                       "lmtnest", "realgdp", "discrim", "exclpop", "discpop",
                       "elf", "polrqnew", "egippolrqnew", "poltrqnew",
                       "egiptpolrqnew"),
     main = c("Log GDP capita", "Polity IV", "Polity IV^2", "Years last mass killing",
              "Assassination", "Total trade", 
              "Trade dependence", "Military personnel", "Military expenditure", "Population", 
              "Total deaths", "Guerrilla", "Regime transition", "Riots",
              "Territory Aims", "Militias", "Physical integrity", "% Urban",
              "Coups", "Post-Cold War", "Mountainous terrain", "Real GDP",
              "Discrimination", "Excl pop", "Discrim pop", "ELF", "Groups/Eth relevant", 
              "Group/Tot pop", "Inc groups/Eth relevant", "Inc groups/Tot pop"),
     density.col = "black", mu.col = "red3")

## Different values of k
## Code for the histogram not included as it is the same as that of the main model.

# 3 variables per model
m1 <- eba(y = "MKstart", free = free.variables,
          exclusive = list(civilwar.variables),
          doubtful = doubtful.variables, k = 0:3,
          data = df2, vif = 7, level = 0.9, draws = 50000,
          se.fun = se.clustered.robust)

# 5 variables per model
m1 <- eba(y = "MKstart", free = free.variables,
          exclusive = list(civilwar.variables),
          doubtful = doubtful.variables, k = 0:5,
          data = df2, vif = 7, draws = 50000,
          level = 0.9, se.fun = se.clustered.robust)
          
## Alternative VIFs

# VIF = 10
# Low VIF
m1 <- eba(y = "MKstart", free = free.variables,
          exclusive = list(civilwar.variables),
          doubtful = doubtful.variables, k = 0:4,
          data = df2, vif = 10, level = 0.9, draws = 50000,
          se.fun = se.clustered.robust)

# VIF = 2.5
m1 <- eba(y = "MKstart", free = free.variables,
          exclusive = list(civilwar.variables),
          doubtful = doubtful.variables, k = 0:4,
          data = df2, vif = 2.5, level = 0.9, draws = 50000,
          se.fun = se.clustered.robust)

# No VIF
m1 <- eba(y = "MKstart", free = free.variables,
          exclusive = list(civilwar.variables),
          doubtful = doubtful.variables, k = 0:4,
          data = df2, level = 0.9, draws = 50000,
          se.fun = se.clustered.robust)

## Generalised linear models

# Logit
m1 <- eba(y = "MKstart", free = free.variables,
          exclusive = list(civilwar.variables),
          doubtful = doubtful.variables, k = 0:4,
          data = df2, level = 0.9, vif = 7, draws = 50000,
          reg.fun = bayesglm, family = binomial(link = "logit"))
          
# Probit
m1 <- eba(y = "MKstart", free = free.variables,
          exclusive = list(civilwar.variables),
          doubtful = doubtful.variables, k = 0:4,
          data = df2, level = 0.9, vif = 7, draws = 50000,
          reg.fun = bayesglm, family = binomial(link = "probit"))
          

##################################
### Distributed Random Forests ###
##################################

## Random Forests

## Prepare the data set
h2o.init(nthreads = -1, max_mem_size = "20g") # memory size

df2a <- as.h2o(df2)

df2a$MKstart <- as.factor(df2a$MKstart)  # encode the binary response as a factor
h2o.levels(df2a$MKstart)

# Partition the data into training, validation and test sets
splits <- h2o.splitFrame(data = df2a, 
                         ratios = c(0.7, 0.15),  # 70%, 15%, 15%
                         seed = 42)  # reproducibility

train <- h2o.assign(splits[[1]], "train.hex")   
valid <- h2o.assign(splits[[2]], "valid.hex") 
test <- h2o.assign(splits[[3]], "test.hex")

y <- "MKstart"
x <- setdiff(names(df2), c(y, "ccode", "year", "rgdppc",
                           "mksyr2", "mksyr3", "sf", "country",
                           "elf2", "polity2sq")) 

# Running the model
rf <- h2o.grid("randomForest", x = x, y = y, training_frame = train, 
               validation_frame = valid, nfolds = 5, grid_id = "gridrf01",
               fold_assignment = "Stratified",
               hyper_params = list(ntrees = c(256, 512, 1024),
                                   max_depth = c(10, 20, 40),
                                   mtries = c(5, 6, 7),
                                   balance_classes = c(TRUE, FALSE),
                                   sample_rate = c(0.5, 0.632, 0.95),
                                   col_sample_rate_per_tree = c(0.5, 0.9, 1.0),
                                   histogram_type = c("UniformAdaptive",
                                                      "Random",
                                                      "QuantilesGlobal",
                                                      "RoundRobin")),
               search_criteria = list(strategy = "RandomDiscrete", 
                                      max_models = 1000, 
                                      stopping_metric = "auc", 
                                      stopping_tolerance = 0.01, 
                                      stopping_rounds = 5, 
                                      seed = 26227709)) 

# Saving the most accurate model
rf.grid <- h2o.getGrid(grid_id = "gridrf01",
                       sort_by = "auc",
                       decreasing = TRUE)

rf2 <- h2o.getModel(rf.grid@model_ids[[1]])
h2o.saveModel(rf2, path = "/root/Documents/mk/")
summary(rf2)
varimp <- as.data.frame(h2o.varimp(rf2))
h2o.varimp_plot(rf2)
h2o.performance(rf2, newdata = test)

# Graphs
a <- h2o.loadModel("/home/sussa/Documents/GitHub/mk/gridrf01_model_21")
print(va <- a %>% h2o.varimp() %>% as.data.frame() %>% head(., 6)) 

par(mgp=c(2.2,0.45,0), tcl=-0.4, mar=c(2,7.5,1,1))
barplot(va$scaled_importance[6:1],
        horiz = TRUE, las = 1, cex.names=0.9,
        names.arg = c("Polity IV", 
                      "Military personnel",
                      "Ethnic polarisation", 
                      "% Urban pop.",
                      "Years mass killing",
                      "Log GDP per capita"),
        main = "")

logrgdppc <- h2o.partialPlot(object = a, data = train, cols = c("logrgdppc"))
p1 <- qplot(logrgdppc$logrgdppc, logrgdppc$mean_response) + geom_line() + theme_classic() + ylim(0, 0.05) +
        xlab("Log GDP per capita") + ylab("Mean response")

mksyr <- h2o.partialPlot(object = a, data = train, cols = c("mksyr"))
p2 <- qplot(mksyr$mksyr, mksyr$mean_response) + geom_line() + theme_classic() + ylim(0, 0.05) +
        xlab("Years since mass killing") +  ylab("Mean response")

percentpopurban <- h2o.partialPlot(object = a, data = train, cols = c("percentpopurban"))
p3 <- qplot(percentpopurban$percentpopurban, percentpopurban$mean_response) + geom_line() +
        theme_classic() + ylim(0, 0.05) + xlab("% Urban") + ylab("Mean response")

egiptpolrqnew <- h2o.partialPlot(object = a, data = train, cols = c("egiptpolrqnew"))
p4 <- qplot(egiptpolrqnew$egiptpolrqnew, egiptpolrqnew$mean_response) + geom_line() +
        theme_classic() + ylim(0, 0.05) + xlab("Ethnic polarisation") + ylab("Mean response")

milper <- h2o.partialPlot(object = a, data = train, cols = c("milper"))
p5 <- qplot(milper$milper, milper$mean_response) + geom_line() + theme_classic() + ylim(0, 0.05) +
        xlab("Military personnel") + ylab("Mean response")

polity2 <- h2o.partialPlot(object = a, data = train, cols = c("polity2"))
p6 <- qplot(polity2$polity2, polity2$mean_response) + geom_line() + theme_classic() + ylim(0, 0.05) +
        xlab("Polity IV") + ylab("Mean response")
        
# Multiplot function: http://www.cookbook-r.com/Graphs/Multiple_graphs_on_one_page_(ggplot2)/
multiplot <- function(..., plotlist=NULL, file, cols=1, layout=NULL) {
        library(grid)
        
        # Make a list from the ... arguments and plotlist
        plots <- c(list(...), plotlist)
        
        numPlots = length(plots)
        
        # If layout is NULL, then use 'cols' to determine layout
        if (is.null(layout)) {
                # Make the panel
                # ncol: Number of columns of plots
                # nrow: Number of rows needed, calculated from # of cols
                layout <- matrix(seq(1, cols * ceiling(numPlots/cols)),
                                 ncol = cols, nrow = ceiling(numPlots/cols))
        }
        
        if (numPlots==1) {
                print(plots[[1]])
                
        } else {
                # Set up the page
                grid.newpage()
                pushViewport(viewport(layout = grid.layout(nrow(layout), ncol(layout))))
                
                # Make each plot, in the correct location
                for (i in 1:numPlots) {
                        # Get the i,j matrix positions of the regions that contain this subplot
                        matchidx <- as.data.frame(which(layout == i, arr.ind = TRUE))
                        
                        print(plots[[i]], vp = viewport(layout.pos.row = matchidx$row,
                                                        layout.pos.col = matchidx$col))
                }
        }
}

multiplot(p1, p4, p2, p5, p3, p6, cols = 3)

## Different seeds

rf <- h2o.grid("randomForest", x = x, y = y, training_frame = train, 
               validation_frame = valid, nfolds = 5, grid_id = "gridrf01b",
               fold_assignment = "Stratified",
               hyper_params = list(ntrees = c(256, 512, 1024),
                                   max_depth = c(10, 20, 40),
                                   mtries = c(5, 6, 7),
                                   balance_classes = c(TRUE, FALSE),
                                   sample_rate = c(0.5, 0.632, 0.95),
                                   col_sample_rate_per_tree = c(0.5, 0.9, 1.0),
                                   histogram_type = c("UniformAdaptive",
                                                      "Random",
                                                      "QuantilesGlobal",
                                                      "RoundRobin")),
               search_criteria = list(strategy = "RandomDiscrete", 
                                      max_models = 1000, 
                                      stopping_metric = "auc", 
                                      stopping_tolerance = 0.01, 
                                      stopping_rounds = 5, 
                                      seed = 44849999)) 

# Saving the most accurate model
rf.grid <- h2o.getGrid(grid_id = "gridrf01b",
                       sort_by = "auc",
                       decreasing = TRUE)

rf2 <- h2o.getModel(rf.grid@model_ids[[1]])
h2o.saveModel(rf2, path = "/root/Documents/mk/")
summary(rf2)
varimp <- as.data.frame(h2o.varimp(rf2))
h2o.varimp_plot(rf2)
h2o.performance(rf2, newdata = test)

# Graphs
a <- h2o.loadModel("/home/sussa/Documents/GitHub/mk/gridrf01b_model_8")
print(va <- a %>% h2o.varimp() %>% as.data.frame() %>% head(., 6)) 

par(mgp=c(2.2,0.45,0), tcl=-0.4, mar=c(2,7.5,1,1))
barplot(va$scaled_importance[6:1],
        horiz = TRUE, las = 1, cex.names=0.9,
        names.arg = c("Population", 
                      "Military personnel",
                      "Trade dependence", 
                      "Log GDP per capita",
                      "% Urban pop.",
                      "Years mass killing"),
        main = "")

mksyr <- h2o.partialPlot(object = a, data = train, cols = c("mksyr"))
p1 <- qplot(mksyr$mksyr, mksyr$mean_response) + geom_line() + theme_classic() + ylim(0, 0.05) +
        xlab("Years since mass killing") +  ylab("Mean response")
        
percentpopurban <- h2o.partialPlot(object = a, data = train, cols = c("percentpopurban"))
p2 <- qplot(percentpopurban$percentpopurban, percentpopurban$mean_response) + geom_line() +
        theme_classic() + ylim(0, 0.05) + xlab("% Urban") + ylab("Mean response")

logrgdppc <- h2o.partialPlot(object = a, data = train, cols = c("logrgdppc"))
p3 <- qplot(logrgdppc$logrgdppc, logrgdppc$mean_response) + geom_line() + theme_classic() + ylim(0, 0.05) +
        xlab("Log GDP per capita") + ylab("Mean response")

tradedependence <- h2o.partialPlot(object = a, data = train, cols = c("tradedependence"))
p4 <- qplot(tradedependence$tradedependence, tradedependence$mean_response) + geom_line() +
        theme_classic() + ylim(0, 0.05) + xlab("Trade dependence") + ylab("Mean response")

milper <- h2o.partialPlot(object = a, data = train, cols = c("milper"))
p5 <- qplot(milper$milper, milper$mean_response) + geom_line() + theme_classic() + ylim(0, 0.05) +
        xlab("Military personnel") + ylab("Mean response")

pop <- h2o.partialPlot(object = a, data = train, cols = c("pop"))
p6 <- qplot(pop$pop, pop$mean_response) + geom_line() + theme_classic() + ylim(0, 0.05) +
        xlab("Population") + ylab("Mean response")

multiplot(p1, p4, p2, p5, p3, p6, cols = 3)


rf <- h2o.grid("randomForest", x = x, y = y, training_frame = train, 
               validation_frame = valid, nfolds = 5, grid_id = "gridrf01c",
               fold_assignment = "Stratified",
               hyper_params = list(ntrees = c(256, 512, 1024),
                                   max_depth = c(10, 20, 40),
                                   mtries = c(5, 6, 7),
                                   balance_classes = c(TRUE, FALSE),
                                   sample_rate = c(0.5, 0.632, 0.95),
                                   col_sample_rate_per_tree = c(0.5, 0.9, 1.0),
                                   histogram_type = c("UniformAdaptive",
                                                      "Random",
                                                      "QuantilesGlobal",
                                                      "RoundRobin")),
               search_criteria = list(strategy = "RandomDiscrete", 
                                      max_models = 1000, 
                                      stopping_metric = "auc", 
                                      stopping_tolerance = 0.01, 
                                      stopping_rounds = 5, 
                                      seed = 1502436)) 

# Saving the most accurate model
rf.grid <- h2o.getGrid(grid_id = "gridrf01c",
                       sort_by = "auc",
                       decreasing = TRUE)

rf2 <- h2o.getModel(rf.grid@model_ids[[1]])
h2o.saveModel(rf2, path = "/root/Documents/mk/")
summary(rf2)
varimp <- as.data.frame(h2o.varimp(rf2))
h2o.varimp_plot(rf2)
h2o.performance(rf2, newdata = test)

# Graphs
a <- h2o.loadModel("/home/sussa/Documents/GitHub/mk/gridrf01c_model_58")
print(va <- a %>% h2o.varimp() %>% as.data.frame() %>% head(., 6)) 

par(mgp=c(2.2,0.45,0), tcl=-0.4, mar=c(2,7.5,1,1))
barplot(va$scaled_importance[6:1],
        horiz = TRUE, las = 1, cex.names=0.9,
        names.arg = c("Population", 
                      "Military personnel",
                      "Trade dependence", 
                      "Log GDP per capita",
                      "% Urban pop.",
                      "Years mass killing"),
        main = "")

mksyr <- h2o.partialPlot(object = a, data = train, cols = c("mksyr"))
p1 <- qplot(mksyr$mksyr, mksyr$mean_response) + geom_line() + theme_classic() + ylim(0, 0.05) +
        xlab("Years since mass killing") +  ylab("Mean response")
        
percentpopurban <- h2o.partialPlot(object = a, data = train, cols = c("percentpopurban"))
p2 <- qplot(percentpopurban$percentpopurban, percentpopurban$mean_response) + geom_line() +
        theme_classic() + ylim(0, 0.05) + xlab("% Urban") + ylab("Mean response")

logrgdppc <- h2o.partialPlot(object = a, data = train, cols = c("logrgdppc"))
p3 <- qplot(logrgdppc$logrgdppc, logrgdppc$mean_response) + geom_line() + theme_classic() + ylim(0, 0.05) +
        xlab("Log GDP per capita") + ylab("Mean response")

tradedependence <- h2o.partialPlot(object = a, data = train, cols = c("tradedependence"))
p4 <- qplot(tradedependence$tradedependence, tradedependence$mean_response) + geom_line() +
        theme_classic() + ylim(0, 0.05) + xlab("Trade dependence") + ylab("Mean response")

milper <- h2o.partialPlot(object = a, data = train, cols = c("milper"))
p5 <- qplot(milper$milper, milper$mean_response) + geom_line() + theme_classic() + ylim(0, 0.05) +
        xlab("Military personnel") + ylab("Mean response")

pop <- h2o.partialPlot(object = a, data = train, cols = c("pop"))
p6 <- qplot(pop$pop, pop$mean_response) + geom_line() + theme_classic() + ylim(0, 0.05) +
        xlab("Population) + ylab("Mean response")

multiplot(p1, p4, p2, p5, p3, p6, cols = 3)

## Genocides in civil wars

## UCDP data
df.ucdpa <- as.h2o(df.ucdp)

df.ucdpa$MKstart <- as.factor(df.ucdpa$MKstart)  # encode the binary response as a factor
h2o.levels(df.ucdpa$MKstart)

# Partition the data into training, validation and test sets
splits <- h2o.splitFrame(data = df.ucdpa, 
                         ratios = c(0.7, 0.15),  # 70%, 15%, 15%
                         seed = 42)  # reproducibility

train <- h2o.assign(splits[[1]], "train.hex")   
valid <- h2o.assign(splits[[2]], "valid.hex") 
test <- h2o.assign(splits[[3]], "test.hex")

y <- "MKstart"
x <- setdiff(names(df.ucdp), c(y, "ccode", "year", "rgdppc",
                           "mksyr2", "mksyr3", "sf", "country",
                           "elf2", "polity2sq")) 

# Running the model
rf <- h2o.grid("randomForest", x = x, y = y, training_frame = train, 
               validation_frame = valid, nfolds = 5, grid_id = "gridrf02",
               fold_assignment = "Stratified",
               hyper_params = list(ntrees = c(256, 512, 1024),
                                   max_depth = c(10, 20, 40),
                                   mtries = c(5, 6, 7),
                                   balance_classes = c(TRUE, FALSE),
                                   sample_rate = c(0.5, 0.632, 0.95),
                                   col_sample_rate_per_tree = c(0.5, 0.9, 1.0),
                                   histogram_type = c("UniformAdaptive",
                                                      "Random",
                                                      "QuantilesGlobal",
                                                      "RoundRobin")),
               search_criteria = list(strategy = "RandomDiscrete", 
                                      max_models = 1000, 
                                      stopping_metric = "auc", 
                                      stopping_tolerance = 0.01, 
                                      stopping_rounds = 5, 
                                      seed = 26227709)) 

rf.grid <- h2o.getGrid(grid_id = "gridrf02",
                       sort_by = "auc",
                       decreasing = TRUE)
rf2 <- h2o.getModel(rf.grid@model_ids[[1]])
h2o.saveModel(rf2, path = "/root/Documents/mk/")
summary(rf2)
varimp <- as.data.frame(h2o.varimp(rf2))
h2o.varimp_plot(rf2)
h2o.performance(rf2, newdata = test)

# Graphs
a <- h2o.loadModel("/home/sussa/Documents/GitHub/mk/gridrf02_model_34")
print(va <- a %>% h2o.varimp() %>% as.data.frame() %>% head(., 6)) 

par(mgp=c(2.2,0.45,0), tcl=-0.4, mar=c(2,7.5,1,1))
barplot(va$scaled_importance[6:1],
        horiz = TRUE, las = 1, cex.names=0.9,
        names.arg = c("Total battle deaths", 
                      "Log GDP per capita",
                       "Trade dependence",
                      "Military personnel",
                      "% Urban pop.",
                      "Years mass killing"),
        main = "")
        
mksyr <- h2o.partialPlot(object = a, data = train, cols = c("mksyr"))
p1 <- qplot(mksyr$mksyr, mksyr$mean_response) + geom_line() + theme_classic() + 
        ylim(0, 0.1) + xlab("Years since mass killing") +  ylab("Mean response")

percentpopurban <- h2o.partialPlot(object = a, data = train, cols = c("percentpopurban"))
p2 <- qplot(percentpopurban$percentpopurban, percentpopurban$mean_response) + geom_line() +
        ylim(0, 0.1) + theme_classic() + xlab("% Urban") + ylab("Mean response")

milper <- h2o.partialPlot(object = a, data = train, cols = c("milper"))
p3 <- qplot(milper$milper, milper$mean_response) + geom_line() + theme_classic() +
        ylim(0, 0.25) + xlab("Military personnel") + ylab("Mean response")

tradedependence <- h2o.partialPlot(object = a, data = train, cols = c("tradedependence"))
p4 <- qplot(tradedependence$tradedependence, tradedependence$mean_response) + geom_line() + theme_classic() +
        ylim(0, 0.1) + xlab("Trade dependence") + ylab("Mean response")

logrgdppc <- h2o.partialPlot(object = a, data = train, cols = c("logrgdppc"))
p5 <- qplot(logrgdppc$logrgdppc, logrgdppc$mean_response) + geom_line() + theme_classic() +
        ylim(0, 0.1) + xlab("Log GDP per capita") + ylab("Mean response")

totalbeaths <- h2o.partialPlot(object = a, data = train, cols = c("totalbeaths"))
p6 <- qplot(totalbeaths$totalbeaths, totalbeaths$mean_response) + geom_line() + theme_classic() +
        ylim(0, 0.35) + xlab("Total battle deaths") + ylab("Mean response")

multiplot(p1, p4, p2, p5, p3, p6, cols = 3)

## COW data
df.cowa <- as.h2o(df.cow)

df.cowa$MKstart <- as.factor(df.cowa$MKstart)
h2o.levels(df.cowa$MKstart)

# Partition the data into training, validation and test sets
splits <- h2o.splitFrame(data = df.cowa, 
                         ratios = c(0.7, 0.15),  # 70%, 15%, 15%
                         seed = 42) 

train <- h2o.assign(splits[[1]], "train.hex")   
valid <- h2o.assign(splits[[2]], "valid.hex") 
test <- h2o.assign(splits[[3]], "test.hex")

y <- "MKstart"
x <- setdiff(names(df.ucdp), c(y, "ccode", "year", "rgdppc",
                               "mksyr2", "mksyr3", "sf", "country",
                               "elf2", "polity2sq")) 

# Running the model
rf <- h2o.grid("randomForest", x = x, y = y, training_frame = train, 
               validation_frame = valid, nfolds = 5, grid_id = "gridrf03",
               fold_assignment = "Stratified",
               hyper_params = list(ntrees = c(256, 512, 1024),
                                   max_depth = c(10, 20, 40),
                                   mtries = c(5, 6, 7),
                                   balance_classes = c(TRUE, FALSE),
                                   sample_rate = c(0.5, 0.632, 0.95),
                                   col_sample_rate_per_tree = c(0.5, 0.9, 1.0),
                                   histogram_type = c("UniformAdaptive",
                                                      "Random",
                                                      "QuantilesGlobal",
                                                      "RoundRobin")),
               search_criteria = list(strategy = "RandomDiscrete", 
                                      max_models = 1000, 
                                      stopping_metric = "auc", 
                                      stopping_tolerance = 0.01, 
                                      stopping_rounds = 5, 
                                      seed = 26227709)) 

rf.grid <- h2o.getGrid(grid_id = "gridrf03",
                       sort_by = "auc",
                       decreasing = TRUE)
rf2 <- h2o.getModel(rf.grid@model_ids[[1]])
h2o.saveModel(rf2, path = "/root/Documents/mk/")
summary(rf2)
varimp <- as.data.frame(h2o.varimp(rf2))
h2o.varimp_plot(rf2)
h2o.performance(rf2, newdata = test)

## Graphs
a <- h2o.loadModel("/root/Documents/mk/gridrf03_model_3")
print(va <- a %>% h2o.varimp() %>% as.data.frame() %>% head(., 6)) 
par(mgp=c(2.2,0.45,0), tcl=-0.4, mar=c(2,7.5,1,1))
barplot(va$scaled_importance[6:1],
        horiz = TRUE, las = 1, cex.names=0.9,
        names.arg = c("Total battle deaths", 
                      "Excluded population",
                      "Yrs since mass killing",
                      "Log GDP per capita",
                      "Ethnic polarisation",
                      "Physical integrity"),
        main = "")

physint <- h2o.partialPlot(object = a, data = train, cols = c("physint"))
p1 <- qplot(physint$physint, physint$mean_response) + geom_line() + theme_classic() + 
        xlab("Physical integrity") +  ylab("Mean response")

egiptpolrqnew <- h2o.partialPlot(object = a, data = train, cols = c("egiptpolrqnew"))
p2 <- qplot(egiptpolrqnew$egiptpolrqnew, egiptpolrqnew$mean_response) + geom_line() +
        theme_classic() + xlab("Ethnic polarisation") + ylab("Mean response")

logrgdppc <- h2o.partialPlot(object = a, data = train, cols = c("logrgdppc"))
p3 <- qplot(logrgdppc$logrgdppc, logrgdppc$mean_response) + geom_line() + theme_classic() +
        ylim(0, 0.1) + xlab("Log GDP per capita") + ylab("Mean response")

mksyr <- h2o.partialPlot(object = a, data = train, cols = c("mksyr"))
p4 <- qplot(mksyr$mksyr, mksyr$mean_response) + geom_line() + theme_classic() + 
        ylim(0, 0.1) + xlab("Years since mass killing") +  ylab("Mean response")

exclpop <- h2o.partialPlot(object = a, data = train, cols = c("exclpop"))
p5 <- qplot(exclpop$exclpop, exclpop$mean_response) + geom_line() + theme_classic() +
        ylim(0, 0.1) + xlab("Excluded population") + ylab("Mean response")

totalbeaths <- h2o.partialPlot(object = a, data = train, cols = c("totalbeaths"))
p6 <- qplot(totalbeaths$totalbeaths, totalbeaths$mean_response) + geom_line() + theme_classic() +
        ylim(0, 0.1) + xlab("Total battle deaths") + ylab("Mean response")

multiplot(p1, p4, p2, p5, p3, p6, cols = 3)

## Ethnic war
df.etha <- as.h2o(df.eth)

df.etha$MKstart <- as.factor(df.etha$MKstart) 
h2o.levels(df.etha$MKstart)

# Partition the data into training, validation and test sets
splits <- h2o.splitFrame(data = df.etha, 
                         ratios = c(0.7, 0.15), 
                         seed = 42) 

train <- h2o.assign(splits[[1]], "train.hex")   
valid <- h2o.assign(splits[[2]], "valid.hex") 
test <- h2o.assign(splits[[3]], "test.hex")

y <- "MKstart"
x <- setdiff(names(df.eth), c(y, "ccode", "year", "rgdppc",
                               "mksyr2", "mksyr3", "sf", "country",
                               "elf2", "polity2sq")) 

# Running the model
rf <- h2o.grid("randomForest", x = x, y = y, training_frame = train, 
               validation_frame = valid, nfolds = 5, grid_id = "gridrf04",
               fold_assignment = "Stratified",
               hyper_params = list(ntrees = c(256, 512, 1024),
                                   max_depth = c(10, 20, 40),
                                   mtries = c(5, 6, 7),
                                   balance_classes = c(TRUE, FALSE),
                                   sample_rate = c(0.5, 0.632, 0.95),
                                   col_sample_rate_per_tree = c(0.5, 0.9, 1.0),
                                   histogram_type = c("UniformAdaptive",
                                                      "Random",
                                                      "QuantilesGlobal",
                                                      "RoundRobin")),
               search_criteria = list(strategy = "RandomDiscrete", 
                                      max_models = 1000, 
                                      stopping_metric = "auc", 
                                      stopping_tolerance = 0.01, 
                                      stopping_rounds = 5, 
                                      seed = 26227709)) 

rf.grid <- h2o.getGrid(grid_id = "gridrf04",
                       sort_by = "auc",
                       decreasing = TRUE)
rf2 <- h2o.getModel(rf.grid@model_ids[[1]])
h2o.saveModel(rf2, path = "/root/Documents/mk/")
summary(rf2)
varimp <- as.data.frame(h2o.varimp(rf2))
h2o.varimp_plot(rf2)
h2o.performance(rf2, newdata = test)

# Graphs
a <- h2o.loadModel("/root/Documents/mk/gridrf04_model_14")
print(va <- a %>% h2o.varimp() %>% as.data.frame() %>% head(., 6)) 

par(mgp=c(2.2,0.45,0), tcl=-0.4, mar=c(2,7.5,1,1))
barplot(va$scaled_importance[6:1],
        horiz = TRUE, las = 1, cex.names=0.9,
        names.arg = c("Military expenditure",
                      "Total trade", 
                      "% Urban",
                      "Trade dependence",
                      "Military personnel",
                      "CINC"),
        main = "")


cinc <- h2o.partialPlot(object = a, data = train, cols = c("cinc"))
p1 <- qplot(cinc$cinc, cinc$mean_response) + geom_line() + theme_classic() + 
        xlab("CINC") +  ylab("Mean response")

milper <- h2o.partialPlot(object = a, data = train, cols = c("milper"))
p2 <- qplot(milper$milper, milper$mean_response) + geom_line() +
        theme_classic() + xlab("Military personnel") + ylab("Mean response")

tradedependence <- h2o.partialPlot(object = a, data = train, cols = c("tradedependence"))
p3 <- qplot(tradedependence$tradedependence, tradedependence$mean_response) + geom_line() + theme_classic() +
        xlab("Trade dependence") + ylab("Mean response")

percentpopurban <- h2o.partialPlot(object = a, data = train, cols = c("percentpopurban"))
p4 <- qplot(percentpopurban$percentpopurban, percentpopurban$mean_response) + geom_line() +
       theme_classic() + xlab("% Urban") + ylab("Mean response")

totaltrade <- h2o.partialPlot(object = a, data = train, cols = c("totaltrade"))
p5 <- qplot(totaltrade$totaltrade, totaltrade$mean_response) + geom_line() +
        theme_classic() + xlab("Total trade") + ylab("Mean response")

milex <- h2o.partialPlot(object = a, data = train, cols = c("milex"))
p6 <- qplot(milex$milex, milex$mean_response) + geom_line() +
        theme_classic() + xlab("Military expenditure") + ylab("Mean response")

multiplot(p1, p4, p2, p5, p3, p6, cols = 3)

####################################
### Genocide/Politicide Variable ###
####################################

## The code below replicates the same analyses presented above
## but using a measure of genocide/politicide coded by Harff (2003).

## Data wrangling
df3 <- haven::read_dta("data/uamkstart.dta") %>% setDT()

sd.cols <- c("UCDPcivilwarstart", "UCDPcivilwarongoing", "COWcivilwarstart",
             "COWcivilwarongoing", "ethnowarstart", "ethnowarongoing",
             "assdummy", "demdummy", "elf", "lmtnest", "pop", "realgdp",
             "rgdppc", "polity2", "exclpop", "discpop", "polrqnew",
             "poltrqnew", "egiptpolrqnew", "egippolrqnew", "discrim",
             "elf2", "interstatewar", "milex", "milper", "percentpopurban",
             "postcoldwar", "coupdummy", "riotdummy", "territoryaims",
             "totaltrade", "tradedependence", "militias", "physint", "cinc",
             "totalbeaths", "change", "guerrilladummy", "sf", "regtrans")

df4 <- cbind(df3, df3[, shift(.SD, 1, give.names = TRUE),
                      by = ccode, .SDcols = sd.cols]) 

# Remove the second `ccode` variable
df4 <- as.data.frame(df4[, -c(75)])

# Add new variables
df4$logrgdppc_lag_1 <- log(df4$rgdppc_lag_1)
df4$polity2sq_lag_1 <- df4$polity2_lag_1^2

# Renaming variables
df5 <- as.data.frame(df4[, c(1:4, 72:116)])
names(df5) <- sub("_.*","", names(df5)) 

#################################################
### Distributed Random Forests - Harff (2003) ###
#################################################

df5a <- as.h2o(df5)

df5a$uamkstart <- as.factor(df5a$uamkstart)  #encode the binary repsonse as a factor
h2o.levels(df5a$uamkstart)

# Partition the data into training, validation and test sets
splits <- h2o.splitFrame(data = df5a, 
                         ratios = c(0.7, 0.15),  # 70%, 15%, 15%
                         seed = 42)  # reproducibility


train <- h2o.assign(splits[[1]], "train.hex")   
valid <- h2o.assign(splits[[2]], "valid.hex") 
test <- h2o.assign(splits[[3]], "test.hex")

y <- "uamkstart"
x <- setdiff(names(df5), c(y, "ccode", "year", "rgdppc",
                           "uamkyr2", "uamkyr3", "sf", "country",
                           "elf2", "polity2sq")) 
# Running the model
rf <- h2o.grid("randomForest", x = x, y = y, training_frame = train, 
               validation_frame = valid, nfolds = 5, 
               grid_id = "gridrf05",
               fold_assignment = "Stratified",
               hyper_params = list(ntrees = c(256, 512, 1024),
                                   max_depth = c(10, 20, 40),
                                   mtries = c(5, 6, 7),
                                   balance_classes = c(TRUE, FALSE),
                                   sample_rate = c(0.5, 0.632, 0.95),
                                   col_sample_rate_per_tree = c(0.5, 0.9, 1.0),
                                   histogram_type = c("UniformAdaptive",
                                                      "Random",
                                                      "QuantilesGlobal",
                                                      "RoundRobin")),
               search_criteria = list(strategy = "RandomDiscrete", 
                                      max_models = 1000, 
                                      stopping_metric = "auc", 
                                      stopping_tolerance = 0.01, 
                                      stopping_rounds = 5, 
                                      seed = 26227709)) 
                                      
# Saving the most accurate model
rf.grid <- h2o.getGrid(grid_id = "gridrf05",
                       sort_by = "auc",
                       decreasing = TRUE)

rf2 <- h2o.getModel(rf.grid@model_ids[[1]])
h2o.saveModel(rf2, path = "/root/Documents/mk/")
summary(rf2)
varimp <- as.data.frame(h2o.varimp(rf2))
h2o.varimp_plot(rf2)
h2o.performance(rf2, newdata = test)

## Graphs
a <- h2o.loadModel("/home/sussa/Documents/GitHub/mk/gridrf05_model_27")
print(va <- a %>% h2o.varimp() %>% as.data.frame() %>% head(., 6)) 

par(mgp=c(2.2,0.45,0), tcl=-0.4, mar=c(2,7.5,1,1))
barplot(va$scaled_importance[6:1],
        horiz = TRUE, las = 1, cex.names=0.9,
        names.arg = c("Real GDP", 
                      "Military expenditure",
                      "Total trade", 
                      "Military personnel",
                      "CINC", 
                      "% Urban"),
        main = "")

percentpopurban <- h2o.partialPlot(object = a, data = train, cols = c("percentpopurban"))
p1 <- qplot(percentpopurban$percentpopurban, percentpopurban$mean_response) + geom_line() +
        theme_classic() + xlab("% Urban") + ylab("Mean response")
        
cinc <- h2o.partialPlot(object = a, data = train, cols = c("cinc"))
p2 <- qplot(cinc$cinc, cinc$mean_response) + geom_line() + theme_classic() + 
        xlab("CINC") +  ylab("Mean response")

milper <- h2o.partialPlot(object = a, data = train, cols = c("milper"))
p3 <- qplot(milper$milper, milper$mean_response) + geom_line() +
        theme_classic() + xlab("Military personnel") + ylab("Mean response")

totaltrade <- h2o.partialPlot(object = a, data = train, cols = c("totaltrade"))
p4 <- qplot(totaltrade$totaltrade, totaltrade$mean_response) + geom_line() + theme_classic() +
        xlab("Total trade") + ylab("Mean response")

milex <- h2o.partialPlot(object = a, data = train, cols = c("milex"))
p5 <- qplot(milex$milex, milex$mean_response) + geom_line() +
        theme_classic() + xlab("Military expenditure") + ylab("Mean response")
        
realgdp <- h2o.partialPlot(object = a, data = train, cols = c("realgdp"))
p6 <- qplot(realgdp$realgdp, realgdp$mean_response) + geom_line() +
        theme_classic() + xlab("Real GDP") + ylab("Mean response")

multiplot(p1, p4, p2, p5, p3, p6, cols = 3)


## UCDP == 1
df.ucdp2 <- df5 %>% filter(UCDPcivilwarongoing == 1)
df.ucdp2a <- as.h2o(df.ucdp2)

df.ucdp2a$uamkstart <- as.factor(df.ucdp2a$uamkstart)  #encode the binary repsonse as a factor
h2o.levels(df.ucdp2a$uamkstart)

# Partition the data into training, validation and test sets
splits <- h2o.splitFrame(data = df.ucdp2a, 
                         ratios = c(0.7, 0.15),  # 70%, 15%, 15%
                         seed = 42)  # reproducibility


train <- h2o.assign(splits[[1]], "train.hex")   
valid <- h2o.assign(splits[[2]], "valid.hex") 
test <- h2o.assign(splits[[3]], "test.hex")

y <- "uamkstart"
x <- setdiff(names(df.ucdp2), c(y, "ccode", "year", "rgdppc",
                           "uamkyr2", "uamkyr3", "sf", "country",
                           "elf2", "polity2sq")) 

# Running the model
rf <- h2o.grid("randomForest", x = x, y = y, training_frame = train, 
               validation_frame = valid, nfolds = 5, grid_id = "gridrf06",
               fold_assignment = "Stratified",
               hyper_params = list(ntrees = c(256, 512, 1024),
                                   max_depth = c(10, 20, 40),
                                   mtries = c(5, 6, 7),
                                   balance_classes = c(TRUE, FALSE),
                                   sample_rate = c(0.5, 0.632, 0.95),
                                   col_sample_rate_per_tree = c(0.5, 0.9, 1.0),
                                   histogram_type = c("UniformAdaptive",
                                                      "Random",
                                                      "QuantilesGlobal",
                                                      "RoundRobin")),
               search_criteria = list(strategy = "RandomDiscrete", 
                                      max_models = 1000, 
                                      stopping_metric = "auc", 
                                      stopping_tolerance = 0.01, 
                                      stopping_rounds = 5, 
                                      seed = 26227709)) 

rf.grid <- h2o.getGrid(grid_id = "gridrf06",
                       sort_by = "auc",
                       decreasing = TRUE)
rf2 <- h2o.getModel(rf.grid@model_ids[[1]])
h2o.saveModel(rf2, path = "/root/Documents/mk/")
summary(rf2)
varimp <- as.data.frame(h2o.varimp(rf2))
h2o.varimp_plot(rf2)
h2o.performance(rf2, newdata = test)

## Graphs
a <- h2o.loadModel("/home/sussa/Documents/GitHub/mk/gridrf06_model_43")
print(va <- a %>% h2o.varimp() %>% as.data.frame() %>% head(., 6)) 

par(mgp=c(2.2,0.45,0), tcl=-0.4, mar=c(2,7.5,1,1))
barplot(va$scaled_importance[6:1],
        horiz = TRUE, las = 1, cex.names=0.9,
        names.arg = c("Excluded population",
                      "% Urban pop.", 
                      "Trade dependence",
                      "Log GDP per capita",
                      "Years since genocide", 
                      "Military personnel"),
        main = "")
        
milper <- h2o.partialPlot(object = a, data = train, cols = c("milper"))
p1 <- qplot(milper$milper, milper$mean_response) + geom_line() +
        theme_classic() + xlab("Military personnel") + ylab("Mean response")

uamkyr <- h2o.partialPlot(object = a, data = train, cols = c("uamkyr"))
p2 <- qplot(uamkyr$uamkyr, uamkyr$mean_response) + geom_line() + theme_classic() + 
        xlab("Years since genocide") +  ylab("Mean response")
        
logrgdppc <- h2o.partialPlot(object = a, data = train, cols = c("logrgdppc"))
p3 <- qplot(logrgdppc$logrgdppc, logrgdppc$mean_response) + geom_line() + theme_classic() +
        xlab("Log GDP per capita") + ylab("Mean response")

tradedependence <- h2o.partialPlot(object = a, data = train, cols = c("tradedependence"))
p4 <- qplot(tradedependence$tradedependence, tradedependence$mean_response) + geom_line() +
        theme_classic() + xlab("Trade dependence") + ylab("Mean response")

percentpopurban <- h2o.partialPlot(object = a, data = train, cols = c("percentpopurban"))
p5 <- qplot(percentpopurban$percentpopurban, percentpopurban$mean_response) + geom_line() +
        theme_classic() + xlab("% Urban") + ylab("Mean response")
        
exclpop <- h2o.partialPlot(object = a, data = train, cols = c("exclpop"))
p6 <- qplot(exclpop$exclpop, exclpop$mean_response) + geom_line() +
        theme_classic() + xlab("Excluded population") + ylab("Mean response")

multiplot(p1, p4, p2, p5, p3, p6, cols = 3)

## COW == 1
df.cow2 <- df5 %>% filter(COWcivilwarongoing == 1)

df.cow2a$uamkstart <- as.factor(df.cow2a$uamkstart)  #encode the binary repsonse as a factor
h2o.levels(df.cow2a$uamkstart)

# Partition the data into training, validation and test sets
splits <- h2o.splitFrame(data = df.cow2a, 
                         ratios = c(0.7, 0.15),  # 70%, 15%, 15%
                         seed = 42)  # reproducibility


train <- h2o.assign(splits[[1]], "train.hex")   
valid <- h2o.assign(splits[[2]], "valid.hex") 
test <- h2o.assign(splits[[3]], "test.hex")

y <- "uamkstart"
x <- setdiff(names(df.cow2), c(y, "ccode", "year", "rgdppc",
                               "uamkyr2", "uamkyr3", "sf", "country",
                               "elf2", "polity2sq")) 

## Running the model
rf <- h2o.grid("randomForest", x = x, y = y, training_frame = train, 
               validation_frame = valid, nfolds = 5, grid_id = "gridrf07",
               fold_assignment = "Stratified",
               hyper_params = list(ntrees = c(256, 512, 1024),
                                   max_depth = c(10, 20, 40),
                                   mtries = c(5, 6, 7),
                                   balance_classes = c(TRUE, FALSE),
                                   sample_rate = c(0.5, 0.632, 0.95),
                                   col_sample_rate_per_tree = c(0.5, 0.9, 1.0),
                                   histogram_type = c("UniformAdaptive",
                                                      "Random",
                                                      "QuantilesGlobal",
                                                      "RoundRobin")),
               search_criteria = list(strategy = "RandomDiscrete", 
                                      max_models = 1000, 
                                      stopping_metric = "auc", 
                                      stopping_tolerance = 0.01, 
                                      stopping_rounds = 5, 
                                      seed = 26227709)) 

rf.grid <- h2o.getGrid(grid_id = "gridrf07",
                       sort_by = "auc",
                       decreasing = TRUE)
rf2 <- h2o.getModel(rf.grid@model_ids[[1]])
h2o.saveModel(rf2, path = "/root/Documents/mk/")
summary(rf2)
varimp <- as.data.frame(h2o.varimp(rf2))
h2o.varimp_plot(rf2)
h2o.performance(rf2, newdata = test)

## Graphs
a <- h2o.loadModel("/home/sussa/Documents/GitHub/mk/gridrf07_model_27")
print(va <- a %>% h2o.varimp() %>% as.data.frame() %>% head(., 6)) 

par(mgp=c(2.2,0.45,0), tcl=-0.4, mar=c(2,7.5,1,1))
barplot(va$scaled_importance[6:1],
        horiz = TRUE, las = 1, cex.names=0.9,
        names.arg = c("Military expenditure",
                      "% Urban pop.", 
                      "Total trade",
                      "Years since genocide", 
                      "Military personnel", 
                      "Trade dependence"),
        main = "")

uamkyr <- h2o.partialPlot(object = a, data = train, cols = c("uamkyr"))
p3 <- qplot(uamkyr$uamkyr, uamkyr$mean_response) + geom_line() + theme_classic() + 
        xlab("Years since genocide") +  ylab("Mean response")

milper <- h2o.partialPlot(object = a, data = train, cols = c("milper"))
p2 <- qplot(milper$milper, milper$mean_response) + geom_line() +
        theme_classic() + xlab("Military personnel") + ylab("Mean response")

totaltrade <- h2o.partialPlot(object = a, data = train, cols = c("totaltrade"))
p4 <- qplot(totaltrade$totaltrade, totaltrade$mean_response) + geom_line() + theme_classic() +
        xlab("Total trade") + ylab("Mean response")

percentpopurban <- h2o.partialPlot(object = a, data = train, cols = c("percentpopurban"))
p5 <- qplot(percentpopurban$percentpopurban, percentpopurban$mean_response) + geom_line() +
        theme_classic() + xlab("% Urban") + ylab("Mean response")

tradedependence <- h2o.partialPlot(object = a, data = train, cols = c("tradedependence"))
p1 <- qplot(tradedependence$tradedependence, tradedependence$mean_response) + geom_line() +
        theme_classic() + xlab("Trade dependence") + ylab("Mean response")

milex <- h2o.partialPlot(object = a, data = train, cols = c("milex"))
p6 <- qplot(milex$milex, milex$mean_response) + geom_line() +
        theme_classic() + xlab("Military expenditure") + ylab("Mean response")

multiplot(p1, p4, p2, p5, p3, p6, cols = 3)

## ETHONSET == 1
df.eth2 <- df5 %>% filter(ethnowarongoing == 1)

df.eth2a <- as.h2o(df.eth2)

df.eth2a$uamkstart <- as.factor(df.eth2a$uamkstart)  #encode the binary repsonse as a factor
h2o.levels(df.eth2a$uamkstart)

# Partition the data into training, validation and test sets
splits <- h2o.splitFrame(data = df.eth2a, 
                         ratios = c(0.7, 0.15),  # 70%, 15%, 15%
                         seed = 42)  # reproducibility


train <- h2o.assign(splits[[1]], "train.hex")   
valid <- h2o.assign(splits[[2]], "valid.hex") 
test <- h2o.assign(splits[[3]], "test.hex")

y <- "uamkstart"
x <- setdiff(names(df.eth2), c(y, "ccode", "year", "rgdppc",
                               "uamkyr2", "uamkyr3", "sf", "country",
                               "elf2", "polity2sq")) 

# Running the model
rf <- h2o.grid("randomForest", x = x, y = y, training_frame = train, 
               validation_frame = valid, nfolds = 5, grid_id = "gridrf08",
               fold_assignment = "Stratified",
               hyper_params = list(ntrees = c(256, 512, 1024),
                                   max_depth = c(10, 20, 40),
                                   mtries = c(5, 6, 7),
                                   balance_classes = c(TRUE, FALSE),
                                   sample_rate = c(0.5, 0.632, 0.95),
                                   col_sample_rate_per_tree = c(0.5, 0.9, 1.0),
                                   histogram_type = c("UniformAdaptive",
                                                      "Random",
                                                      "QuantilesGlobal",
                                                      "RoundRobin")),
               search_criteria = list(strategy = "RandomDiscrete", 
                                      max_models = 1000, 
                                      stopping_metric = "auc", 
                                      stopping_tolerance = 0.01, 
                                      stopping_rounds = 5, 
                                      seed = 26227709)) 

rf.grid <- h2o.getGrid(grid_id = "gridrf08",
                       sort_by = "auc",
                       decreasing = TRUE)
rf2 <- h2o.getModel(rf.grid@model_ids[[1]])
h2o.saveModel(rf2, path = "/root/Documents/mk/")
summary(rf2)
varimp <- as.data.frame(h2o.varimp(rf2))
h2o.varimp_plot(rf2)
h2o.performance(rf2, newdata = test)

## Graphs
a <- h2o.loadModel("/home/sussa/Documents/GitHub/mk/gridrf08_model_55")
print(va <- a %>% h2o.varimp() %>% as.data.frame() %>% head(., 6)) 

par(mgp=c(2.2,0.45,0), tcl=-0.4, mar=c(2,7.5,1,1))
barplot(va$scaled_importance[6:1],
        horiz = TRUE, las = 1, cex.names=0.9,
        names.arg = c("Total battle deaths", 
                      "Military personnel",
                      "Log GDP per capita", 
                      "CINC",
                      "% Urban",
                      "Military expenditure"),
        main = "")

cinc <- h2o.partialPlot(object = a, data = train, cols = c("cinc"))
p3 <- qplot(cinc$cinc, cinc$mean_response) + geom_line() + theme_classic() + 
        xlab("CINC") +  ylab("Mean response")

milper <- h2o.partialPlot(object = a, data = train, cols = c("milper"))
p5 <- qplot(milper$milper, milper$mean_response) + geom_line() +
        theme_classic() + xlab("Military personnel") + ylab("Mean response")

logrgdppc <- h2o.partialPlot(object = a, data = train, cols = c("logrgdppc"))
p4 <- qplot(logrgdppc$logrgdppc, logrgdppc$mean_response) + geom_line() + theme_classic() +
        xlab("Log GDP per capita") + ylab("Mean response")

percentpopurban <- h2o.partialPlot(object = a, data = train, cols = c("percentpopurban"))
p2 <- qplot(percentpopurban$percentpopurban, percentpopurban$mean_response) + geom_line() +
        theme_classic() + xlab("% Urban") + ylab("Mean response")

totalbeaths <- h2o.partialPlot(object = a, data = train, cols = c("totalbeaths"))
p6 <- qplot(totalbeaths$totalbeaths, totalbeaths$mean_response) + geom_line() +
        theme_classic() + xlab("Total battle deaths") + ylab("Mean response")

milex <- h2o.partialPlot(object = a, data = train, cols = c("milex"))
p1 <- qplot(milex$milex, milex$mean_response) + geom_line() +
        theme_classic() + xlab("Military expenditure") + ylab("Mean response")

multiplot(p1, p4, p2, p5, p3, p6, cols = 3)


##############################################
### Genocides and Politicides (Harff 2003) ###
##############################################

# Preparing the dataset
df3 <- haven::read_dta("data/uamkstart.dta") %>% setDT()
sd.cols <- c("UCDPcivilwarstart", "UCDPcivilwarongoing", "COWcivilwarstart",
             "COWcivilwarongoing", "ethnowarstart", "ethnowarongoing",
             "assdummy", "demdummy", "elf", "lmtnest", "pop", "realgdp",
             "rgdppc", "polity2", "exclpop", "discpop", "polrqnew",
             "poltrqnew", "egiptpolrqnew", "egippolrqnew", "discrim",
             "elf2", "interstatewar", "milex", "milper", "percentpopurban",
             "postcoldwar", "coupdummy", "riotdummy", "territoryaims",
             "totaltrade", "tradedependence", "militias", "physint", "cinc",
             "totalbeaths", "change", "guerrilladummy", "sf", "regtrans")

df4 <- cbind(df3, df3[, shift(.SD, 1, give.names = TRUE),
                    by = ccode, .SDcols = sd.cols]) 

# Remove the second `ccode` variable
df4 <- as.data.frame(df4[, -c(75)])

# Add new variables
df4$logrgdppc_lag_1 <- log(df4$rgdppc_lag_1)
df4$polity2sq_lag_1 <- df4$polity2_lag_1^2

# Renaming variables
df5 <- as.data.frame(df4[, c(1:4, 72:116)])
names(df5) <- sub("_.*","", names(df5)) 

## Extreme Bounds
free.variables <- c("logrgdppc", "polity2", "uamkyr")
civilwar.variables <- c("UCDPcivilwarongoing", "UCDPcivilwarstart",
                        "COWcivilwarongoing", "COWcivilwarstart",
                        "ethnowarongoing", "ethnowarstart")
doubtful.variables <- c("UCDPcivilwarongoing", "UCDPcivilwarstart",
                        "COWcivilwarongoing", "COWcivilwarstart",
                        "ethnowarongoing", "ethnowarstart", "assdummy",
                        "totaltrade", "tradedependence", "milper", "milex",
                        "pop", "totalbeaths", "guerrilladummy", "regtrans",
                        "riotdummy", "territoryaims", "militias",
                        "physint", "percentpopurban", "coupdummy",
                        "postcoldwar",  "lmtnest", "realgdp", "discrim",
                        "exclpop", "discpop", "elf",  "polrqnew",
                        "egippolrqnew", "poltrqnew", "egiptpolrqnew",
                        "polity2sq")
m1 <- eba(y = "uamkstart", free = free.variables,
          exclusive = list(civilwar.variables),
          doubtful = doubtful.variables, k = 0:4,
          data = df5, vif = 7, level = 0.9, 
          se.fun = se.clustered.robust)
          
hist(m1, variables = c("logrgdppc", "polity2", "polity2sq", "uamkyr",
                       "UCDPcivilwarongoing",
                       "UCDPcivilwarstart", "COWcivilwarongoing",
                       "COWcivilwarstart", "ethnowarongoing", "ethnowarstart",
                       "assdummy", "totaltrade", "tradedependence", "milper",
                       "milex","pop", "totalbeaths", "guerrilladummy", "regtrans",
                       "riotdummy", "territoryaims", "militias", "physint",
                       "percentpopurban", "coupdummy", "postcoldwar",
                       "lmtnest", "realgdp", "discrim", "exclpop", "discpop",
                       "elf", "polrqnew", "egippolrqnew", "poltrqnew",
                       "egiptpolrqnew"),
     main = c("Log GDP capita", "Polity IV", "Polity IV^2", "Years last genocide",
              "UCDP ongoing", "UCDP onset", "COW ongoing", "COW onset", 
              "Ethnic ongoing", "Ethnic onset", "Assassination", "Total trade", 
              "Trade dependence", "Military personnel", "Military expenditure", "Population", 
              "Total deaths", "Guerrilla", "Regime transition", "Riots",
              "Territory Aims", "Militias", "Physical integrity", "% Urban",
              "Coups", "Post-Cold War", "Mountainous terrain", "Real GDP",
              "Discrimination", "Excl pop", "Discrim pop", "ELF", "Groups/Eth relevant", 
              "Group/Tot pop", "Inc groups/Eth relevant", "Inc groups/Tot pop"),
     density.col = "black", mu.col = "red3")

### Ongoing Civil Wars

# UCDPcivilwarongoing == 1
df.ucdp2 <- df5 %>% filter(UCDPcivilwarongoing == 1)
doubtful.variables <- c("assdummy", "totaltrade", "tradedependence",
                        "milper", "milex", "pop", "totalbeaths",
                        "guerrilladummy", "regtrans", "riotdummy",
                        "territoryaims", "militias", "physint",
                        "percentpopurban", "coupdummy", "postcoldwar",
                        "lmtnest", "realgdp", "discrim", "exclpop",
                        "discpop", "elf",  "polrqnew", "egippolrqnew",
                        "poltrqnew", "egiptpolrqnew", "polity2sq")

m1 <- eba(y = "uamkstart", free = free.variables,
          doubtful = doubtful.variables, k = 0:4,
          data = df.ucdp2, vif = 7, draws = 50000,
          level = 0.9, se.fun = se.clustered.robust)
          
hist(m1, variables = c("logrgdppc", "polity2", "polity2sq", "uamkyr",
                       "assdummy", "totaltrade", "tradedependence", "milper",
                       "milex","pop", "totalbeaths", "guerrilladummy", "regtrans",
                       "riotdummy", "territoryaims", "militias", "physint",
                       "percentpopurban", "coupdummy", "postcoldwar",
                       "lmtnest", "realgdp", "discrim", "exclpop", "discpop",
                       "elf", "polrqnew", "egippolrqnew", "poltrqnew",
                       "egiptpolrqnew"),
     main = c("Log GDP capita", "Polity IV", "Polity IV^2", "Years last genocide",
              "Assassination", "Total trade", 
              "Trade dependence", "Military personnel", "Military expenditure", "Population", 
              "Total deaths", "Guerrilla", "Regime transition", "Riots",
              "Territory Aims", "Militias", "Physical integrity", "% Urban",
              "Coups", "Post-Cold War", "Mountainous terrain", "Real GDP",
              "Discrimination", "Excl pop", "Discrim pop", "ELF", "Groups/Eth relevant", 
              "Groups/Tot pop", "Inc groups/Eth relevant", "Inc groups/Tot pop"),
     density.col = "black", mu.col = "red3")
     
# COWcivilwarongoing == 1
df.cow2 <- df5 %>% filter(COWcivilwarongoing == 1)
doubtful.variables <- c("assdummy", "totaltrade", "tradedependence",
                        "milper", "milex", "pop", "totalbeaths",
                        "guerrilladummy", "regtrans", "riotdummy",
                        "territoryaims", "militias", "physint",
                        "percentpopurban", "coupdummy", "postcoldwar",
                        "lmtnest", "realgdp", "discrim", "exclpop",
                        "discpop", "elf",  "polrqnew", "egippolrqnew",
                        "poltrqnew", "egiptpolrqnew", "polity2sq")

m1 <- eba(y = "uamkstart", free = free.variables,
          doubtful = doubtful.variables, k = 0:4,
          data = df.cow2, vif = 7, draws = 50000,
          level = 0.9, se.fun = se.clustered.robust)

hist(m1, variables = c("logrgdppc", "polity2", "polity2sq", "uamkyr",
                       "assdummy", "totaltrade", "tradedependence", "milper",
                       "milex","pop", "totalbeaths", "guerrilladummy", "regtrans",
                       "riotdummy", "territoryaims", "militias", "physint",
                       "percentpopurban", "coupdummy", "postcoldwar",
                       "lmtnest", "realgdp", "discrim", "exclpop", "discpop",
                       "elf", "polrqnew", "egippolrqnew", "poltrqnew",
                       "egiptpolrqnew"),
     main = c("Log GDP capita", "Polity IV", "Polity IV^2", "Years last genocide",
              "Assassination", "Total trade", 
              "Trade dependence", "Military personnel", "Military expenditure", "Population", 
              "Total deaths", "Guerrilla", "Regime transition", "Riots",
              "Territory Aims", "Militias", "Physical integrity", "% Urban",
              "Coups", "Post-Cold War", "Mountainous terrain", "Real GDP",
              "Discrimination", "Excl pop", "Discrim pop", "ELF", "Groups/Eth relevant", 
              "Groups/Tot pop", "Inc groups/Eth relevant", "Inc groups/Tot pop"),
     density.col = "black", mu.col = "red3")
     
# ethnic civil war == 1
df.eth2 <- df5 %>% filter(ethnowarongoing == 1)
doubtful.variables <- c("assdummy", "totaltrade", "tradedependence",
                        "milper", "milex", "pop", "totalbeaths",
                        "guerrilladummy", "regtrans", "riotdummy",
                        "territoryaims", "militias", "physint",
                        "percentpopurban", "coupdummy", "postcoldwar",
                        "lmtnest", "realgdp", "discrim", "exclpop", 
                        "discpop", "elf",  "polrqnew", "egippolrqnew",
                        "poltrqnew", "egiptpolrqnew", "polity2sq")

m1 <- eba(y = "uamkstart", free = free.variables,
          doubtful = doubtful.variables, k = 0:4,
          data = df.eth2, vif = 7, draws = 50000,
          level = 0.9, se.fun = se.clustered.robust)

hist(m1, variables = c("logrgdppc", "polity2", "polity2sq", "uamkyr",
                       "assdummy", "totaltrade", "tradedependence", "milper",
                       "milex","pop", "totalbeaths", "guerrilladummy", "regtrans",
                       "riotdummy", "territoryaims", "militias", "physint",
                       "percentpopurban", "coupdummy", "postcoldwar",
                       "lmtnest", "realgdp", "discrim", "exclpop", "discpop",
                       "elf", "polrqnew", "egippolrqnew", "poltrqnew",
                       "egiptpolrqnew"),
     main = c("Log GDP capita", "Polity IV", "Polity IV^2", "Years last genocide",
              "Assassination", "Total trade", 
              "Trade dependence", "Military personnel", "Military expenditure", "Population", 
              "Total deaths", "Guerrilla", "Regime transition", "Riots",
              "Territory Aims", "Militias", "Physical integrity", "% Urban",
              "Coups", "Post-Cold War", "Mountainous terrain", "Real GDP",
              "Discrimination", "Excl pop", "Discrim pop", "ELF", "Groups/Eth relevant", 
              "Groups/Tot pop", "Inc groups/Eth relevant", "Inc groups/Tot pop"),
     density.col = "black", mu.col = "red3")
     
########################################################
### Genocide/Politicide -- Distributed Random Forest ###
########################################################

df5a <- as.h2o(df5)

df5a$uamkstart <- as.factor(df5a$uamkstart)  #encode the binary repsonse as a factor
h2o.levels(df5a$uamkstart)

# Partition the data into training, validation and test sets
splits <- h2o.splitFrame(data = df5a, 
                         ratios = c(0.7, 0.15),  # 70%, 15%, 15%
                         seed = 42)  # reproducibility


train <- h2o.assign(splits[[1]], "train.hex")   
valid <- h2o.assign(splits[[2]], "valid.hex") 
test <- h2o.assign(splits[[3]], "test.hex")

y <- "uamkstart"
x <- setdiff(names(df5), c(y, "ccode", "year", "rgdppc",
                           "uamkyr2", "uamkyr3", "sf", "country",
                           "elf2", "polity2sq")) 

# Running the model
rf <- h2o.grid("randomForest", x = x, y = y, training_frame = train, 
               validation_frame = valid, nfolds = 5, 
               grid_id = "gridrf05",
               fold_assignment = "Stratified",
               hyper_params = list(ntrees = c(256, 512, 1024),
                                   max_depth = c(10, 20, 40),
                                   mtries = c(5, 6, 7),
                                   balance_classes = c(TRUE, FALSE),
                                   sample_rate = c(0.5, 0.632, 0.95),
                                   col_sample_rate_per_tree = c(0.5, 0.9, 1.0),
                                   histogram_type = c("UniformAdaptive",
                                                      "Random",
                                                      "QuantilesGlobal",
                                                      "RoundRobin")),
               search_criteria = list(strategy = "RandomDiscrete", 
                                      max_models = 500, 
                                      stopping_metric = "auc", 
                                      stopping_tolerance = 0.01, 
                                      stopping_rounds = 5, 
                                      seed = 26227709)) 

# Saving the most accurate model
rf.grid <- h2o.getGrid(grid_id = "gridrf05",
                       sort_by = "auc",
                       decreasing = TRUE)

rf2 <- h2o.getModel(rf.grid@model_ids[[1]])
h2o.saveModel(rf2, path = "/root/Documents/mk/")
summary(rf2)
varimp <- as.data.frame(h2o.varimp(rf2))
h2o.varimp_plot(rf2)
h2o.performance(rf2, newdata = test)

## UCDP == 1
df.ucdp2a <- as.h2o(df.ucdp2)

df.ucdp2a$uamkstart <- as.factor(df.ucdp2a$uamkstart)  #encode the binary repsonse as a factor
h2o.levels(df.ucdp2a$uamkstart)

# Partition the data into training, validation and test sets
splits <- h2o.splitFrame(data = df.ucdp2a, 
                         ratios = c(0.7, 0.15),  # 70%, 15%, 15%
                         seed = 42)  # reproducibility


train <- h2o.assign(splits[[1]], "train.hex")   
valid <- h2o.assign(splits[[2]], "valid.hex") 
test <- h2o.assign(splits[[3]], "test.hex")

y <- "uamkstart"
x <- setdiff(names(df.ucdp2), c(y, "ccode", "year", "rgdppc",
                           "uamkyr2", "uamkyr3", "sf", "country",
                           "elf2", "polity2sq")) 

# Running the model
rf <- h2o.grid("randomForest", x = x, y = y, training_frame = train, 
               validation_frame = valid, nfolds = 5, grid_id = "gridrf06",
               fold_assignment = "Stratified",
               hyper_params = list(ntrees = c(256, 512, 1024),
                                   max_depth = c(10, 20, 40),
                                   mtries = c(5, 6, 7),
                                   balance_classes = c(TRUE, FALSE),
                                   sample_rate = c(0.5, 0.632, 0.95),
                                   col_sample_rate_per_tree = c(0.5, 0.9, 1.0),
                                   histogram_type = c("UniformAdaptive",
                                                      "Random",
                                                      "QuantilesGlobal",
                                                      "RoundRobin")),
               search_criteria = list(strategy = "RandomDiscrete", 
                                      max_models = 100, 
                                      stopping_metric = "auc", 
                                      stopping_tolerance = 0.01, 
                                      stopping_rounds = 5, 
                                      seed = 26227709)) 

rf.grid <- h2o.getGrid(grid_id = "gridrf06",
                       sort_by = "auc",
                       decreasing = TRUE)
rf2 <- h2o.getModel(rf.grid@model_ids[[1]])
h2o.saveModel(rf2, path = "/root/Documents/mk/")
summary(rf2)
varimp <- as.data.frame(h2o.varimp(rf2))
h2o.varimp_plot(rf2)
h2o.performance(rf2, newdata = test)

## COW == 1
df.cow2a <- as.h2o(df.cow2)

df.cow2a$uamkstart <- as.factor(df.cow2a$uamkstart)  #encode the binary repsonse as a factor
h2o.levels(df.cow2a$uamkstart)

# Partition the data into training, validation and test sets
splits <- h2o.splitFrame(data = df.cow2a, 
                         ratios = c(0.7, 0.15),  # 70%, 15%, 15%
                         seed = 42)  # reproducibility


train <- h2o.assign(splits[[1]], "train.hex")   
valid <- h2o.assign(splits[[2]], "valid.hex") 
test <- h2o.assign(splits[[3]], "test.hex")

y <- "uamkstart"
x <- setdiff(names(df.cow2), c(y, "ccode", "year", "rgdppc",
                           "uamkyr2", "uamkyr3", "sf", "country",
                           "elf2", "polity2sq")) 

# Running the model
rf <- h2o.grid("randomForest", x = x, y = y, training_frame = train, 
               validation_frame = valid, nfolds = 5, grid_id = "gridrf07",
               fold_assignment = "Stratified",
               hyper_params = list(ntrees = c(256, 512, 1024),
                                   max_depth = c(10, 20, 40),
                                   mtries = c(5, 6, 7),
                                   balance_classes = c(TRUE, FALSE),
                                   sample_rate = c(0.5, 0.632, 0.95),
                                   col_sample_rate_per_tree = c(0.5, 0.9, 1.0),
                                   histogram_type = c("UniformAdaptive",
                                                      "Random",
                                                      "QuantilesGlobal",
                                                      "RoundRobin")),
               search_criteria = list(strategy = "RandomDiscrete", 
                                      max_models = 100, 
                                      stopping_metric = "auc", 
                                      stopping_tolerance = 0.01, 
                                      stopping_rounds = 5, 
                                      seed = 26227709)) 

rf.grid <- h2o.getGrid(grid_id = "gridrf07",
                       sort_by = "auc",
                       decreasing = TRUE)
rf2 <- h2o.getModel(rf.grid@model_ids[[1]])
h2o.saveModel(rf2, path = "/root/Documents/mk/")
summary(rf2)
varimp <- as.data.frame(h2o.varimp(rf2))
h2o.varimp_plot(rf2)
h2o.performance(rf2, newdata = test)

## ETHONSET == 1
df.eth2a <- as.h2o(df.eth2)

df.eth2a$uamkstart <- as.factor(df.eth2a$uamkstart)  #encode the binary repsonse as a factor
h2o.levels(df.eth2a$uamkstart)

# Partition the data into training, validation and test sets
splits <- h2o.splitFrame(data = df.eth2a, 
                         ratios = c(0.7, 0.15),  # 70%, 15%, 15%
                         seed = 42)  # reproducibility


train <- h2o.assign(splits[[1]], "train.hex")   
valid <- h2o.assign(splits[[2]], "valid.hex") 
test <- h2o.assign(splits[[3]], "test.hex")

y <- "uamkstart"
x <- setdiff(names(df.eth2), c(y, "ccode", "year", "rgdppc",
                           "uamkyr2", "uamkyr3", "sf", "country",
                           "elf2", "polity2sq")) 

# Running the model
rf <- h2o.grid("randomForest", x = x, y = y, training_frame = train, 
               validation_frame = valid, nfolds = 5, grid_id = "gridrf08",
               fold_assignment = "Stratified",
               hyper_params = list(ntrees = c(256, 512, 1024),
                                   max_depth = c(10, 20, 40),
                                   mtries = c(5, 6, 7),
                                   balance_classes = c(TRUE, FALSE),
                                   sample_rate = c(0.5, 0.632, 0.95),
                                   col_sample_rate_per_tree = c(0.5, 0.9, 1.0),
                                   histogram_type = c("UniformAdaptive",
                                                      "Random",
                                                      "QuantilesGlobal",
                                                      "RoundRobin")),
               search_criteria = list(strategy = "RandomDiscrete", 
                                      max_models = 100, 
                                      stopping_metric = "auc", 
                                      stopping_tolerance = 0.01, 
                                      stopping_rounds = 5, 
                                      seed = 26227709)) 

rf.grid <- h2o.getGrid(grid_id = "gridrf08",
                       sort_by = "auc",
                       decreasing = TRUE)
rf2 <- h2o.getModel(rf.grid@model_ids[[1]])
h2o.saveModel(rf2, path = "/root/Documents/mk/")
summary(rf2)
varimp <- as.data.frame(h2o.varimp(rf2))
h2o.varimp_plot(rf2)
h2o.performance(rf2, newdata = test)
     
\end{verbatim}


\doublespacing
\normalsize

%%%%%%%%%%%%%%%%%%%%%%%%%%%%
% BIBLIOGRAPHY
\clearpage
\phantomsection
\addcontentsline{toc}{chapter}{Bibliography}
\bibliography{references/bibliography}
%%%%%%%%%%%%%%%%%%%%%%%%%%%%

%%%%%%%%%%%%%%%%%%%%%%%%%%%%
% START APPENDICES
\appendix
%%%%%%%%%%%%%%%%%%%%%%%%%%%%

%%%%%%%%%%%%%%%%%%%%%%%%%%%%
% END DOCUMENT
\end{document}
%%%%%%%%%%%%%%%%%%%%%%%%%%%%