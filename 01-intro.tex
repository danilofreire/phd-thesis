\chapter{Introduction}
\label{chap:intro}

The literature on political and criminal violence has increased exponentially over the last decades. Although interstate wars have traditionally occupied a privileged position in this research agenda, scholars have broadened their analyses and considered a myriad of hitherto understudied phenomena. Civil wars \citep[][]{collier2004greed,fearon2003ethnicity,kalyvas2006logic}, genocides \citep[][]{mamdani2014victims, power2013problem}, ethnic conflicts \citep[][]{kaufmann1996possible, montalvo2005ethnic, sambanis2001ethnic}, wartime sexual abuse \citep[][]{cohen2013explaining,wood2006variation,wood2009armed}, electoral violence \citep[][]{hoglund2009electoral,wilkinson2006votes}, state-sponsored killings \citep[][]{harff1988toward, krain1997state,krain2005international,uzonyi2014unpacking}, terrorism \citep[][]{de2005quality,bueno2007propaganda,pape2003strategic}, drug-related violence \citep[][]{holmes2006drugs,lessing2015logics,richani2013systems, shirk2010drug}, street gangs \citep[][]{franzese2016youth,jones2009youth,rodgers2006living,sobel1987direct}, and prison gangs \citep[][]{dias2011pulverizaccao,freire2014,skarbek2011governance,skarbek2012prison,skarbek2014social} have moved to the centre stage of the discipline. This dissertation contributes to this growing literature.

The following chapters discuss different types of organised violence. In order to clarify specific aspects of my research questions, I employ an eclectic combination of research designs, ranging from statistical analysis to field experiments and qualitative case studies. This diversity of methods not only reflects the multiple angles by which violence can be analysed, but it is also a pragmatic response to problems which are common to researchers in this area, such as incomplete data, biased information, and the practical impossibility of random assignment of treatment. By using a number of methodological tools, I hope to overcome some of these difficulties and offer compelling evidence for the results presented below.

Regarding the geographical scope of this dissertation, most of the essays deals with issues of violence in Latin America, specially in Brazil. According to the World Bank, Latin America is home to about 8\% of the global population, yet it accounts for more than 30\% of the world's homicides.\footnote{See: \url{https://goo.gl/d2WC3V}. Access: April 2017.} Moreover, the yearly ranking by the Citizen's Council for Public Security and Criminal Justice (Consejo Ciudadano para la Seguridad Pública y la Justicia Penal), a Mexican non-governmental organisation, shows that 43 of the 50 most violent cities in the world are located in Latin America, including all in the top 10.\footnote{For the complete ranking, see \url{http://www.seguridadjusticiaypaz.org.mx/biblioteca/prensa/summary/6-prensa/239-las-50-ciudades-mas-violentas-del-mundo-2016-metodologia}. Access: April 2017.} Given the severity of violence in the continent, Latin America was a natural candidate to be the primary focus of this study.

Brazil provides a major example of the challenges of fighting violence in Latin America. Brazil has the highest absolute number of homicides in the world, about 56,000 per year, and the country hosts 19 of the 50 world's deadliest cities according to the above-mentioned ranking \citep{mapa2014, mexico2014,unodc2013}. Homicide rates have increased markedly after democratisation (1985), and whereas the country has tried several policies to reduce violence, the results are yet to be evaluated in a consistent fashion. 

The first chapter attempts to address that issue. Although Brazil remains notably affected by civil violence, the state of São Paulo has made significant inroads into fighting criminality. In the last decade, São Paulo has witnessed a 70\% decline in homicide rates, a result that policy-makers attribute to a series of crime-reducing measures implemented by the state government \citep{goertzel2009,kahn2005papel}. While recent academic studies seem to confirm this downward trend, no estimation of the total impact of state policies on homicide rates currently exists. I fill this gap by employing the synthetic control method \citep{abadie2003,abadie2010,abadie2014}, a generalisation of differences-in-differences \citep{angrist2008mostly,bertrand2004much,imbens2009recent}, to compare these measures against an artificial São Paulo. The results indicate a large drop in homicide rates in actual São Paulo when contrasted with the synthetic counterfactual, with about 20,000 lives saved during the period. The theoretical usefulness of the synthetic control method for public policy analysis, the role of the \textit{Primeiro Comando da Capital}, a local prison gang, as a moderating variable, and the practical implications of the security measures taken by the São Paulo state government are also discussed.

The second chapter offers a rational choice account for the \textit{jogo do bicho}, the animal game, possibly the largest illegal gambling game in the world. The lottery has been running for over 120 years and according to estimations of Fundação Getúlio Vargas, a Brazilian think tank, it profits up to 800 million dollars per year.\footnote{See \url{http://goo.gl/9kNeX8} and \url{http://goo.gl/8FSAZl} (in Portuguese). Access: April 2017} The \emph{jogo do bicho} has exerted a significant impact on the Brazilian society. The lottery has been a major sponsor of the Carnival Parade in Rio de Janeiro, which is among the world's most famous popular festivals, and it has remained an important driver of state corruption in the country \citep{bezerra2009mecenato,chazkel2011laws,da1999aguias,labronici2012paratodos,magalhaes2005ganhou,soares1993jogo}. I investigate the institutions that have caused the \emph{jogo do bicho}'s notable growth and long-term survival outside the boundaries of the Brazilian law. I show how \textit{bicheiros} or bookmakers promote social order, solve information asymmetries, and reduce negative externalities via costly signalling and the provision of club goods. I also explain the emergence of the informal rules that govern the game as well as their enforcement mechanisms.

