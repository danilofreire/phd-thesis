\documentclass[a4paper,12pt]{article}
\usepackage[utf8]{inputenc}
\usepackage{microtype}
\usepackage{libertine}
\usepackage[libertine]{newtxmath}
\usepackage[scaled=.9]{inconsolata}
\usepackage[usenames,dvipsnames]{xcolor}
\definecolor{darkblue}{rgb}{0.0,0.0,0.55}
\usepackage{setspace}
\usepackage[top=2cm,bottom=2cm,left=2cm,right=2cm]{geometry}
\usepackage[backref,pagebackref]{hyperref}
\usepackage{graphicx}
\usepackage{float}
\usepackage{mathtools}
\usepackage{caption}
\usepackage[UKenglish]{babel}
\usepackage[UKenglish]{isodate}
\cleanlookdateon
\usepackage[authoryear]{natbib}
\usepackage{babelbib}
\exhyphenpenalty=1000
\hyphenpenalty=1000
\widowpenalty=1000
\clubpenalty=1000
\renewcommand*{\backref}[1]{}
\renewcommand*{\backrefalt}[4]{%
	\ifcase #1 (Not cited.)%
	\or        Cited on page~#2.%
	\else      Cited on pages~#2.%
	\fi}
\renewcommand{\backreftwosep}{ and~}
\renewcommand{\backreflastsep}{ and~}
\urlstyle{same}  % don't use monospace font for urls
\setcitestyle{aysep={}} 
\usepackage{etoolbox}
\makeatletter
\patchcmd{\NAT@citex}
	  {\@citea\NAT@hyper@{%
		 \NAT@nmfmt{\NAT@nm}%
		 \hyper@natlinkbreak{\NAT@aysep\NAT@spacechar}{\@citeb\@extra@b@citeb}%
		 \NAT@date}}
	  {\@citea\NAT@nmfmt{\NAT@nm}%
	   \NAT@aysep\NAT@spacechar\NAT@hyper@{\NAT@date}}{}{}
	\patchcmd{\NAT@citex}
	  {\@citea\NAT@hyper@{%
		 \NAT@nmfmt{\NAT@nm}%
		 \hyper@natlinkbreak{\NAT@spacechar\NAT@@open\if*#1*\else#1\NAT@spacechar\fi}%
		   {\@citeb\@extra@b@citeb}%
		 \NAT@date}}
	  {\@citea\NAT@nmfmt{\NAT@nm}%
	   \NAT@spacechar\NAT@@open\if*#1*\else#1\NAT@spacechar\fi\NAT@hyper@{\NAT@date}}
	  {}{}
\makeatother
\hypersetup{pdftitle={Beasts of Prey or Rational Animals? Private Governance in Brazil's Jogo do bicho},
	pdfauthor={Danilo Freire},
	pdfborder={0 0 0},
	breaklinks=true,
	linkcolor=Mahogany,
	citecolor=Mahogany,
	urlcolor=darkblue,
	colorlinks=true}

\doublespacing

\title{\textbf{Beasts of Prey or Rational Animals?\\ Private Governance in Brazil's \textit{Jogo do Bicho}}}

\author{Danilo Freire\thanks{Postdoctoral Research Associate, Political Theory Project, Brown University, 8 Fones Alley, Providence, RI 02912, \href{mailto:danilofreire@brown.edu}{\texttt{danilofreire@brown.edu}}, \href{http://danilofreire.com}{\texttt{http://danilofreire.com}}. I would like to thank Toke Aidt, Paulo Roberto Araujo, Jose Bolaños, Diogo Costa, Guilherme Duarte, Robert McDonnell, David Skarbek, Graham Denyer Willis, the participants of academic seminars at King's College London, the University of Cambridge and the Prometheus Institut in Berlin, the editors of Latin American Perspectives, and four reviewers for their valuable comments. I gratefully acknowledge the financial support of the National Council for Scientific and Technological Development.}
}

\date{\today}

\begin{document}
\maketitle

\begin{abstract}
\noindent
This article presents a rational choice account of Brazil's \textit{jogo do bicho} (`animal game'), possibly the largest illegal lottery game in the world. Over 120 years, the \textit{jogo do bicho} has grown from a local raffle to a multimillion-dollar business, and it has played a major role in Brazil's cultural and political life. My analysis has three goals. First, I examine the mechanisms that fostered the \textit{jogo do bicho}'s notable growth outside of the boundaries of the law. Second, I investigate how the lottery financiers combine costly signals and selective incentives to induce cooperation from members of community. Lastly, I discuss the relationship between the lottery sponsors and Brazilian representatives, particularly how the \textit{bicheiros} exploit the fragmentation of the Brazilian political system to advance their long-term goals. 

 \vspace{.5cm}
 \noindent
 \textbf{Keywords}: Brazil; criminal organisations; corruption; \textit{jogo do bicho}; private governance
  
 \vspace{.25cm}
 \noindent
 \textbf{JEL Classification Codes}: D72, K42, Z00
\end{abstract}

\newpage

\section{Introduction}
\label{sec:intro}

In 1892, Baron João Batista de Viana Drummond came up with a new idea to fund his cash-strapped zoo. Situated in a quiet neighbourhood in the north of Rio de Janeiro, the \textit{Jardim Zoológico}, or Zoological Garden, hosted a variety of exotic species and offered breath-taking views of the city. But it lacked visitors. An experienced businessman, Drummond realised the zoo would have to provide other kinds of entertainment to keep itself afloat. One of his plans seemed particularly promising: a lottery raffle.

The rules were simple. In the morning, the Baron would choose one animal from a list of 25 beasts and put its picture inside a wooden box at the zoo's entrance. Visitors who wanted to join the raffle received a ticket bearing the stamp of one of those 25 animals.\footnote{At first, the zoo staff distributed the tickets at random, but they soon allowed participants to choose the animals they preferred. This small change made the game considerably more appealing to the public, who often resorted to clairvoyants and fortune tellers to guess the winning animals \citep[71--74]{da1999aguias}.} At five in the afternoon, Drummond opened the box, showed the picture to the public, and paid to every winner a cash prize worth 20 times the zoo's admission fee. The Baron called the lottery the \textit{jogo do bicho}, or the animal game, and it was well-received by the public. Eager to capitalise on that initial success, Drummond stated that visitors could buy tickets not only at the zoo, but also in many stores across Rio de Janeiro. What was once a small raffle soon became a large gambling market of its own. 

A \textit{jogo do bicho} craze swept the whole city after independent sellers entered the marketplace. A network of street bookmakers, called \textit{bicheiros}, made the lottery available in every part of Rio by scalping tickets or promoting their own versions of the numbers game \citep[37]{chazkel2011laws}. The lottery became so widespread that Olavo Bilac, a major literary figure in nineteenth-century Brazil, summarised the situation as follows: `Today $[1985]$ in Rio de Janeiro, the game is everything. $[\dots]$  Nobody works! Everybody plays' \citep[43]{pacheco1957antologia}.\footnote{Unless otherwise noted, all translations from the Portuguese are my own.} 

But this tolerant state of affairs did not last. Civil servants and police officers criminalised the \textit{jogo do bicho} on the grounds of `public safety', and in the late 1890s they launched a country-wide campaign against the lottery \citep{benatte2002jogos}. The campaign extended for several decades and received considerable support from the \textit{Companhia das Loterias Nacionaes do Brazil}, the National Lottery Company, a public-private partnership founded four years after, and perhaps motivated by, the creation of the animal game \citep[82]{da1999aguias}. The Brazilian government officially banned the \textit{jogo do bicho} in 1946 and it remains illegal until this day.  
 
Yet the game has survived. The \textit{jogo do bicho} has outlasted more than 30 Brazilian presidents and thrived under military regimes and democratic governments alike \citep{jupiara2015poroes}. But more than a act of defiance, the \textit{jogo do bicho} is a successful capitalist enterprise \citep{labronici2014sorteio, magalhaes2005ganhou}. A recent study by Fundação Getúlio Vargas, a Brazilian think tank, affirmed that the \textit{jogo do bicho} earns from BRL 1.3 to BRL 2.8 billion per year (USD 400 to USD 850 million), making it the largest clandestine gambling game in the world.\footnote{See \href{https://www.huffpostbrasil.com/2015/12/07/legalizacao-jogos-brasil_n_8737212.html}{\texttt{https://www.huffpostbrasil.com/2015/12/07/legalizacao-jogos-brasil\_n\_8737212.html}} (in Portuguese). Access: August 2018.} \citet[171]{schneider1996brazil} estimated that in the 1990s, the game furnished about 50,000 jobs in the Rio de Janeiro city alone, almost the same number of employees that the oil giant Petrobras had in 2011 \citep{exame2013petrobras}.\footnote{In 1966, Time Magazine wrote that the \textit{jogo do bicho} was `the largest single industry in Latin America' and employed about 1\% of the Brazilian workforce. See \href{http://content.time.com/time/magazine/article/0,9171,842527-1,00.html}{\texttt{http://content.time.com/time/magazine/article/0,9171,842527-1,00.html}}. Access: August 2018.}

In this article I offer a rational choice interpretation of the \textit{jogo do bicho}. More specifically, I use an array of bibliographical sources to show how the game operators, called \textit{bicheiros}, have developed unique strategies to solve collective action problems and maximise their political strategies. The tactics are unusual and often repressive, such as bribing corrupt police officers to threaten undisciplined employees. Nevertheless, I argue here that such strategies are effective, and while they seem counter-intuitive at first, they do address the long-term needs of the \textit{jogo do bicho} financiers. 

Like any business manager, \textit{bicheiros} have to run their firm with low costs to increase profits. However, the fact that the \textit{jogo do bicho} is clandestine imposes additional difficulties for its operation. In particular,  \textit{bicheiros} face two main challenges to keep the lottery running. First, they need to gather public support so that gamblers are not discouraged to engage in the lottery despite it being illegal. Second, \textit{bicheiros} also have to ensure that the state repression is not prohibitively costly to their business, otherwise they would be better off by shutting it down. I argue below that the \textit{bicheiros} have succeeded in both by using carefully-designed reputation strategies and employing costly signals to the communities they serve. 

I use the case of Rio de Janeiro to illustrate how the \textit{jogo do bicho} has overcome the obstacles to its expansion. Rio is a particularly interesting case because in no other part of Brazil the game financiers established such an effective patronage network. \textit{Bicheiros} have sponsored political campaigns, financed cultural activities and football teams, and sometimes even run in local elections themselves. I discuss the ways by which the \textit{bicheiros} have exploited fragilities of the Brazilian political system to their advantage and how those practices have weakened Brazilian democracy.   

My analysis discusses three strands of academic literature. First, this work contributes to the scholarship on extra-legal institutions, mainly to the literature on collective action within criminal organisations. For instance, \citet{gambetta1996sicilian} examines the strategies used by the Sicilian Mafia to settle disputes among their members and enforce rules in the areas they exercise control. \citet{leeson2009invisible,leeson2010pirational} affirms that pirate groups employed hard-to-fake signals to increase the profitability of their operations. \citet{skarbek2011governance,skarbek2012prison,skarbek2014social}, in turn, highlights the role of written and implicit norms in mitigating rent-seeking and coordinating productive activities in California prison gangs. I argue that \textit{bicheiros} have employed reputation strategies and provided club goods to enforce private contracts and foster trust in the community. 

Second, this work relates to the literature on signalling theory and asymmetric information \citep[e.g.,][]{akerlof1970market,connelly2011signaling,spence1973job}. I provide evidence that \textit{bicheiros} were aware of their social stigma, and as a response they devised signalling strategies to convey reliable information and reduce the uncertainty associated with clandestine markets. Their main tool to increase credibility was costly signalling \citep{gambetta2009codes,kimbrough2015commitment, schelling1960strategy}. \textit{Bicheiros} believed that by sacrificing their immediate interests they could gain a reputation of honesty that would benefit them in the long run.  
 
Lastly, this work connects to the literature on state capture, which is one of the most important topics in public choice theory \citep{rose1978corruption,shleifer2002grabbing,tollison1982rent}. More specifically, I use the Brazilian case to illustrate how politicians and civil servants can be co-opted by criminal groups and produce sub-optimal social outcomes. \citet{queiroz1992carnaval} explored why \textit{bicheiros} turned into patrons of the Carnival's samba schools and affirmed that this influence gave them leverage over political authorities. \citet{misse2007illegal} investigated the links between bicheiros and police officers, and suggested that the illegal lottery had been the main cause of police corruption in Rio de Janeiro until the 1970s. In a similar vein, \citet{jupiara2015poroes} analyse the relationship between the \textit{jogo do bicho} and the military regime in Brazil (1964--1985). I supplement this literature by highlighting how asymmetrical information, agency dilemmas, and rent-seeking behaviour offer convincing explanations to the issues presented above. Although those concepts have a long tradition in public choice, scholars have not applied those ideas thus far to understand the dynamics of the \textit{jogo do bicho}. By doing so, I integrate seemingly contradictory historical facts into a single narrative that connects micro-level decisions to macro-level outcomes. 

\section{An Overview of the \textit{Jogo do Bicho}}
\label{sec:overview}

\subsection{Historical Background: How the \textit{Bicheiros} Avoided Extinction}
\label{sub:historical_background}

The late nineteenth-century Brazil had four characteristics that explain the emergence of the \textit{jogo do bicho}: 1) a growing urban population excluded from the formal labour market; 2) an inflow of immigrants whose extended family networks helped them engage in trade; 3) an expansion of the monetary supply in the first years of the republic (1880s--1890s); and 4) a judicial system that, albeit repressive, had only imperfect law enforcement. I discuss each of these elements below.

I start with the impact of urban poverty on the animal game. Brazil abolished slavery in the late 1880s, a period in which the country was rapidly urbanising. Brazil's growing cities offered new occupations for former slaves who desired to move away from their former masters \citep{andrews1991blacks, skidmore1993black}. Increasing numbers of Asian and European immigrants joined the freed slaves and moved to the cities soon after they arrived in Brazil \citep{hall1969origins, lesser2013immigration}. However, the hopes of the African-Brazilians and the new foreign settlers would soon be frustrated by a series of economic downturns. The Brazilian labour market suffered a severe contraction in the wake of the \textit{Encilhamento} financial crisis of 1891, and the economic instability aggravated the already difficult conditions of the working classes \citep{topik2014political, triner2005baring}.

In that regard, large swathes of the urban population turned to the informal economy. As \citet[115]{chazkel2011laws} observes, there were few occupations available to lower-class women and foreigners in the 1890s, and a large number of poor workers became street vendors. The profession requires little technical skills and has low barriers of entry, but it can be very profitable if for whatever reason there is a strong demand for particular product. The \textit{jogo do bicho} was one of those products that offered a high rate of return. The game was simple to operate and that simplicity attracted more people willing to try their luck. As the game gained a following in Brazil's First Republic, it comprised an important share of the extra-legal economy in Rio de Janeiro.

Immigration also influenced the \textit{jogo do bicho} via social ties. Most foreigners who moved to Brazil came from countries, such as Portugal, Spain or Italy, where extended families were the basic form of social organisation \citep{lobo2001imigraccao, trento1989outro}. Family and neighbourhood networks created incentives for immigrants to establish trade relations and enforce cooperation through community responsibility systems \citep{roth2014prison}. Because of these particular social characteristics, in the 1890s foreigners were over-represented in the Brazilian trade in general \citep{mattos1991vadios, oliveira2001brasil} and in the \textit{jogo do bicho} in particular \citep{magalhaes2005ganhou, villar2008contravencao}. Although kinship bonds became less relevant over time, these links offered an important element of social cohesion in the \textit{jogo do bicho}'s formative years.

Next is the impact of expanded monetary supply. The abolition of slavery and the growing industrialisation of Brazil increased the amount of capital available in the country \citep{franco1987reformas, schulz2008financial}. Moreover, the 1888 Banking Act gave extra liquidity to local financial markets, what made credit more widely available in cities like São Paulo and Rio de Janeiro. Individuals received a temporary boost in personal income, a part of which they spent on leisure activities such as the \textit{jogo do bicho}. Moreover, the lottery attracted new entrants as it became more profitable, and in only a few years similar versions of the animal game were available throughout Brazil \citep[79]{da1999aguias}.  

The last necessary condition for the emergence of the \textit{jogo do bicho} is weak law enforcement. \citet[69--100]{chazkel2011laws} notes that until the 1940s police district chiefs operated within a large margin of discretion and repression against bookmakers was idiosyncratic. In the early years of \textit{jogo do bicho}, lottery `bankers' were allowed to operate virtually free from police interference, what surely collaborated to the game's rapid initial expansion \citep[544]{chazkel2007beyond}. Prosecution against the \textit{bicheiros} hardened in 1917 after the promulgation of the Civil Code and in 1941 the animal game was banned.\footnote{See: \href{http://www.planalto.gov.br/ccivil_03/decreto-lei/Del3688.htm}{\texttt{http://www.planalto.gov.br/ccivil\_03/decreto-lei/Del3688.htm}} (in Portuguese). Access: September 2017.} Five years later, the federal government declared that all games of chance were illegal in Brazil.\footnote{The 1946 decree stated that gambling was `harmful to morality and the good customs', hence `[\dots] the repression against games of chance [was] an imperative of the universal consciousness'. The text can be read at: \href{http://www.planalto.gov.br/ccivil_03/decreto-lei/Del9215.htm}{\texttt{http://www.planalto.gov.br/ccivil\_03/decreto-lei/Del9215.htm}} (in Portuguese). Access: September 2018.} Recent estimations show that the prohibition of the \textit{jogo do bicho} have prevented the state from earning BRL 15 to BRL 20 billion (USD 4.5 to USD 6 billion) per year in expected taxation revenues, aside from the subjective utility losses for players. \citep{fsp2016legalizarbicho}.

Since the mid-twentieth century, the \textit{jogo do bicho} has been both iniquitous and ubiquitous in Brazil. For all purposes, the \textit{jogo do bicho} remains illegal, yet there is almost no Brazilian city which does not have its own local \textit{bicheiros}. In that regard, the \textit{jogo} does not operate exclusively at the margins of the Brazilian law; given its size, it is clear that the \textit{jogo} consciously exploits and subsidises large sectors of the formal economy. One good example is the relationship between the \textit{jogo} and the informal workforce. The \textit{jogo do bicho} gives a boost to the Brazilian economy by providing jobs for unskilled workers who cannot easily join the labour market. By doing so, the \textit{jogo} indirectly prevents some of poorest members of the Brazilian society from demanding more inclusive government policies, although they are those who would benefit the most from public assistance. Because of the income generated by the \textit{jogo do bicho}, poor workers are able to consume without having to resort to government funding, which can be spent with other, probably wealthier, sectors of the population. I describe the animal game organisation structure and their impact in the Brazilian economy in further detail below.  

\subsection{A Hierarchical Organisational Structure}
\label{sub:organisation}

The animal game operates with three levels of hierarchy. At the bottom level are the \textit{bicheiros}, those in charge of selling \textit{jogo do bicho} tickets \citep{chazkel2007beyond, da1999aguias}. \textit{Bicheiros} are the most visible part of the \textit{jogo do bicho}, and the name loosely describes all those involved in the lottery organisation. Yet their meaning in the \textit{jogo do bicho} structure is more particular and refers to street-level ticker sellers. The \textit{bicheiros}  usually build their vending stands inside the premises of a local shop, such as a small grocery store, and are recognisable by their chairs facing the street, stamps and blocks of paper \citep[259]{chazkel2011laws}. The street bookmakers usually work alone, but may employ up to 10 people depending on how busy their betting site is \citep[69]{labronici2014sorteio}.

The \textit{gerentes} (managers) oversee all \textit{jogo do bicho} stands in a given area. Their task is akin to that of a firm accountant. Gerentes control the cash flow between the \textit{bicheiros} and the bankers, manage the payroll of the employees, and provide financial information to the top members of the organisation. They also supervise individuals who carry menial tasks in the business, transfer money to other gambling branches and double-check the balance sheets of the betting sites \citetext{\citealp[71]{labronici2012paratodos}; \citealp[142]{misse2007illegal}}.

The \textit{banqueiros}, or the Portuguese for bankers, occupy the top position in the \textit{jogo do bicho} hierarchy. They comprise the small financial elite of the game. A 2012 report by the Brazilian Federal Police affirmed that 10 \textit{banqueiros} controlled the market throughout the country; five of them based in the state of Rio de Janeiro \citep{globo2012contraventores}. Apart from funding the game, the bankers provide support for the employees to undertake their activities. The \textit{banqueiros}' main attributions include paying bribes to police personnel, bailing out sellers arrested by security forces, and offering judicial assistance to employees in case of legal persecution \citep[75]{labronici2012paratodos}.

\textit{Banqueiros} run their businesses from fortified houses in unknown locations, the \textit{fortalezas} (`forts'). The first \textit{fortalezas} likely appeared in the 1950s, when the animal game was already well-established across the Brazilian territory. The period coincides with a time when the \textit{jogo do bicho} finances had become increasingly concentrated in fewer hands \citep[259]{chazkel2011laws}. Due to the growing scope of the \textit{jogo do bicho} economy, \textit{banqueiros} decided to move their operations away from the public to avoid police persecution and reduce coordination costs.

\textit{Banqueiros} solve problems of internal cooperation by providing club goods \citep{buchanan1965economic} while simultaneously shunning cheaters through selective punishments \citep{bo2005cooperation, roth1978equilibrium}. The first club good offered to \textit{bicheiros} by their bosses is private security. As the game is illegal, street sellers cannot rely on official institutions to protect themselves. Thus, the game bankers have built an extensive network of gunmen and bribed police officers to protect their employees from other criminals \citetext{\citealp[48]{chinelli1993vazio}; \citealp[51]{labronici2012paratodos}}. `Zé' (Little Joe), a bicheiro interviewed by \citet[52]{labronici2012paratodos}, described eloquently the deterring effect of the \emph{jogo do bicho} informal security personnel:

\begin{quote}
 [\dots] bums are scared and they don't mess around with us; they think there's a guard nearby or something like that. Look at all this money here! [shows the interviewer a handful of cash] It's not ours [referring to street-corner bookmakers]. And if it's not ours, it's someone else's. When I worked in Penha (\emph{a low middle-class neighbourhood in the city of Rio de Janeiro -- translator's note}), the owner of a pub close to where I used to work always asked me to stay at the front door of his pub. People know that bums are afraid of \emph{bicheiros}.
\end{quote}

But the \textit{banqueiros} do not use violence only against other criminals. They often employ violent methods against competitors and their own staff, too. \citet{jupiara2015poroes} argue that Ailton Guimarães Jorge, a former Army officer, tortured and murdered rival lottery bosses in the late 1970s. One of his former allies, Army Colonel Paulo Malhães, told to the Rio de Janeiro State Truth Commission that Guimarães `went on a rampage' to consolidate his power \citep{belem2015guimaraes}. Castor de Andrade, Rio's most influential animal game \textit{banqueiro}, also employed similar methods to run his business. Andrade kept an armed bodyguard of 23 men and allegedly murdered a number of competitors. In a famous case, Andrade shot Euclides Ponar, an old \textit{jogo do bicho} boss known as `Grey-Headed Chinese',\footnote{In Portuguese, \textit{China Cabe\c{c}a Branca}.} after Ponar denounced a fraud in lottery draws in 1976. Andrade directly planned the assassination of at least other four competitors in the 1980s \citep{globo2017castor}. 

Despite these serious events, killings are rare in the animal game. Since the lottery bosses can credibly indicate that violence is a low-cost option for them, the mere threat of punishment is enough to induce cooperation. This is a good strategy for the \textit{banqueiros}. Interestingly, the fact that they had committed violent crimes in the past reduces the need to commit them in the present, as and a result they can spend less money on security and increase profits. As it happens in many traditional markets, if a group is able to form a cartel, they can increase the price of their services without fearing immediate competition. The same logic is valid for the \textit{jogo do bicho}, although the means it employs are unconventional.  

The threat of violence is not the only means the \textit{bicheiros} have at hand. They balance it  with financial benefits to low-rank members of the organisation. For instance, street \emph{bicheiros} keep all tips they receive from players, often have small expenses covered by their bosses, and may request interest-free loans to pay for healthcare treatment or other unexpected bills \citep{labronici2012paratodos}.

The most important financial mechanism implemented by bankers to help \emph{bicheiros} is the \emph{descarga}, loosely translated as `the unloading'. The descarga is the \emph{jogo do bicho}'s main hedging technique and its purpose is to insure small bookmakers against credit risk \citetext{\citealp[59]{labronici2012paratodos}; \citealp[178]{magalhaes2005ganhou}; }; \citealp[75]{soares1993jogo}}. Booking agents are sometimes unable to honour expensive bets. The top prize in the animal game pays up to 4,000 times the amount invested, thus \emph{bicheiros} may have to raise thousands of Brazilian Reals in a single day to pay the lucky winners. To prevent the \emph{quebra da banca} (`bust of the bank'), \emph{bicheiros} and small bankers buy an insurance from wealthier financiers, who offer this service for a fee that ranges from 20\% to 25\% of the total selling amount \citep{fsp2006descarga}. The \emph{descarga} guarantees that small bookmakers will not have liquidity problems, thus permitting bookmakers to continue investing in the \emph{jogo do bicho}.

The descarga has significantly changed the distribution of resources in the \emph{jogo do bicho}, and the richest bankers benefited the most from it. Simple probability dictates that a booking agent rarely pays the highest lottery prize, and yet the bankers receive a commission for \emph{every game} they hedge. Over time, there is a transfer of income from the bottom to the top of the animal game structure due to the fees. This accumulation of capital is probably the reason why in the 1990s bankers started offering other types of entertainment such as slot machines and sports lotteries \citep{estado2006cacaniquel,globo2015cacaniquel,terra2011cacaniquel}. They simply had more capital to invest. In sum, while the descarga has made the game more resilient at the aggregated level, it increased profits for the richest financiers at the expense of small bookmakers.

By combining both carrots and sticks, the \textit{jogo do bicho} bosses are able to effectively coordinate their employees. A more difficult question, however, is how the bankers elicit cooperation from \textit{external} members, such as gamblers, community leaders, or public officers. It is puzzling because \textit{bicheiros} do not use violence to induce individuals to play the lottery, nor have they ever clashed with the Brazilian government. Precisely because violence could shun off profits, \textit{bicheiros} devised other mechanisms to create a friendly environment for the illegal lottery. In the next section, I investigate two of them, costly signalling and reputation building. 

\section{External Cooperation}%
\label{sec:external}

Evolutionary game theory \citep{axelrod1984evolution, axelrod1985achieving, smith1982evolution} and empirical studies \citep{isaac1984divergent, ostrom1990governing} have both demonstrated that long-term cooperation is possible in a great number of situations. The main requirement for sustained cooperation is that players should believe that future pay-offs will be higher than present ones. If that condition is true, fear of retaliation will induce individuals not to cheat. 

In theory, the same should also apply to illegal organisations. Yet in practice we see that criminal groups tend to be short-term oriented, or, in more technical terms, they tend to discount the future more heavily that most people. This makes cooperative behaviour rather uncommon among criminal organisations, and there is considerable evidence suggesting that illegal groups have considerable difficulty to maintain themselves in the long run.  

The \emph{jogo do bicho} is an exception to this rule. The game has been running for more than a century with no considerable interruption and with moderately low levels of violence -- at least when compared to other illegal activities such as drug trafficking. Moreover, 

 Here I analyse two means by which the \textit{bicheiros} elicit voluntary cooperation from gamblers and members of the community, costly signalling and reputation strategies. In the following section I discuss how the \textit{jogo} colludes with corrupt politicians and civil servants for mutual benefit.   

\subsection{Winning Hearts, Minds, and Pockets: Market Dynamics}%
\label{sub:pockets}

The \emph{jogo do bicho} entrepreneurs have made considerable efforts to present themselves as honest brokers. The first trust-enhancing mechanism they have employed to foster external cooperation was the use of a \emph{fixed-multiplier formula} for pay-outs. It works as follows. If a player wins the lowest prize of the animal game, he or she receives 18 times his/her investment regardless of the size of the bet. Bigger prizes naturally offer higher returns; a lucky winner of the top prize wins up to 4,000 times the value of his/her bet \citetext{\citealp[89]{labronici2012paratodos}; \citealp[20]{magalhaes2005ganhou}}.

This stands in sharp contrast to the common practice of sharing a prize among winners. Lottery pay-outs demand high levels of interpersonal trust: players rely on unverifiable information about the total funds collected by the lottery, and they can never be sure whether the payments are evenly distributed. The fixed-multiplier formula alleviates such problems of adverse selection \citep{akerlof1970market, cohen2010testing, levin2001information}. As players and vendors known the prize value beforehand, the method provides consumers with complete information about their individual prizes while also binding the \emph{bicheiros} to a contract that can be easily enforced. This technique offers buyers a simple yet effective screening strategy that induces \emph{bicheiros} to provide honest information about the game \citep{spence1973job, stiglitz1981credit}.

\emph{Bicheiros} have addressed information asymmetries in another ways. Since the 1950s, when the \emph{jogo do bicho} bankers had moved their operations to the \emph{fortalezas}, the public could not oversee the lottery draws \citep[259]{chazkel2011laws}. This could lead to a decline in trust among buyers and vendors of lottery tickets and, as a result, to reduced profits. \emph{Bicheiros} have mitigated this problem with a two-pronged strategy. First, they started to utilise the winning numbers from the licit government-run lottery, the \emph{Loteria Federal}, instead of their own draws \citetext{\citealp[546]{chazkel2007beyond}; \citealp[89]{labronici2012paratodos}; \citealp[39-40]{mello1989historia}}. The federal lottery numbers are public information. The media broadcasts the draws on radio and TV, so any interested player can verify the selected numbers. The Loteria Federal is also audited by two independent state institutions, a private accounting firm, and voluntary members of the public; hence, \emph{bicheiros} can free ride on the lottery's long-standing reputation of credibility.\footnote{As of April 2016, the lottery was audited by the \emph{Controladoria Geral da União} (Comptroller General of Brazil), the \emph{Tribunal de Contas da União} (General Accounting Office), and by Ernst \& Young. The balls are measured and weighted every three months by the National Institute of Metrology, Quality and Technology (Inmetro), the Brazilian equivalent of United Kingdom's National Physical Laboratory or the American National Standards Institute. See \url{http://noticias.uol.com.br/cotidiano/ultimas-noticias/2016/04/08/auditoria-dos-sorteios-da-caixa-e-confiavel-veja-como-e-o-processo.htm} (in Portuguese). Access: December 2016.}

Second, they included representatives of all major \emph{jogo do bicho} bankers in every draw and independently publicise the game results. Certain \emph{bicheiros} went as far as publishing the numbers in Rio's newspapers. In the early twentieth century, some tabloids were entirely dedicated to the game \citep[60]{magalhaes2005ganhou}. Booking agents see this strategy as a credible signal from the game financiers, as providing contrasting information would indicate game manipulation. Moreover, collusion can also be spotted if the draws show repeated numbers or unusual patterns.

These efforts have proved popular with the game enthusiasts. One often-repeated saying about the \emph{jogo do bicho} is that `in the \emph{jogo do bicho}, what is written down counts' \citep[159]{chazkel2011laws}, that is, buyers and sellers do fulfill their informal obligations without third-party enforcement. Such mutual confidence reduces the potential for conflict in the game. As the public does not see the \emph{jogo do bicho} as violent or harmful, the stigma of repugnance associated with gambling becomes less pervasive. By reducing the possibilities of cheating and putting long-term interests first, the \emph{jogo do bicho} bankers have avoided the fate of other repugnant markets and run their business relatively undisturbed for decades \citep[20]{da1999aguias}.



\section{Concluding Remarks}
\label{sec:conclusion3}

Past research has shown that criminal organisations face considerable challenges to elicit cooperation from their members and establish close ties with the population \citep[e.g.][]{gambetta1996sicilian,skarbek2011governance,skarbek2012prison,varese2001russian,varese2011mafias}. Yet, the \textit{jogo do bicho} offers a convincing example that it is possible for an illegal syndicate to operate with low levels of violence for more than a hundred years. \textit{Bicheiros} employ a number of strategies to obtain reliable information from their subordinates while offering club goods and other selected benefits to workers. Furthermore, by investing in the Carnival parade \textit{bicheiros} have been able to gather popular and government support. Poor communities have associated with the \textit{bicheiros} to receive welfare provision, whereas politicians have collaborated with them to reap the financial and electoral benefits the \textit{jogo do bicho}'s networks can provide.

Nevertheless, the \textit{jogo do bicho} has also created negative externalities. Violence is used to punish defectors and to constrain competitors. The clientelistic relationship that \textit{bicheiros} have with local politicians have lead to sub-optimal outcomes, such as predatory political campaigning, distortions in electoral representation, and impunity for human rights violations. These negative externalities have long-term effects and still impact the Brazilian public sphere.

Although the \textit{jogo do bicho} has received an increasing attention from scholars, much of its inner workings remain poorly understood. First, the relationship between \textit{bicheiros} and drug dealers is a topic that deserves attention. Brazil has become one of the world's largest consumers of illicit drugs and South America's principal drug trafficking transit route \citep{miraglia2015drugs,misse2011crime}. The question whether \textit{bicheiros} collaborated or opposed the emergent drug dealing business is still unclear. Second, the extent to which \textit{bicheiros} use other businesses, such as hotels or factories, to laundry money has been mentioned by members of the Brazilian judiciary \citep{globo2012bicheiro,globo2015cacaniquel}; however, there is no reliable estimate on its size. Lastly, more research is required to clarify how \textit{bicheiros} from different parts of Brazil coordinate their activities and prevent large-scale conflicts. Cases studies are usually focused on Rio de Janeiro's \textit{bicheiros}, but scholars would benefit from comparative analyses with a larger number of states. This is an important step to elucidate how \textit{bicheiros} continue to influence politics and the public across Brazil.

\newpage
\bibliography{bibliography.bib}
\bibliographystyle{apalike}
\end{document}

